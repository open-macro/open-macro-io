%2multibyte Version: 5.50.0.2960 CodePage: 1252

\documentclass[11pt]{article}
%%%%%%%%%%%%%%%%%%%%%%%%%%%%%%%%%%%%%%%%%%%%%%%%%%%%%%%%%%%%%%%%%%%%%%%%%%%%%%%%%%%%%%%%%%%%%%%%%%%%%%%%%%%%%%%%%%%%%%%%%%%%%%%%%%%%%%%%%%%%%%%%%%%%%%%%%%%%%%%%%%%%%%%%%%%%%%%%%%%%%%%%%%%%%%%%%%%%%%%%%%%%%%%%%%%%%%%%%%%%%%%%%%%%%%%%%%%%%%%%%%%%%%%%%%%%
\usepackage{amssymb}
\usepackage{graphicx}
\usepackage{amsmath}
\usepackage{lineno}
\usepackage{fancybox}
\usepackage{fancyhdr}
\usepackage{rotating}
\usepackage{sectsty}
\usepackage[normalem]{ulem}
\usepackage{color}

\setcounter{MaxMatrixCols}{10}
%TCIDATA{OutputFilter=LATEX.DLL}
%TCIDATA{Version=5.50.0.2960}
%TCIDATA{Codepage=1252}
%TCIDATA{<META NAME="SaveForMode" CONTENT="1">}
%TCIDATA{BibliographyScheme=Manual}
%TCIDATA{Created=Wed Mar 08 17:14:21 2000}
%TCIDATA{LastRevised=Friday, April 17, 2015 16:36:16}
%TCIDATA{<META NAME="GraphicsSave" CONTENT="32">}
%TCIDATA{<META NAME="DocumentShell" CONTENT="General\Blank Document">}
%TCIDATA{Language=American English}
%TCIDATA{CSTFile=LaTeX article (bright).cst}
%TCIDATA{PageSetup=72,72,72,72,0}
%TCIDATA{Counters=arabic,2}
%TCIDATA{AllPages=
%H=36
%F=36,\PARA{038<p type="texpara" tag="Body Text" > \ \ \ \ \ \ \ \ \ \ \ \ \ \ \ \ \ \ \ \ \ \ \ \ \ \ \ \ \ \ \ \ \ \ \ \ \ \ \ \ \ \ \ \ \ \ \  \thepage }
%}


\newtheorem{theorem}{Theorem}
\newtheorem{acknowledgement}[theorem]{Acknowledgement}
\newtheorem{algorithm}[theorem]{Algorithm}
\newtheorem{axiom}[theorem]{Axiom}
\newtheorem{case}[theorem]{Case}
\newtheorem{claim}[theorem]{Claim}
\newtheorem{conclusion}[theorem]{Conclusion}
\newtheorem{condition}[theorem]{Condition}
\newtheorem{conjecture}[theorem]{Conjecture}
\newtheorem{corollary}[theorem]{Corollary}
\newtheorem{criterion}[theorem]{Criterion}
\newtheorem{definition}{Definition}
\newtheorem{example}[theorem]{Example}
\newtheorem{exercise}[theorem]{Exercise}
\newtheorem{lemma}{Lemma}
\newtheorem{notation}[theorem]{Notation}
\newtheorem{problem}[theorem]{Problem}
\newtheorem{proposition}{Proposition}
\newtheorem{remark}[theorem]{Remark}
\newtheorem{solution}[theorem]{Solution}
\newtheorem{summary}[theorem]{Summary}
\newenvironment{proof}[1][Proof]{\textbf{#1.} }{\ \rule{0.5em}{0.5em}}
\input{tcilatex}
\setlength{\oddsidemargin}{0in}
\setlength{\evensidemargin}{0in}
\setlength{\textwidth}{6.7in}
\setlength{\textheight}{9.in}
\setlength{\footskip}{0.5in}
\renewcommand{\topmargin}{-.54in}
\renewcommand{\baselinestretch}{1.1}
\addtolength{\parskip}{12pt}
\renewcommand{\thesubsection}{\Alph{subsection}.}
\renewcommand{\thesubsubsection}{\thesubsection\arabic{subsubsection}}
\renewcommand{\thesection}{\Roman{section}}
\renewcommand{\thesubsection}{\Alph{subsection}.}
\renewcommand{\thesubsubsection}{\thesubsection\arabic{subsubsection}.}
\pagestyle{fancy}
\fancyhf{}
\chead{\thepage}
\renewcommand{\headrulewidth}{0pt}
\renewcommand{\footrulewidth}{0pt}
\renewcommand{\thesection}{\Roman{section}.}
\makeatletter
\def\@biblabel#1{}
\renewenvironment{thebibliography}[1]
     {\section*{{\refname}
        \@mkboth{\refname}{\refname}}\small
      \list{\@biblabel{\@arabic\c@enumiv}}           {\settowidth\labelwidth{\@biblabel{#1}}           \leftmargin\bibindent
           \setlength{\itemindent}{-\leftmargin}
           \@openbib@code
           \usecounter{enumiv}           \let\p@enumiv\@empty
           \renewcommand\theenumiv{\@arabic\c@enumiv}}      \sloppy\clubpenalty4000\widowpenalty4000      \sfcode`\.\@m}
     {\def\@noitemerr
       {\@latex@warning{Empty `thebibliography' environment}}      \endlist}
\renewcommand\newblock{\hskip .11em\@plus.33em\@minus.07em}
\makeatother
\sectionfont{\centering}
\subsectionfont{\centering}
\subsubsectionfont{\centering}

\begin{document}

\title{\textbf{Current Account Norms in Natural Resource Rich and Capital
Scarce Economies}}
\date{March 20th, 2013}
\author{}
\maketitle

\begin{abstract}
The permanent income hypothesis implies that frictionless open economies
with exhaustible natural resources should save abroad most of their resource
windfalls and, therefore, feature current account surpluses. Resource-rich
developing countries (RRDCs), on the other hand, face substantial
development needs and tight external borrowing constraints. By relaxing
these constraints and providing a key financing source for public investment
in RRDCs, temporary resource revenues might then be associated with current
account deficits, or at least low surpluses. This paper develops a
neoclassical model with private and public investment and several frictions
that capture pervasive features in RRDCs, including absorptive capacity
constraints, inefficiencies in investment, and borrowing constraints that
can be relaxed when resources lower the country risk premium. Since
consumption and investment decisions are optimal, the model is used to
assess external sustainability in RRDCs by incorporating the role of
investment and these frictions in shaping the current account \textit{norm}.
We apply the model to the Economic and Monetary Community of Central Africa
and discuss how it can be used to benchmakr the current account in RRDCs.

\bigskip \bigskip \bigskip

%TCIMACRO{\TeXButton{noindent}{\noindent}}%
%BeginExpansion
\noindent%
%EndExpansion
\textbf{Keywords: }Current Account, External Sustainability, Developing
Economies

%TCIMACRO{\TeXButton{noindent}{\noindent}}%
%BeginExpansion
\noindent%
%EndExpansion
\textbf{JEL Classifications:} E21, F32, F41, O13.\medskip
\end{abstract}

%TCIMACRO{\TeXButton{thispagestyle}{\thispagestyle{empty}}}%
%BeginExpansion
\thispagestyle{empty}%
%EndExpansion
\pagebreak 
%TCIMACRO{%
%\TeXButton{restart page numbers}{\pagenumbering{arabic}
%\setcounter{page}{3}}}%
%BeginExpansion
\pagenumbering{arabic}
\setcounter{page}{3}%
%EndExpansion
\pagebreak

\section{The Model}

\quad\ We use a flexible-price model of a small open economy, but enrich it
with investment inefficiencies, absorptive capacity constraints and a
country risk premium that captures foreign credit constraints. Preferences
are identical accross countries. There is exogenous productivity growth at
the rate $\mathfrak{g}_{a}$ and population growth at the rate $\mathfrak{g}%
_{n}$, so in the long-run all the variables grow at the rate $\mathfrak{g}$,
where $(1+\mathfrak{g})=(1+\mathfrak{g}_{a})(1+\mathfrak{g}_{n})$. To
facilitate the description of the model, we present its structure in
stationary terms. This involves, when required, rescaling variables by the
effective units of labor $A_{t}L_{t}$, where $A_{t}$ is the productivity
level and and $L_{t}$ denotes labor. That is, $x_{t}\equiv \frac{X_{t}}{%
A_{t}L_{t}}$ for all the variables $X_{t}$. In this way all the transformed
variables are constant in the long run (steady state).

The economy is populated by a large number of identical and infinitely lived
households, who are endowed with perfect foresight. The representative agent
derives utility from private consumption $(c_{t})$ and public consumption $%
(g_{t})$, but not leisure, according to:

\begin{equation}
\sum_{t=0}^{\infty }\beta ^{t}\left[ \frac{\left( c_{t}-\varkappa
c_{t-1}\right) ^{1-\gamma }}{1-\gamma }+\kappa \frac{\left( g_{t}-\varkappa
g_{t-1}\right) ^{1-\gamma }}{1-\gamma }\right] .  \label{1}
\end{equation}%
The parameter $\beta $ equals $\mathcal{B}(1+\mathfrak{g}_{a})^{1-\gamma }(1+%
\mathfrak{g}_{n})^{^{1-\gamma }}A_{0}^{1-\gamma }L_{0}^{^{1-\gamma }}$ where 
$\mathcal{B}$ is the discount factor and satisfies $\mathcal{B}\in (0,1)$.
The coefficient of relative risk aversion is given by $\gamma $. The
parameter $\kappa $ controls the preference share for private and public
consumption and $\varkappa \in (0,1)$ denotes the intensity of internal
habit formation. We introduce habit formation to allow for a smooth path of
private consumption, as discussed by Christiano et al. (2005), and to avoid
unrealistically drastic adjustments in public consumption in the simulations.

The economy has two sectors: the non-oil sector ($n$) and the oil sector ($o$%
), whose outputs are denoted by $y_{t}^{n}$ and $y_{t}^{o},$ respectively.%
\footnote{%
The oil sector here represents a pure windfall in any exhautible resource
sector.} The production function in the non-oil sector is given by

\begin{equation}
y_{t}^{n}=ak_{t-1}^{\theta _{k}}s_{t-1}^{\theta _{s}},\qquad \text{with}%
\qquad \theta _{k}+\theta _{s}<1,  \label{2}
\end{equation}%
where $k_{t}$ and $s_{t}$ are private and public capital, respectively;
while, for simplicity oil production is exogenous.\footnote{%
Although important to explain real exchange rate movements, currently we
abstract from sectoral reallocations. This could be a relevant extension to
our model.}

We incorporate two types of investment frictions that capture inefficiencies
in investing and absorptive capacity constraints. As in Agenor (2010) and
Berg et al. (2013), among others, we assume that all public investment $%
i_{t}^{s}$ does not necessarily translate into productive public capital $%
s_{t}$. The public capital accumulation equation is

\begin{equation}
(1+\mathfrak{g})s_{t+1}=e_{s}i_{t}^{s}+(1-\delta _{s})s_{t},\text{ \ \ \ \ \
\ \ with \ \ \ \ \ \ }e_{s}\in \lbrack 0,1]  \label{3}
\end{equation}%
where $\delta _{s}$ is the depreciation of public capital and the efficiency
parameter $e_{s}$ captures the idea that one dollar spent on public
investment may translate into less than one dollar of productive public
capital. The traditional \textquotedblleft perpetual inventory
method\textquotedblright\ usually imposes $e_{s}=1$ and then uses this
equation to infer the stock of public capital from information on public
investment and assumptions about depreciation rates. However, assuming full
efficiency is problematic, particularly in developing economies. Whether
because of waste and corruption, an absence of market pressures to ensure
that all projects have the highest possible rate of return, or simply
misclassification of current spending (e.g. salary payments to civil
servants) as investment, a dollar of public investment spending may not
always yields a full dollar of public capital, as argued by Pritchett
(2000). Similar inefficiencies exist in the creation of private capital,
when investing the amount $i_{t}^{k}$. Therefore

\begin{equation}
(1+\mathfrak{g})k_{t+1}=e_{k}i_{t}^{k}+(1-\delta _{k})k_{t}.  \label{4}
\end{equation}

We model absorptive capacity constraints as investment adjustment costs that
only play a role off steady state. These costs take the form of

\begin{equation}
AC_{t}^{s}=\frac{\phi _{s}}{2}\left( \frac{s_{t}}{s_{t-1}}-1\right)
^{2}s_{t-1}\text{ \ \ and \ \ }AC_{t}^{k}=\frac{\phi _{k}}{2}\left( \frac{%
k_{t}}{k_{t-1}}-1\right) ^{2}k_{t-1}.  \label{AC}
\end{equation}%
As in Buffie et al. (2012), these reflect the fact that skilled
administrators are in scarce supply in RRDCs and, therefore, ambitious
public and private investment programs are often plagued by poor planning,
weak oversight, and a myriad of coordination problems, all of which
contribute to costs which can increase with the pace of scaling up.\footnote{%
Development agencies report that cost overruns of 35\% and more are common
for new projects in Africa. The most important factor by far is inadequate
competitive bidding for tendered contracts. See Foster and Briceno-Garmendia
(2010), Lledo and Poplawski-Ribeiro (2013), and Guerguil et al. (2014),
among others.} The parameters $\phi _{k}$ and $\phi _{s}$ determine the
severity of these absorptive capacity constraints.

Developing economies are also characterized by their inability to fully
access international capital market, because of borrowing constraints. One
might think that at each period $t$, foreign lenders impose an aggregate
borrowing limit on the domestic economy. This can stipulate that the level
of external liabilities $d_{t}$ must satisfy $d_{t}\leq \bar{d}+\psi V_{t},$
where $\bar{d}$ is an exogenous limit on debt, $V_{t}=\sum_{i=t}^{T}\left( 
\frac{1}{1+r^{\ast }}\right) ^{i-t}y_{i}^{o}$ is the net present value (NPV)
of the oil output flows from time $t$ until the depletion time $T,$ $r^{\ast
}$ is the risk-free world interest rate, and $\psi \in \lbrack 0,1]$. Note
that this means that resource revenues ($\psi V_{t}$) can help relax
borrowing constraints enabling the country to contract debt beyond the limit 
$\bar{d}$ at a lower cost, as argued by Mansoorian (1991) and empirically
evinced by Arezki and Br\"{u}ckner (2012) in democratic emerging economies.

To capture these borrowing constraints, important for a better benchmarking
of the current account in RRDCs, we assume the country faces an interest
rate with a country risk premium that depends on the stock of its external
liabilities. In particular, we assume that the premium ($r_{t}-r^{\ast }$)
can be represented by

\begin{equation}
\Pi (d_{t})=r_{t}-r^{\ast }=\frac{\pi }{\rho _{1}^{2}}\left[ e^{\rho
_{1}(d_{t}-\bar{d}-\psi V_{t})}-\rho _{2}(d_{t}-\overline{d}-\psi
V_{t})-\rho _{3}\right] ,  \label{5}
\end{equation}%
where $\pi ,\rho _{1},\rho _{2}>0,$ $\rho _{3}=e^{-\rho _{1}\overline{d}%
}+\rho _{2}\overline{d}$ and $\overline{d}$ is the steady-state level of
debt.\footnote{%
This specification is borrowed from Kim and Ruge-Murcia (2009), who use this
functional form to model asymmetric nominal wage adjusment costs.} This
representation combines an exponential with a linear function. For low
values of $\rho _{2}$, as in the calibration below, the exponential form
drives the premium for highly indebted countries; while the linear form
becomes the main driver of the premium for creditor countries---i.e.,
creditors will actually face almost no premium as the supply curve of funds
becomes flatter.

The country risk premium specification, which depends on debt, serves
several purposes. First, although our motivation is to capture occasionally
binding credit constraints, this specification helps us get around the
highly complex and technical issues related to having inequality constraints
in dynamic optimization.\footnote{%
This is in the same spirit of the recent literature on incomplete markets
and heterogenous agents models. In this literature, the problem of
maximizing an objective function subject to an inequality constraint is
replaced with an unconstrained maximization problem, whose objective
function or budget constraint include a penalty function that tries to
capture the effects of the inequality constraint. This approach allows the
use of perturbation methods to simulate these models. See Preston and Roca
(2007) and Algan et al. (2010), among others.} Second, it ensures
stationarity of foreign debt holdings, as explained by Schmitt-Grohe and
Uribe (2003). Third, it allows us to model different degrees of
international capital mobility. The degree of the capital account openness
depends to a great extent on the composite parameter $\frac{\pi }{\rho
_{1}^{2}}.$ For very small values of this ratio, the capital account is in
effect fully open reflecting perfect international capital markets. For very
high values, on the other hand, the capital account is almost fully closed.
Last, including the value of oil wealth $V_{t}$ in the specification helps
reduce the risk premium and the interest rate paid on debt. This in turn
creates incentives to borrow more, which is in line with some of the
empirical facts discussed by Manzano and Rigobon (2007).

The current account can be expressed as%
\begin{equation}
ca_{t}=d_{t-1}-(1+\mathfrak{g})d_{t},  \label{CA}
\end{equation}%
while the resource constraint of the economy corresponds to

\begin{equation}
(1+\mathfrak{g}%
)d_{t}=(1+r_{t-1})d_{t-1}+c_{t}+i_{t}^{k}+AC_{t}^{k}+g_{t}+i_{t}^{s}+AC_{t}^{s}-y_{t}^{n}-y_{t}^{o}-T_{t},
\label{6}
\end{equation}%
where $T_{t}$ denotes exogenous transfers to the economy not related to
natural resources.

We assume that there is a social planner who chooses the sequences for
consumption, private and public capital stock, private and public
investment, and borrowing $%
\{c_{t},g_{t},i_{t}^{s},i_{t}^{k},s_{t},k_{t},d_{t}\}_{t=0}^{\infty }$ to
maximize (\ref{1}) subject to (\ref{2})-(\ref{6}), given $k_{0},s_{0},d_{0}$
and the exogenous path for $\{y_{t}^{o}\}_{t=0}^{\infty }$ and $%
\{T_{t}\}_{t=0}^{\infty }$.\footnote{%
Since we want to derive current account benchmarks with optimality content,
we focus on the social planner problem, where the government takes optimally
both private and public decisions. By doing this, the government
internalizes and, therefore, mitigates the negative effect that fiscal
policies may have on the private sector. To some extent, this explains why
domestic borrowing is ruled out: although domestic borrowing can be a
financing source for public investment, it can substantially crowd out the
private sector and, as a result, is dominated by external borrowing (see
Buffie et al. 2012).} The first order conditions of this problem presented
in the Appendix can be reduced to\footnote{%
Transversality conditions on $s_{t},$ $k_{t},$ and $d_{t}$ are also imposed.}

\begin{equation}
\hat{c}_{t}^{-\gamma }-\varkappa (1+\mathfrak{g})\beta \hat{c}%
_{t+1}^{-\gamma }=\beta \left[ 1+r^{\ast }+\Pi (d_{t})+\Pi ^{\prime
}(d_{t})d_{t}\right] \left[ \hat{c}_{t+1}^{-\gamma }-\varkappa (1+\mathfrak{g%
})\beta \hat{c}_{t+2}^{-\gamma }\right]  \label{EE1}
\end{equation}

\begin{equation}
\hat{c}_{t}^{-\gamma }-\varkappa (1+\mathfrak{g})\beta \hat{c}%
_{t+1}^{-\gamma }=\kappa \lbrack \hat{g}_{t}^{-\gamma }-\varkappa (1+%
\mathfrak{g})\beta \hat{g}_{t+1}^{-\gamma }]  \label{EE2}
\end{equation}

\begin{equation}
\frac{\hat{c}_{t}^{-\gamma }-\varkappa (1+\mathfrak{g})\beta \hat{c}%
_{t+1}^{-\gamma }}{\hat{c}_{t+1}^{-\gamma }-\varkappa (1+\mathfrak{g})\beta 
\hat{c}_{t+2}^{-\gamma }}=\beta (1+\mathfrak{g})\frac{\left[ e_{k}\theta _{k}%
\frac{y_{t+1}^{n}}{k_{t}}+(1-\delta _{k})-e_{k}\frac{\phi _{k}}{2}\left( 
\frac{k_{t+1}}{k_{t}}-1\right) ^{2}+e_{k}\phi _{k}\left( \frac{k_{t+1}}{k_{t}%
}-1\right) \frac{k_{t+1}}{k_{t}}\right] }{\left[ (1+\mathfrak{g})+e_{k}\phi
_{k}\left( \frac{k_{t}}{k_{t-1}}-1\right) \right] },  \label{EE3}
\end{equation}

\begin{equation}
\frac{\hat{c}_{t}^{-\gamma }-\varkappa (1+\mathfrak{g})\beta \hat{c}%
_{t+1}^{-\gamma }}{\hat{c}_{t+1}^{-\gamma }-\varkappa (1+\mathfrak{g})\beta 
\hat{c}_{t+2}^{-\gamma }}=\beta (1+\mathfrak{g})\frac{\left[ e_{s}\theta _{s}%
\frac{y_{t+1}^{n}}{s_{t}}+(1-\delta _{s})-e_{s}\frac{\phi _{s}}{2}\left( 
\frac{s_{t+1}}{s_{t}}-1\right) ^{2}+e_{s}\phi _{s}\left( \frac{s_{t+1}}{s_{t}%
}-1\right) \frac{s_{t+1}}{s_{t}}\right] }{\left[ (1+\mathfrak{g})+e_{s}\phi
_{s}\left( \frac{s_{t}}{s_{t-1}}-1\right) \right] }.  \label{EE4}
\end{equation}

The interpretation of these conditions is straightforward. Condition (\ref%
{EE1}) is the Euler equation for private consumption $c_{t}$ including the
effects of internal habits, since $\hat{c}_{t}=c_{t}-\varkappa c_{t-1}$.
Equation (\ref{EE2}) equates the marginal utility of private and public
consumption, where $\hat{g}_{t}=g_{t}-\varkappa g_{t-1}$. Conditions (\ref%
{EE3}) and (\ref{EE4}) set optimal private and public investment by equating
the marginal cost and benefit of postponing consumption one period ahead.
Note that we have assumed that the social planner internalizes the effect of
more borrowing on the country risk premium and, therefore, on the cost of
debt $1+r_{t}$---this explains the term $\Pi ^{\prime }(d_{t})d_{t}$. This
is consistent with the view that the social planner will use \textit{marginal%
} borrowing decisions to affect the marginal increase of the cost of debt.
Moreover, as is common in the literature of capital adjustment costs, the
planner internalizes the effect of more investment on the absorptive
capacity constraints.

We provide now a definition of equilibrium in this open economy model.

\begin{definition}
\textit{Given }$k_{0},$ $s_{0},$ and $d_{0},$ and \textit{the sequences }$%
\{y_{t}^{o}\}_{t=0}^{\infty }$ and $\{T_{t}\}_{t=0}^{\infty },$ an
equilibrium\textit{\ is a set of sequences }$%
\{c_{t},g_{t},i_{t}^{s},i_{t}^{k},s_{t},k_{t},d_{t},y_{t}^{n},AC_{t}^{s},AC_{t}^{k},r_{t},ca_{t}\}_{t=0}^{\infty } 
$\textit{\ satisfying equations (\ref{2})-(\ref{6}) and the first order
conditions (\ref{EE1})-(\ref{EE4}). }
\end{definition}

\section{Appendix}

Letting $\eta _{t}^{s}$ , $\eta _{t}^{k}$ and $\lambda _{t}$ be the
Lagrangian multipliers on equations (\ref{3}), (\ref{4}) and (\ref{6})
respectively, the first order conditions are given by:

\begin{equation}
c_{t}:[c_{t}-\varkappa c_{t-1}^{{}}]^{-\gamma }-\varkappa \beta (1+\mathfrak{%
g})[c_{t+1}-\varkappa c_{t}]^{-\gamma }=\lambda _{t},  \label{A1}
\end{equation}

\begin{equation}
g_{t}:[g_{t}-\varkappa g_{t-1}^{{}}]^{-\gamma }-\varkappa \beta (1+\mathfrak{%
g})[g_{t+1}-\varkappa g_{t}]^{-\gamma }=\lambda _{t},
\end{equation}

\begin{equation}
i_{t}^{k}:e_{k}\eta _{t}^{k}=\lambda _{t},  \label{A3}
\end{equation}

\begin{equation}
i_{t}^{s}:e_{s}\eta _{t}^{s}=\lambda _{t},  \label{A4}
\end{equation}

\begin{equation}
d_{t}:\lambda _{t}=\beta \lambda _{t+1}\left[ 1+r^{\ast }+\frac{\pi }{\rho
_{1}^{2}}\left[ e_{{}}^{\rho _{1}(d_{t}-\bar{d}-\psi V_{t})}-\rho _{2}(d_{t}-%
\overline{d}-\psi V_{t})-\rho _{3}\right] +\frac{\pi }{\rho _{1}^{2}}\left[
\rho _{1}e_{{}}^{\rho _{1}(d_{t}-\bar{d}-\psi V_{t})}-\rho _{2}\right] d_{t}%
\right] ,  \label{A5}
\end{equation}

\begin{equation}
k_{t}:\lambda _{t}=\beta (1+\mathfrak{g})\lambda _{t+1}\frac{\left[
e_{k}\theta _{k}\frac{y_{t+1}^{n}}{k_{t}}+(1-\delta _{k})-e_{k}\frac{\phi
_{k}}{2}\left( \frac{k_{t+1}}{k_{t}}-1\right) ^{2}+e_{k}\phi _{k}\left( 
\frac{k_{t+1}}{k_{t}}-1\right) \frac{k_{t+1}}{k_{t}}\right] }{\left[ (1+%
\mathfrak{g})+e_{k}\phi _{k}\left( \frac{k_{t}}{k_{t-1}}-1\right) \right] },
\label{A6}
\end{equation}

\begin{equation}
s_{t}:\lambda _{t}=\beta (1+\mathfrak{g})\lambda _{t+1}\frac{\left[
e_{s}\theta _{s}\frac{y_{t+1}^{n}}{s_{t}}+(1-\delta _{s})-e_{s}\frac{\phi
_{s}}{2}\left( \frac{s_{t+1}}{s_{t}}-1\right) ^{2}+e_{s}\phi _{s}\left( 
\frac{s_{t+1}}{s_{t}}-1\right) \frac{s_{t+1}}{s_{t}}\right] }{\left[ (1+%
\mathfrak{g})+e_{s}\phi _{s}\left( \frac{s_{t}}{s_{t-1}}-1\right) \right] },
\label{A7}
\end{equation}

\begin{equation}
\eta _{t}^{s}:(1+\mathfrak{g})s_{t+1}=e_{s}i_{t}^{s}+(1-\delta _{s})s_{t},
\label{A8}
\end{equation}

\begin{equation}
\eta _{t}^{k}:(1+\mathfrak{g})k_{t+1}=e_{k}i_{t}^{k}+(1-\delta _{k})k_{t},
\label{A9}
\end{equation}

\begin{equation}
\lambda _{t}:(1+\mathfrak{g}%
)d_{t}=(1+r_{t-1})d_{t-1}+c_{t}+i_{t}^{k}+AC_{t}^{k}+g_{t}+i_{t}^{s}+AC_{t}^{s}-y_{t}^{n}-y_{t}^{o}-T_{t}.
\label{A10}
\end{equation}

At steady-state, we find that:

\begin{equation}
\lbrack c(1-\varkappa )]^{-\gamma }[1-\varkappa \beta (1+\mathfrak{g}%
)]=\lambda ,  \label{SS1}
\end{equation}

\begin{equation}
\kappa \lbrack g(1-\varkappa )]^{-\gamma }[1-\varkappa \beta (1+\mathfrak{g}%
)]=\lambda ,  \label{SS2}
\end{equation}

\begin{equation}
e_{k}\eta ^{k}=\lambda ,  \label{SS3}
\end{equation}

\begin{equation}
e_{s}\eta ^{s}=\lambda ,  \label{SS4}
\end{equation}

\begin{equation}
1=\beta \left[ 1+r^{\ast }+\frac{\pi }{\rho _{1}^{2}}[1-\rho _{3}+(\rho
_{1}-\rho _{2})d]\right] ,  \label{SS5}
\end{equation}

\begin{equation}
1=\beta \left[ e_{k}\theta _{k}\frac{y^{n}}{k}+(1-\delta _{k})\right] ,
\label{SS6}
\end{equation}

\begin{equation}
1=\beta \left[ e_{s}\theta _{s}\frac{y^{n}}{s}+(1-\delta _{s})\right] ,
\label{SS7}
\end{equation}

\begin{equation}
s(\mathfrak{g}+\delta _{s})=e_{s}i^{s},  \label{SS8}
\end{equation}

\begin{equation}
k(\mathfrak{g}+\delta _{k})=e_{k}i^{k},  \label{SS9}
\end{equation}

\begin{equation}
(\mathfrak{g}-r)d+y^{n}+y^{o}+T=c+i^{k}+g+i^{s}.  \label{SS10}
\end{equation}

Combining (\ref{SS1}) and (\ref{SS2}):

\begin{equation}
\kappa =\left( \frac{g}{c}\right) ^{\gamma }  \label{SS11}
\end{equation}

\end{document}
