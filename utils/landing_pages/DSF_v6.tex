
\documentclass[11pt]{article}
%%%%%%%%%%%%%%%%%%%%%%%%%%%%%%%%%%%%%%%%%%%%%%%%%%%%%%%%%%%%%%%%%%%%%%%%%%%%%%%%%%%%%%%%%%%%%%%%%%%%%%%%%%%%%%%%%%%%%%%%%%%%%%%%%%%%%%%%%%%%%%%%%%%%%%%%%%%%%%%%%%%%%%%%%%%%%%%%%%%%%%%%%%%%%%%%%%%%%%%%%%%%%%%%%%%%%%%%%%%%%%%%%%%%%%%%%%%%%%%%%%%%%%%%%%%%
\usepackage{amsfonts}
\usepackage{amssymb}
\usepackage{graphicx}
\usepackage{amsmath}
\usepackage{lineno}
\usepackage{fancybox}
\usepackage{rotating}
\usepackage{fancyhdr}
\usepackage{sectsty}

\setcounter{MaxMatrixCols}{10}
%TCIDATA{OutputFilter=LATEX.DLL}
%TCIDATA{Version=5.50.0.2953}
%TCIDATA{<META NAME="SaveForMode" CONTENT="1">}
%TCIDATA{BibliographyScheme=Manual}
%TCIDATA{Created=Wed Mar 08 17:14:21 2000}
%TCIDATA{LastRevised=Sunday, November 18, 2012 11:18:57}
%TCIDATA{<META NAME="GraphicsSave" CONTENT="32">}
%TCIDATA{<META NAME="DocumentShell" CONTENT="General\Blank Document">}
%TCIDATA{Language=American English}
%TCIDATA{CSTFile=LaTeX article (bright).cst}
%TCIDATA{PageSetup=72,72,72,72,0}
%TCIDATA{Counters=arabic,2}
%TCIDATA{AllPages=
%H=72,\PARA{038<p type="texpara" tag="Body Text" >Draft: Plesase do not distribute}
%F=36,\PARA{038<p type="texpara" tag="Body Text" > \ \ \ \ \ \ \ \ \ \ \ \ \ \ \ \ \ \ \ \ \ \ \ \ \ \ \ \ \ \ \ \ \ \ \ \ \ \ \ \ \ \ \ \ \ \ \  \thepage }
%}


\newtheorem{theorem}{Theorem}
\newtheorem{acknowledgement}[theorem]{Acknowledgement}
\newtheorem{algorithm}[theorem]{Algorithm}
\newtheorem{axiom}[theorem]{Axiom}
\newtheorem{case}[theorem]{Case}
\newtheorem{claim}[theorem]{Claim}
\newtheorem{conclusion}[theorem]{Conclusion}
\newtheorem{condition}[theorem]{Condition}
\newtheorem{conjecture}[theorem]{Conjecture}
\newtheorem{corollary}[theorem]{Corollary}
\newtheorem{criterion}[theorem]{Criterion}
\newtheorem{definition}{Definition}
\newtheorem{example}[theorem]{Example}
\newtheorem{exercise}[theorem]{Exercise}
\newtheorem{lemma}{Lemma}
\newtheorem{notation}[theorem]{Notation}
\newtheorem{problem}[theorem]{Problem}
\newtheorem{proposition}{Proposition}
\newtheorem{remark}[theorem]{Remark}
\newtheorem{solution}[theorem]{Solution}
\newtheorem{summary}[theorem]{Summary}
\newenvironment{proof}[1][Proof]{\textbf{#1.} }{\ \rule{0.5em}{0.5em}}
\input{tcilatex} %This is used by scientific word
\setlength{\oddsidemargin}{0in}
\setlength{\evensidemargin}{0in}
\setlength{\textwidth}{6.7in}
\setlength{\textheight}{9.in}
\setlength{\footskip}{0.5in}
\renewcommand{\topmargin}{-.54in}
\renewcommand{\baselinestretch}{1.1}
\addtolength{\parskip}{12pt}
\renewcommand{\thesubsection}{\Alph{subsection}.}
\renewcommand{\thesubsubsection}{\thesubsection\arabic{subsubsection}}
\renewcommand{\thesection}{\Roman{section}}
\renewcommand{\thesubsection}{\Alph{subsection}.}
\renewcommand{\thesubsubsection}{\thesubsection\arabic{subsubsection}.}
\pagestyle{fancy}
\fancyhf{}
\chead{\thepage}
\renewcommand{\headrulewidth}{0pt}
\renewcommand{\footrulewidth}{0pt}
\renewcommand{\thesection}{\Roman{section}.}
\makeatletter
\def\@biblabel#1{}
\renewenvironment{thebibliography}[1]
     {\section*{{\refname}
        \@mkboth{\refname}{\refname}}\small
      \list{\@biblabel{\@arabic\c@enumiv}}           {\settowidth\labelwidth{\@biblabel{#1}}           \leftmargin\bibindent
           \setlength{\itemindent}{-\leftmargin}
           \@openbib@code
           \usecounter{enumiv}           \let\p@enumiv\@empty
           \renewcommand\theenumiv{\@arabic\c@enumiv}}      \sloppy\clubpenalty4000\widowpenalty4000      \sfcode`\.\@m}
     {\def\@noitemerr
       {\@latex@warning{Empty `thebibliography' environment}}      \endlist}
\renewcommand\newblock{\hskip .11em\@plus.33em\@minus.07em}
\makeatother
\sectionfont{\centering}
\subsectionfont{\centering}
\subsubsectionfont{\centering}

\begin{document}

\author{Edward F. Buffie\thanks{\baselineskip=11ptDepartment of Economics,
Indiana University, Wylie Hall Rm 105, 100 S. Woodlawn, Bloomington, IN
47405 Email: ebuffie@indiana.edu.}, Andrew Berg, Catherine Pattillo, Rafael
Portillo, Luis-Felipe Zanna\thanks{\baselineskip=11ptInternational Monetary
Fund, 700 19th Street, N.W., Washington, D.C., 20431. Email: aberg@imf.org,
cpattillo@imf.org, rportilloocando@imf.org, fzanna@imf.org.}}
\title{{\Large \textbf{Public Investment, Growth, and Debt Sustainability:
Putting Together the Pieces }}\thanks{\baselineskip=11pt We thank Valerio
Crispolti, Raphael Espinoza, Giovanni Ganelli, Andrew Jewell, Alvar Kangur,
Chris Papageorgiou, Jens Reinke, Carlo Sdralevich, Susan Yang, and
participants of the MMDG seminar of the African department at the IMF, the
2011 AERC/UNU-WIDER Macroeconomics of Foreign Aid Meeting, and the 2012 CSAE
conference in Oxford for useful comments. All errors remain ours. This
working paper is part of a research project on macroeconomic policy in
low-income countries supported by the U.K.'s Department for International
Development. }}
\date{This Draft: May10, 2012 }
\maketitle

\begin{abstract}
\baselineskip=15pt We develop a model to study the macroeconomic effects of
public investment surges in low-income countries, making explicit: (i) the
investment-growth linkages; (ii) public external and domestic debt
accumulation; (iii) the fiscal policy reactions necessary to ensure
debt-sustainability; and (iv) the macroeconomic adjustment required to
ensure internal and external balance.

Well-executed high-yielding public investment programs can substantially
raise output and consumption and be self-financing in the long run. However,
even if the long run looks good, transition problems can be formidable when
concessional financing does not cover the full cost of the investment
program. Covering the resulting gap with tax increases or spending cuts
requires sharp macroeconomic adjustments, crowding out private investment
and consumption and delaying the growth benefits of public investment.
Covering the gap with domestic borrowing market is not helpful either:
higher domestic rates increase the financing challenge and private
investment and consumption are still crowded out. Supplementing with
external commercial borrowing, on the other hand, can smooth these difficult
adjustments, reconciling the scaling up with feasibility constraints on
increases in tax rates. But the strategy may be also risky. With poor
execution, sluggish fiscal policy reactions, or persistent negative
exogenous shocks, this strategy can easily lead to unsustainable public debt
dynamics. Front-loaded investment programs and weak structural conditions
(such as low returns to public capital and poor execution of investments)
make the fiscal adjustment more challenging and the risks greater.\bigskip

%TCIMACRO{\TeXButton{noindent}{\noindent}}%
%BeginExpansion
\noindent%
%EndExpansion
\textbf{Keywords: }Public Investment, Growth, Debt Sustainability, Fiscal
Policy, Infrastructure, Aid.

%TCIMACRO{\TeXButton{noindent}{\noindent}}%
%BeginExpansion
\noindent%
%EndExpansion
\textbf{JEL Classifications:} E62, F34, H63, O43, H54.\medskip
\end{abstract}

%TCIMACRO{%
%\TeXButton{restart page numbers}{\pagenumbering{arabic}
%\setcounter{page}{3}}}%
%BeginExpansion
\pagenumbering{arabic}
\setcounter{page}{3}%
%EndExpansion

\section{Introduction}

\quad\ \thinspace Many low-income countries (LICs) are facing dire
infrastructure gaps.\footnote{%
See Foster and Brice\~{n}o-Garmendia (2010).} For the first time in decades,
many also have substantial growth momentum, low debt levels, and access to
nonconcessional foreign credit. Meanwhile, aid resources are not increasing
as promised.\footnote{%
For example, Redifer (2010) points out that the four East African countries
with new Fund's Policy Support Instrument Programs (Mozambique, Rwanda,
Tanzania, and Uganda), official aid has on average not increased in line
with public investment spending and is not projected to do so in the next
three years; therefore new financing sources, such as external commercial
borrowing, must be tapped if public investment is to be scaled up.} The
opportunity to borrow nonconcessionally to meet infrastructure needs has
thus become very tempting.\footnote{%
A recent survey by Citigroup describes the new borrowing environment:
\textquotedblleft Knowing they want to borrow money to spend on projects to
close the infrastructure deficit, governments in SSA have faced a wave of
lenders looking to get money out of the door and into their pockets: whether
investment bankers extolling the virtues of issuing Eurobonds; apparently
cheap BRIC country loans, but with long-term catches on payment
implications\ .\ .\ .\textquotedblright\ (Cowan, 2010, p.9). A call to
borrow for development neeeds is endorsed by UNCTAD (2004) and EURODAD
(2001, 2009), in the context of the \textit{human development approach} to
debt sustainability.}

The risks associated with excessive borrowing to finance these public
investment plans need to be considered as well. The Highly Indebted Poor
Country (HIPC) and Multilateral Debt Relief initiatives (MDRI) reduced the
external debt of the poorest countries by ninety percent over the past
decade. This followed decades of struggle by these countries to work their
way out from onerous debt levels, which were mostly accumulated in the
1970s, the last time many of them were growing rapidly and had access to new
foreign lending on a large scale. Debt indicators in LICs are still far
below the levels seen in the mid-90s. However, rapid accumulation of new
debt in these countries---especially the increasing reliance on more
expensive domestic and external commercial debt---could bring back the
spectre of debt crisis, macroeconomic instability, and severely impaired
development prospects.\footnote{%
See IMF and World Bank (2006) and Barkbu et al. (2008), among others.}

The IMF and World Bank (IMF-WB) use a debt sustainability framework (DSF) to
identify overborrowing situations that may endanger macroeconomic stability.
In the DSF, a baseline set of 20-year projections for borrowing, GDP growth,
exports, and other key macroeconomic variables underpin an analysis of key
debt ratios. In this debt sustainability analysis (DSA), a country is at
\textquotedblleft high risk\textquotedblright\ of debt distress if any of
the debt ratios---such as debt/GDP and debt/exports---exceed a specified
threshold in the baseline scenario over the 20-year horizon. The thresholds
in turn have been determined based on empirical evidence linking these
ratios to subsequent episodes of debt distress.\footnote{%
The thresholds depend on the quality of policies and institutes as measured
by the Country Policy and Institutional Assessment (CPIA) index of the World
Bank.} A country is at \textquotedblleft moderate risk\textquotedblright\ if
the ratios exceed the thresholds in one of several specified alternative
scenarios or \textquotedblleft stress tests\textquotedblright\ that simulate
negative growth shocks and nominal exchange rate shocks, among others.%
\footnote{%
See IMF(2010) for the most recent guidelines on the application of the DSF
to LICs.} But, of course, judgement is used by staff when assigning risk
ratings.

The DSF has helped countries monitor their risk of debt distress and
sharpened the IMF-WB's assessments and policy advice, but it has been also
subject to several criticisms.

One of the criticisms is that the DSF does not contain a consistent analytic
framework for creating the 20-year projections. Sachs (2002), for instance,
argues that \textquotedblleft the so-called debt sustainability analysis is
built on the flimsiest of foundations,\textquotedblright \thinspace\ arguing
that it is little more than a set of accounting identities and exogenous
projections. In the same vein, Eaton (2002) and Hjertholm's (2003) have
raised concerns that IMF-WB's debt projections are not derived from an
integrated, internally consistent macroeconomic framework. A specific aspect
of this criticism is that the IMF-WB projections do not take sufficiently
into account the relationship between public investment and growth. That is,
the projections generally do not make an explicit linkage between the public
investment that the proposed nonconcessional borrowing is meant to finance
and the resulting growth that should make the operation self-financing. This
inflates debt indicators, such as debt-to-GDP ratios, creates a bias toward
conservative borrowing limits, and can amount \textquotedblleft to
sacrificing growth to imprecisely known debt sustainability
risks\textquotedblright\ (Wyplosz, 2007).

Another concern relates to the treatment of fiscal policy in the
forward-looking framework. The DSF concept of solvency requires that debt
stay below the thresholds absent a \textquotedblleft major
correction\textquotedblright\ in policies. It cannot, in the view of Wyplosz
(2007), be invoked to argue that a particular level of debt, or even a
prolonged rising path, signifies debt distress. After all, even a very good
project may take some time to pay off, and in the interim the debt ratios
may exceed a given threshold. Moreover, the stress tests assume that the
government does not react to shocks, contradicting evidence that primary
balances in fact respond to rising public debt and potentially making the
tests too conservative.\footnote{%
For this evidence see Celasun et al. (2007), among others. In the current
review of the DSF, Fund and World Bank staff are proposing the inclusion, on
an optional basis, of a new stress test reflecting dynamic linkages between
macroeconomic variables.}

This paper addresses these criticisms by proposing an internally consistent
quantitative macroeconomic framework that may be useful in constructing the
scenarios necessary for debt sustainability analysis. The model has many
LIC-specific components, but it is centered on the public investment-growth
nexus.\footnote{%
The Fund acknowledges the need for strengthening analysis of the
investment/growth nexus in DSAs, including through development and
operationalizing models to provide a consistent way to assess the complex
interlinkages (IMF and World Bank, 2009).} In the end, judgement will be
critical in making projections and scenarios for the DSF, whether through
purely ad hoc forecasts or the careful calibration of the model. But the
model should serve to: make explicit the assumptions underlying the
projections, furthering discussions internally and with stakeholders based
on different simulated scenarios; help apply empirical information, for
example on project rates of return; and allow more systematic risk
assessments.

In putting the model through its paces to analyze debt-led public investment
scaling ups in a typical Sub-Saharan African (SSA) LIC, the paper
demonstrates the importance of a coherent forward-looking analysis, with
explicit policy reaction functions that may respond to debt levels. An
overarching conclusion regarding the debt sustainability impact of ambitious
public investment plans is that it is \textit{not} enough to compare the
rate of return of these plans to their cost of funding.\footnote{%
Thus we disagree with Wyplosz (2007): \textquotedblleft If external
borrowing is growth enhancing, the risk of over borrowing is small, possibly
inexistent. If, instead, external borrowing does not exert any favorable
growth effect, and possibly stunts growth, DSA is moot . . .
(p.14)\textquotedblright .} Rather, the absorptive capacity of the country,
the efficiency of public investment spending, the response of the private
sector, the authorities' ability to adjust taxes and spending, and other
factors shape the benefits---and the debt sustainability risks---of these
investment plans.

A further criticism is that the DSF-based debt limits policy is not flexible
enough in allowing countries with IMF-supported programs to borrow
nonconcessionally. In response, the IMF has recently made its policies more
flexible.\footnote{%
The IMF has recently modified its policies on nonconcessional borrowing by
LICs in the context of IMF-supported programs to reflect better the
diversity of LICs and their financing patterns, and offer more flexibility
depending on countries' debt vulnerabilities and public financial management
capacity. See IMF(2009b) for the guidelines on debt limits in Fund-supported
programs.} The question of how to move from an analysis of debt
sustainability to the application of borrowing limits in IMF-supported
programs is outside the scope of this paper. Rather, the sole focus here is
on the macroeconomic framework---the projections and scenarios---underlying
the DSF. However, we hope that the availability of more coherent medium-term
framework, incorporating public investment/growth linkages, will allow
better analysis and application of borrowing limits in difficult cases.%
\footnote{%
The IMF and World Bank Boards have stated that until the investment-growth
nexus is incorporated concerns will persist that \textquotedblleft the DSF
has unduly constrained the ability of LICs to finance their development
goals.\textquotedblright See IMF and World Bank (2009).}

With these goals in mind, we construct an optimizing intertemporal model
that embeds features that seem crucial to capturing the main mechanisms and
policy issues of interest for DSAs in LICs.

The model incorporates a neoclassical production function with private and
public capital. Because public capital is productive, government spending
can raise output directly and crowd in as well as crowd out private
investment. The parameters of the production function determine the rate of
return to installed public capital.

Several distinct features capture aspects of the challenges LIC governments
have faced historically in making productive public investments. First,
spending on public investment does not always imply an equivalent increase
in the stock of public capital. Depending on the \textquotedblleft
efficiency\textquotedblright\ of public investment, some of the spending may
be wasted or spent on poor (inframarginal) projects. In addition, we assume
an \textquotedblleft absorptive capacity\textquotedblright\ problem: due to
coordination problems or supply bottlenecks during the implementation phase
of public investment projects, unusually high investment rates may result in
large costs overruns that affect the budget. Both efficiency and absorptive
capacity play key roles in determining the final impact of public investment
on growth, along with the rate of return to public capital.\footnote{%
As in Berg et al. (2010b), we also introduce learning-by-doing externalities
in the production of both sectors, defined in terms of sectoral outputs.
These externalities capture the Dutch-disease (Dutch-vigor) notion that real
exchange appreciation (depreciation) may harm (help) productivity growth in
the traded sector, which is a major concern in LICs that face aid surges,
including substantial increases in concessional borrowing. Nevertheless, in
this paper we do not elaborate on the implications of these externalities
for debt sustainability.}

We allow for different government financing options and state explicitly the
fiscal policy reactions that may ensure debt sustainability. In our
analysis, we take available aid and concessional borrowing flows as
exogenously given. Absent additional financing sources, the government
adjusts taxes and transfers to finance the public investment scaling up. The
model then considers external commercial borrowing and domestic borrowing to
help finance the public investment surge, with taxes and transfers
responding to stabilize debt levels over time. The model allows the
imposition of feasibility constraints on the pace or level of these fiscal
adjustments for taxes and transfers, potentially yielding explosive debt
trajectories.

Finally, the model contains a number of other features and shocks that are
common in LICs and shape the macroeconomic effects of public investment
surges. The model has traded and non-traded sectors and separate prices for
exports and imports. These features allow an analysis of the real exchange
rate and the need to achieve external and internal balance; they also permit
the analysis of shocks to the terms of trade (TOT). On the private sector
side, it incorporates hand-to-mouth consumers and limited access to
international capital markets to capture financial market imperfections. The
presence of these consumers helps break Ricardian equivalence. The limited
access to international capital markets is key to making nonconcessional
borrowing by the government important. With fully open capital accounts, it
would not matter whether the government borrowed domestically or abroad, for
example, as private agents could borrow abroad to lend to the government. In
addition to TOT shocks, the model incorporates shocks to the government
external debt risk premium (or world interest rates) and negative total
factor productivity (TFP) shocks, as a way to model natural disaster shocks.

Because the future is uncertain, asserting that a path of public debt is
unsustainable is still challenging. As Wyplosz (2007) argues
\textquotedblleft \textit{it is future balances that matter, not the past
and not just the current debt level. Huge debts can be paid back, and small
debts may not be sustainable, it all depends on what the primary balance
will look like in the future, including the very distant future.}%
\textquotedblright\ Our model provides a logically consistent framework that
helps unveil the trade-offs and potential risks associated with different
types of financing and fiscal policy reactions. Given a calibration and some
assumptions about structural conditions, financing options and fiscal policy
reactions of a particular LIC, the model can help IMF country teams or
country authorities build different scenarios to inform the DSA. This should
help articulate and dissect ambitious borrowing plans that aim to push
growth above historical averages along with analysis of \textquotedblleft an
alternative high-investment, low-growth payoff scenario\textquotedblright\
to counterbalance potential tendencies toward excessive optimism.\footnote{%
On calls to pursue alternative scenarios in DSAs, see IMF and World Bank
(2006) and Barkbu et al. (2008).}

We calibrate the model to the \textquotedblleft average\textquotedblright\
LIC and pursue different policy experiments, whose results speak directly to
many of the issues that have preoccupied the literature:

\begin{itemize}
\item Despite the low tax-take in LICs, increases in infrastructure
investment may be self-financing in the long run. The favorable long-run
effect on the budget reflects extra increases in output and revenue
associated with strong crowding in of private capital. For this to happen
the economy must feature strong \textit{structural conditions}, such as high
returns on public capital, high public investment efficiency and high
collection rates of user fees, among others.

\item Even very good (high rate-of-return) projects may not be fully
self-financing, however, because most of the direct benefits of the higher
public capital accrue to the private sector and average tax collection rates
are quite low.

\item Even if the investment program is self-financing in the long run,
transition problems can be formidable. Absent additional borrowing or aid,
the revenue gains from growth will not materialize soon enough to obviate
the need for difficult fiscal adjustments on the transition path, especially
when the scaling up is front-loaded. Tax rates may have to increase sharply,
crowding out private investment and consumption and further aggravating the
near-term fiscal challenge.

\item Nonconcessional external borrowing \textit{can }smooth away difficult
fiscal adjustments, reconciling scaling up of public investment with
constraints on feasible increases in tax rates (or cuts in spending). But
this strategy may be risky. Low rates of return, inefficient public
investment, sluggish fiscal adjustment, or low absorptive capacity can
easily lead to unsustainable public debt.

\item Borrowing in the domestic debt market is ineffective in smoothing the
path of fiscal adjustment and avoiding private sector crowding out. It does
not provide additional resources from abroad, so the public investment
scaling up still requires a decline in private consumption and investment in
the first few years. In addition, interest rates are likely to be higher
than with external borrowing, which further deteriorates the prospects for
private investment.

\item Commercial borrowing can make the economy more vulnerable to
macroeconomic instability in the presence of persistent unexpected shocks,
such as to the TOT, TFP, or public debt risk premium. These shocks are less
prone to ignite explosive paths of public debt when aid or concessional
lending responds positively to negative shocks.

\item Because there is uncertainty about the underlying parameters and
because of exogenous shocks, size matters. If the increase in public
investment is small relative to the size of the economy, then the risk of
debt distress does not depend much on these parameters or on the shocks. But
as the investment grows larger, they become more critical to the risks of
debt distress.
\end{itemize}

Our work distinguishes itself from the literature that studies the
macroeconomic effects of public investment by analyzing the trade-offs of
different types of public debt, while underscoring the role for debt
sustainability of i) structural and policy conditions and ii) exogenous
shocks. The seminal works by Barro (1990), Sala-i-Martin (1992), Futagami et
al. (1993) and Glomm and Ravikumar (1994) analyze the growth impact of
public investment in the context of endogenous growth models. More recently,
Chatterjee and Turnovsky (2007) and Agenor (2010), among others, have relied
on these endogenous growth setups to explore the importance of some
LIC-specific features for the public capital accumulation and growth nexus.
The former explore the importance of tied vs untied aid, while the latter
emphasizes the role of infrastructure network effects and the efficiency of
public investment. All these models, however, assume government \textit{%
balanced} budget rules, thereby abstracting from public debt accumulation.
On the other hand, Turnovsky (1999), Greiner et al. (2005) and Greiner
(2007), among others, incorporate government debt in their endogenous growth
framework. In contrast to our work, they do not allow for different
financing schemes and ignore the role played by the structural and policy
conditions for debt sustainability. Finally, the works by Adam and Bevan
(2006), Cerra et al. (2008), and Berg et al. (2010b), among others, look
into the macroeconomic effects of aid-financed public investment expansions.
But here again, external public debt accumulation is not allowed and,
therefore, the interaction of structural and policy conditions with debt
dynamics is missing.

Finally, our paper is also related to the literature about debt
sustainability, but our emphasis is on LICs. Celasun et al. (2007), Garcia
and Rigobon (2005), and Mendoza and Oviedo (2004), among others, focus on
emerging economies; while Bohn (1998) and Ghosh et al. (2011), among others,
concentrate on advanced economies.

The main body of the paper is organized into seven sections. In Sections II
and III we explicate the model and calibrate it to the \textquotedblleft
average\textquotedblright\ LIC. Following these, Section IV analyzes the
long-run impact of a permanent, large increase in public investment. Then
Sections V and VI investigate the medium-term trade-offs and potential risks
associated with the different financing schemes. Finally, Section VII
concludes.

\section{The Model}

\quad\ \thinspace Our framework is the standard two-sector model of a small
open economy embellished with multiple types of public sector debt and
multiple tax and spending variables. The country produces a traded good $%
q_{x}$ and a non-traded good $q_{n}$ from private capital $k$, labor $L$,
and government-supplied infrastructure $z$. Besides these domestically
produced goods, agents can import a traded good for consumption $c_{m}$ and
machines $\mathfrak{m}_{mm}$ to produce factories (private capital) and
infrastructure (public capital). All quantity variables except labor are
detrended by $(1+g)^{t}$, where $g$ is the exogenous long-run growth rate of
real GDP.\footnote{%
In the long run all variables, including real GDP, grow at the same
exogenous growth rate $g.$\ However, in the short to medium term,
significant public and private capital accumulation, resulting from scaling
up investment, implies that the growth rate of the economy can go above $g.$}
A composite good produced abroad is the numeraire, with the associated
consumer price index (CPI) denoted by $P_{t}^{\ast },$ which is assumed to
be equal to one for simplicity. Since the time horizon for the DSA is about
20 years, the model abstracts from money and all nominal rigidities.%
\footnote{%
We are currently working on a version that includes money and nominal price
rigidities along the lines of the model in Berg et al. (2010b).}

We lay out the model in stages, starting with the specification of
technology.

\subsection{Firms}

\subsubsection{Technology}

\quad\ \thinspace In each sector $j$, the representative firm use
Cobb-Douglas technologies to convert labor $L_{j,t}$, private capital $%
k_{j,t-1}$, and effectively productive infrastructure $z_{t-1}^{e}$, which
is a public good, into output:\footnote{%
We assume Cobb-Douglas technologies but, to some extent, we do not expect
significant changes in our results by considering CES technologies.} 
\begin{equation}
q_{x,t}=A_{x,t}\left( z_{t-1}^{e}\right) ^{\psi _{x}}\left( k_{x,t-1}\right)
^{\alpha _{x}}\left( L_{x,t}\right) ^{1-\alpha _{x}},  \label{qx}
\end{equation}%
and%
\begin{equation}
q_{n,t}=A_{n,t}\left( z_{t-1}^{e}\right) ^{\psi _{n}}\left( k_{n,t-1}\right)
^{\alpha _{n}}\left( L_{n,t}\right) ^{1-\alpha _{n}}.  \label{qn}
\end{equation}%
The firm productivities are expressed as 
\begin{equation*}
A_{x,t}=a_{x}\left( \frac{q_{x,t-1}^{I}}{\bar{q}_{x}^{I}}\right) ^{\sigma
_{x}}\left( k_{x,t-1}^{I}\right) ^{\xi _{x}}\text{ \ \ \ \ \ \ \ and \ \ \ \
\ \ \ \ }A_{n,t}=a_{n}\left( \frac{q_{n,t-1}^{I}}{\bar{q}_{n}^{I}}\right)
^{\sigma _{n}}\left( k_{n,t-1}^{I}\right) ^{\xi _{n}}
\end{equation*}%
and feature sector-specific externalities of two types, with variables with
the superindex $I$ denoting sectoral quantities: a \textquotedblleft
static\textquotedblright\ externality associated with private capital
accumulation---$\left( k_{j,t-1}^{I}\right) ^{\xi _{j}}$ for $j=x,n$ ---as
in Arrow (1962); and a \textquotedblleft
learning-by-doing\textquotedblright\ externality that depends on the
deviations of the lagged sector output from the (initial) steady state---$%
\left( \frac{q_{j,t-1}^{I}}{\bar{q}_{j}^{I}}\right) ^{\sigma j}$ for $j=x,n$%
. When the latter externality is greater in the traded sector than that in
the non-traded sector, then it can capture the notion of Dutch-disease as in
Berg et al. (2010b), where a decline in the traded sector imposes an
economic cost through a sectoral loss in total-factor productivity (TFP).

Factories and infrastructure are built by combining one imported machine
with $a_{j}$ ($j=k,z$) units of a non-traded input (e.g., construction). The
supply prices of private capital and infrastructure are thus 
\begin{equation}
P_{k,t}=P_{mm,t}+a_{k}P_{n,t},  \label{Pk}
\end{equation}%
and%
\begin{equation}
P_{z,t}=P_{mm,t}+a_{z}P_{n,t},  \label{Pz}
\end{equation}%
where $P_{n}$ is the (relative) price of the non-traded good and $P_{mm}$ is
the (relative) price of imported machinery.

\subsubsection{Factor Demands}

\quad\ \thinspace Competitive profit-maximizing firms equate the marginal
value product of each input to its factor price. This yields the input
demand equations 
\begin{equation}
P_{n,t}(1-\alpha _{n})\frac{q_{n,t}}{L_{n,t}}=w_{t},  \label{Ln_DD}
\end{equation}%
\begin{equation}
P_{x,t}(1-\alpha _{x})\frac{q_{x,t}}{L_{x,t}}=w_{t},  \label{Lx_DD}
\end{equation}%
\begin{equation}
P_{n,t}\alpha _{n}\frac{q_{n,t}}{k_{n,t-1}}=r_{n,t},  \label{Kn_DD}
\end{equation}%
and%
\begin{equation}
P_{x,t}\alpha _{x}\frac{q_{x,t}}{k_{x,t-1}}=r_{x,t},  \label{Kx_DD}
\end{equation}%
where $w$ is the wage and $r_{j}$ is the rental earned by capital in sector $%
j$. Labor is intersectorally mobile, so the same wage appears in (\ref{Ln_DD}%
) and (\ref{Lx_DD}). Capital is sector-specific, but $r_{x}$ differs from $%
r_{n}$ only on the transition path. After adjustment is complete and $k_{x}$
and $k_{n}$ have settled at their equilibrium levels, the rentals are equal.

\subsection{Consumers}

\quad\ \thinspace There are two types of private agents, savers and
non-savers, with the former and the latter distinguished by the superscripts 
$\mathfrak{s}$ and $\mathfrak{h,}$ respectively. Labor supply of savers is
fixed at $L^{\mathfrak{s}}$ while that of non-savers is $L^{\mathfrak{h}%
}=aL^{\mathfrak{s}}$ with $a>0$. The two types of agents consume the
domestic traded good $c_{x,t}^{i}$, the foreign traded good $c_{m,t}^{i}$,
and the domestic non-traded good $c_{n,t}^{i}$ for $i=\mathfrak{s},\mathfrak{%
h}.$ These goods are combined into a CES basket%
\begin{equation}
c_{t}^{i}=\left[ \rho _{x}^{\frac{1}{\epsilon }}\left( c_{x,t}^{i}\right) ^{%
\frac{\epsilon -1}{\epsilon }}+\rho _{m}^{\frac{1}{\epsilon }}\left(
c_{m,t}^{i}\right) ^{\frac{\epsilon -1}{\epsilon }}+(\rho _{n})^{\frac{1}{%
\epsilon }}\left( c_{n,t}^{i}\right) ^{\frac{\epsilon -1}{\epsilon }}\right]
^{^{\frac{\epsilon }{\epsilon -1}}}\text{ \ \ \ for }i=\mathfrak{s},%
\mathfrak{h}  \label{c_basket}
\end{equation}%
where $\rho _{x}$, $\rho _{m}$, and $\rho _{n}$ are CES distribution
parameters and $\epsilon $ is the intratemporal elasticity of substitution.
In addition $\rho _{n}=1-\rho _{x}-\rho _{m}$.

The (relative) CPI associated with the basket (\ref{c_basket}) is $P_{t}=%
\left[ \rho _{x}P_{x,t}^{1-\epsilon }+\rho _{m}P_{m,t}^{1-\epsilon }+\rho
_{n}P_{n,t}^{1-\epsilon }\right] ^{^{\frac{1}{1-\epsilon }}},$ while the
demand functions for each good can be expressed as 
\begin{equation*}
c_{j,t}^{i}=\rho _{j}\left( \frac{P_{j,t}}{P_{t}}\right) ^{-\epsilon
}c_{t}^{i}\text{ \ \ \ \ \ \ for \ \ \ }j=x,m,n\ \ \ \text{and \ \ }\ \text{ 
}i=\mathfrak{s},\mathfrak{h}.
\end{equation*}

Savers can invest $i_{x}$ and $i_{n}$ amounts in private capital that
depreciates at the rate $\delta $, pay user fees charged for infrastructure
services according to $\mu z^{e}$, can buy domestic bonds $b$---which cannot
be bought in any market by foreigners---and pay a real interest rate $r,$
and can contract foreign debt $b^{\ast }$ that charges an exogenous real
interest rate $r^{\ast }$. They solve the intertemporal problem

\begin{equation*}
Max\sum_{t=0}^{\infty }\beta ^{t}\frac{\left( c_{t}^{\mathfrak{s}}\right)
^{1-1/\tau }}{1-1/\tau },
\end{equation*}%
subject to

\begin{eqnarray}
P_{t}b_{t}^{\mathfrak{s}}-b_{t}^{\mathfrak{s}\ast } &=&r_{x,t}k_{x,t-1}^{%
\mathfrak{s}}+r_{n,t-1}k_{n,t-1}^{\mathfrak{s}}+w_{t}L_{t}^{\mathfrak{s}}+%
\frac{\mathcal{R}_{t}}{1+a}+\frac{\mathcal{T}_{t}}{1+a}-\frac{%
1+r_{t-1}^{\ast }}{1+g}b_{t-1}^{\mathfrak{s}\ast }+\frac{1+r_{t-1}}{1+g}%
P_{t}b_{t-1}^{\mathfrak{s}}  \notag \\[0.08in]
&&-P_{k,t}\left( i_{x,t}^{\mathfrak{s}}+i_{n,t}^{\mathfrak{s}}+AC_{x,t}^{%
\mathfrak{s}}+AC_{n,t}^{\mathfrak{s}}\right) -P_{t}c_{t}^{\mathfrak{s}%
}(1+h_{t})-\mu z_{t-1}^{e}-\mathcal{P}_{t}^{\mathfrak{s}}-\Phi _{t}^{%
\mathfrak{s}},  \label{BCs}
\end{eqnarray}

\begin{equation}
(1+g)k_{x,t}^{\mathfrak{s}}=i_{x,t}^{\mathfrak{s}}+(1-\delta )k_{x,t-1}^{%
\mathfrak{s}},  \label{kx_accum}
\end{equation}%
and

\begin{equation}
(1+g)k_{n,t}^{\mathfrak{s}}=i_{n,t}^{\mathfrak{s}}+(1-\delta )k_{n,t-1}^{%
\mathfrak{s}},  \label{kn_accum}
\end{equation}%
where $\beta =1/[(1+\varrho )(1+g)^{(1-\tau )/\tau }]$ is the discount
factor; $\varrho $ is the pure time preference rate; $\tau $ is the
intertemporal elasticity of substitution; $\delta $ is the depreciation
rate; $\mathcal{R}$ are remittances; $\mathcal{T}$ are (net) transfers; $h$
denotes the consumption value added tax (VAT); and $\Phi ^{\mathfrak{s}}$
are profits from domestic firms. Remittances and transfers are proportional
to the agent's share in aggregate employment. Observe that in the budget
constraint (\ref{BCs}), the trend growth rate appears in several places in (%
\ref{BCs})-(\ref{kn_accum}), reflecting the fact that some variables are
dated at $t$ and others at $t$-$1$ and that $P_{t}$ multiplies $b_{t}^{%
\mathfrak{s}}$ and $b_{t-1}^{\mathfrak{s}}$because domestic bonds are
indexed to the price level.\footnote{%
The convention for detrending the capital stocks differs from that for other
variables. Because $K_{j,t-1}^{\mathfrak{s}}$---the capital stock before
detrending---is the capital stock in use at time $t$, we define $k_{j,t-1}^{%
\mathfrak{s}}\equiv K_{j,t-1}^{\mathfrak{s}}/(1+g)^{t}$. Under this
convention, $\bar{\imath}_{j}^{\mathfrak{s}}=(\delta +g)\bar{k}_{j}^{%
\mathfrak{s}}$ in the long run---as required for the capital stock to grow
at the trend growth rate $g$.} Also note that there are adjustment costs
incurred in changing the capital stock---$AC_{j,t}^{\mathfrak{s}}\equiv 
\frac{v}{2}\left( \frac{i_{j,t}^{\mathfrak{s}}}{k_{j,t-1}^{\mathfrak{s}}}%
-\delta -g\right) ^{2}k_{j,t-1}^{\mathfrak{s}},$ for $j=x,n$ and with $v>0$%
---and portfolio adjustment costs associated with foreign liabilities---$%
\mathcal{P}_{t}^{\mathfrak{s}}\equiv \frac{\eta }{2}(b_{t}^{\mathfrak{s}\ast
}-\bar{b}^{\mathfrak{s}\ast })^{2},$ where $\bar{b}^{\ast }$ is the
(initial) steady-state value of the private foreign liabilities.\footnote{%
For simplicity, we assume that adjustment costs are zero when the capital
stock grows at the trend growth rate $g$. This ensures that adjustment costs
are zero across steady states as in models that ignore trend growth.}

The choice variables in the optimization problem are $c_{t}^{\mathfrak{s}}$, 
$b_{t}^{\mathfrak{s}}$, $b_{t}^{\mathfrak{s}\ast }$, $i_{j,t}^{\mathfrak{s}}$%
, and $k_{j,t}^{\mathfrak{s}}$ for $j=x,n$. Routine manipulations of the
first-order conditions deliver: 
\begin{equation}
c_{t}^{\mathfrak{s}}=c_{t+1}^{\mathfrak{s}}\left( \beta \frac{1+r_{t}}{1+g}%
\frac{1+h_{t}}{1+h_{t+1}}\right) ^{-\tau },  \label{Euler Equation}
\end{equation}

\begin{equation}
(1+r_{t})\frac{P_{t+1}}{P_{t}}=\frac{1+r_{t}^{\ast }}{\left[ 1-\eta (b_{t}^{%
\mathfrak{s}\ast }-\bar{b}^{\mathfrak{s}\ast })\right] },  \label{UIP}
\end{equation}%
\begin{equation}
\frac{r_{x,t+1}}{P_{k,t+1}}+1-\delta +v\Upsilon _{x,t+1}^{\mathfrak{s}%
}\left( \frac{i_{x,t+1}^{\mathfrak{s}}}{k_{x,t}^{\mathfrak{s}}}+1-\delta
\right) -\frac{v}{2}\left( \Upsilon _{x,t+1}^{\mathfrak{s}}\right)
^{2}=(1+r_{t})\frac{P_{t+1}}{P_{t}}\frac{P_{k,t}}{P_{k,t+1}}\left(
1+v\Upsilon _{x,t}^{\mathfrak{s}}\right) ,  \label{ix_FOC}
\end{equation}%
and%
\begin{equation}
\frac{r_{n,t+1}}{P_{k,t+1}}+1-\delta +v\Upsilon _{n,t+1}^{\mathfrak{s}%
}\left( \frac{i_{n,t+1}^{\mathfrak{s}}}{k_{n,t}^{\mathfrak{s}}}+1-\delta
\right) -\frac{v}{2}\left( \Upsilon _{n,t+1}^{\mathfrak{s}}\right)
^{2}=(1+r_{t})\frac{P_{t+1}}{P_{t}}\frac{P_{k,t}}{P_{k,t+1}}\left(
1+v\Upsilon _{n,t}^{\mathfrak{s}}\right) ,  \label{in_FOC}
\end{equation}%
where $\Upsilon _{j,t}^{\mathfrak{s}}=\left( \frac{i_{j,t}^{\mathfrak{s}}}{%
k_{j,t-1}^{\mathfrak{s}}}-\delta -g\right) $ for $j=x,n.$ Each of these
equations admits a straightforward intuitive interpretation. Equation (\ref%
{Euler Equation}) is a slightly irregular Euler equation in which the slope
of the consumption path depends on the real interest rate adjusted for trend
growth and on changes in the VAT. The other three equations are arbitrage
conditions. Equation (\ref{UIP}) equalizes the real interest rate on
domestic bonds to the real interest rate on foreign private debt, adjusted
by portfolio costs. Similarly, equations (\ref{ix_FOC}) and (\ref{in_FOC})
require the return on capital in each sector, net of marginal adjustment
costs, to equal the real interest rate.

In our modelling decisions, we balance realism and flexibility in
introducing portfolio adjustment costs to capture different degrees of
integration of the private sector into world capital markets. Equation (\ref%
{UIP}) implicitly defines a private demand for foreign debt, which can be
explicitly expressed as:%
\begin{equation*}
\eta (b_{t}^{\mathfrak{s}\ast }-\bar{b}^{\mathfrak{s}\ast })=1-\frac{%
1+r_{t}^{\ast }}{(1+r_{t})\frac{P_{t+1}}{P_{t}}}.
\end{equation*}%
In this equation, the value of $\eta $ controls the degree of capital
mobility. For some emerging market economies, a low $\eta $ may be
appropriate reflecting an open capital account. Elastic capital flows then
keep the domestic rate close to the foreign rate. In LICs, where $\eta $ is
comparatively big, the capital account is fairly closed, and the private
sector has limited capacity to borrow from abroad.\footnote{\baselineskip%
=12pt From a technical point of view, the portfolio costs also help to
ensure stationarity of $b_{t}^{\mathfrak{s}\ast }.$ See Schmitt-Groh\'{e}
and Uribe (2003) for alternative methods to ensure stationarity of net
foreign assets.}

In addition, we assume that on its foreign debt the private sector pays a
constant premium $\mathfrak{u}$ over the interest rate that the government
pays on external commercial debt $r_{dc}$, so 
\begin{equation*}
r_{t}^{\ast }=r_{dc,t}+\mathfrak{u.}
\end{equation*}%
This specification together with (\ref{UIP}) allows us to match the low
capital and investment ratios observed in LICs. The reason is that at the
steady state the domestic interest rate $r$ and the foreign rate $r^{\ast }$
are equal, meaning that $\bar{r}=\bar{r}_{dc}+\mathfrak{u.}$ Thus even if $%
\bar{r}_{dc}$ is low, by picking an appropriate $\mathfrak{u,}$ we can
obtain a somewhat high domestic real interest rate $\bar{r}$ and,
consequently, a high return on private capital. But because of decreasing
returns, this high return implies realistically that the ratios of the
capital stock and investment to GDP must be low at the initial steady-state
equilibrium.

Non-savers have the same utility function as that of savers and consume all
of their income from wages, remittances, and transfers each period. The
non-saver's budget constraint then reads 
\begin{equation}
(1+h_{t})P_{t}c_{t}^{\mathfrak{h}}=w_{t}L^{\mathfrak{h}}+\frac{a}{1+a}(%
\mathcal{R}_{t}+\mathcal{T}_{t})\text{.}  \label{hm_BC}
\end{equation}

In specifying this part of the model we have aimed for realism combined with
flexibility and generality. The realism of hand-to-mouth consumers
(non-savers) is indisputable given that in LICs a substantial portion of
households fall into this category.\footnote{%
For instance the 2009 Steadman Survey finds that 62 percent of Ugandans do
not have access to financial services.} From the modeling perspective their
inclusion allows us to break Ricardian equivalence.

We aggregate across both types of households, so $x_{t}=x_{t}^{\mathfrak{s}%
}+x_{t}^{\mathfrak{h}}$\ for $x_{t}=c_{t},$ $c_{l,t},$ $L_{t},$ $b_{t}^{\ast
},$ $b_{t},$ $i_{j,t},$ $k_{j,t},$ $AC_{j,t},$ $\mathcal{P}_{t},$ $\Phi
_{t}, $ and the subindices $l=x,n,m$ and $j=x,n.$ Bear in mind that $b_{t}^{%
\mathfrak{h}\ast }=b_{t}^{\mathfrak{h}}=i_{j,t}^{\mathfrak{h}}=k_{j,t}^{%
\mathfrak{h}}=AC_{j,t}^{\mathfrak{h}}=\mathcal{P}_{t}^{\mathfrak{h}}=\Phi
_{t}^{\mathfrak{h}}=0$ for $j=x,n$.

\subsection{The Government}

\subsubsection{Infrastructure, Public Investment and Efficiency}

\quad\ \thinspace Casual observation and indirect empirical evidence support
the conjecture in Hulten (1996) and Pritchett (2000): often the productivity
of infrastructure is high but the return on public investment very low for
the simple reason that a good deal of public investment \textit{spending}
does not increase the stock of productive capital. Measurement error
introduced by equating growth of productive capital with net investment can
explain why estimated TFP growth is zero or negative in many LDCs, as
suggested by Pritchett (2000); and why empirical studies generally find a
much stronger positive relationship between growth and physical indicators
of infrastructure than between growth and capital stock series calculated
via the perpetual inventory method, as reviewed by Straub (2008).\footnote{%
\baselineskip=12pt To understand the role of efficiency, it may be useful to
imagine that all the available public investment projects at a given point
in time are ranked from highest to lowest rate of return. In an efficient
investment process, an additional dollar is spent on the best available
project. It is possible, though, because of incompetence, corruption, or
imperfect information, that a government may choose worse projects. A lower
efficiency is a measure of the degree of deviation from the optimal process.
A complementary way to think about efficiency is simply that a fraction of
spending is simply wasted, e.g. misclassified as investment when it in fact
just covers transfers to civil servants.}

We thus allow for inefficiencies in public capital creation. Public
investment $i_{z}$ produces additional infrastructure $z$ according to:%
\begin{equation}
(1+g)z_{t}=(1-\delta )z_{t-1}+i_{z,t},  \label{z_accum}
\end{equation}%
but some of the newly built infrastructure may not be economically valuable
productive infrastructure, since \textit{effectively productive} capital $%
z_{t}^{e},$ which is actually used in technologies (\ref{qx}) and (\ref{qn}%
), evolves according to: 
\begin{equation}
z_{t}^{e}=\bar{s}\bar{z}+s(z_{t}-\bar{z}),\quad \text{with}\quad \bar{s}\in
\lbrack 0,1]\quad \text{and}\quad s\in \lbrack 0,1],  \label{z efficiency}
\end{equation}%
where $\bar{s}$ and $s$ are parameters of efficiency at and off steady
state, and $\bar{z}$ is public capital at the (initial) steady state.

Note that by combining equations (\ref{z_accum}) and (\ref{z efficiency}),
we obtain%
\begin{equation}
(1+g)z_{t}^{e}=(1-\delta )z_{t-1}^{e}+s(i_{z,t}-\bar{\imath}_{z})+\bar{s}%
\bar{\imath}_{z},  \label{I_efficiency}
\end{equation}%
where $\bar{\imath}_{z}=(\delta +g)\bar{z}$ is the public investment at the
(initial) steady state$.$ This is the same specification for public
investment inefficiencies as that of Berg et al.\thinspace (2010b), and it
is similar to the one in Agenor (2010). Since $s\in \lbrack 0,1],$ this
specification makes clear that \textit{one} dollar of additional public
investment $(i_{z,t}-\bar{\imath}_{z})$ does not translate into \textit{one}
dollar of effectively productive capital ($z_{t}^{e}$). For the simulations
below, we assume $\bar{s}=0.6$ and $s=0.6.$ These values are slightly higher
that the estimates of Arestoff and Hurlin (2006) for some emerging
economies. In principle, the public investment management quality index of
Dabla-Norris et al. (2011) could help calibrate this parameter.

\subsubsection{Fiscal Adjustment and the Public Sector Budget Constraint}

\quad\ \thinspace The government spends on transfers, debt service, and
infrastructure investment. It collects revenue from the consumption VAT and
from user fees for infrastructure services, which are expressed as a fixed
multiple/fraction $f$ of recurrent costs, that is $\mu =f\delta P_{zo}$.
When revenues fall short of expenditures, the resulting deficit is financed
through domestic borrowing $\Delta b_{t}=b_{t}-b_{t-1},$ external
concessional borrowing $\Delta d_{t}=d_{t}-d_{t-1},$ or external commercial
borrowing $\Delta d_{c,t}=d_{c,t}-d_{c,t-1}$ viz.: 
\begin{eqnarray}
P_{t}\Delta b_{t}+\Delta d_{c,t}+\Delta d_{t} &=&\frac{r_{t-1}-g}{1+g}%
P_{t}b_{t-1}+\frac{r_{d,t-1}-g}{1+g}d_{t-1}+\frac{r_{dc,t-1}-g}{1+g}d_{c,t-1}
\label{Gov_BC} \\[0.08in]
&&+P_{z,t}\mathbb{I}_{z,t}+\mathcal{T}_{t}-h_{t}P_{t}c_{t}-\mathcal{G}_{t}-%
\mathcal{N}_{t}-\mu z_{t-1}^{e},  \notag
\end{eqnarray}%
where $d$, $dc$, $\mathcal{G}$, and $\mathcal{N}$ denote concessional debt,
external commercial debt, grants, and natural resource revenues (if any);%
\footnote{%
We model natural resources revenues as a net foreign transfer, following
Dagher et al. (2012). As such \ the measure of GDP used below corresponds to
non-oil GDP. For a more comprehensive analysis of oil production, foreign
investment, and managing natural resources in LICs see Berg et al. (2012).}
and $r_{d}$ and $r_{dc}$ are the real interest rates (in dollars) on
concessional and commercial loans. The interest rate on concessional loans
is assumed to be constant $r_{d,t}=r_{d},$ while the interest rate on
external commercial debt incorporates a risk premium that depends on the
deviations of the external public debt to GDP ratio $\left( \frac{%
d_{t}+d_{c,t}}{y_{t}}\right) $ from its (initial) steady-state value $\left( 
\frac{\bar{d}+\bar{d}_{c}}{\bar{y}}\right) $. That is, 
\begin{equation}
r_{dc,t}=r^{f}+\upsilon _{g}e^{\eta _{g}\left( \frac{d_{t}+d_{c,t}}{y_{t}}-%
\frac{\bar{d}+\bar{d}_{c}}{\bar{y}}\right) },  \label{rdc}
\end{equation}%
where $r^{f}$ is a risk-free world interest rate and $%
y_{t}=P_{x,t}q_{x,t}+P_{n,t}q_{n,t}$ is GDP. Van der Ploeg and Venables
(2011) provide positive estimates for $\eta _{g}$. But note by setting $%
\upsilon _{g}>0$ and $\eta _{g}=0,$ our specification embeds the case of an
exogenous risk premium that does depend on public debt.

The term $P_{z,t}\mathbb{I}_{z,t}$ in the budget constraint (\ref{Gov_BC})
corresponds to public investment outlays including costs overruns associated
with absorptive capacity constraints. It is defined as%
\begin{equation*}
\mathbb{I}_{z,t}=\mathcal{H}_{t}(i_{z,t}-\bar{\imath}_{z})+\bar{\imath}_{z}.
\end{equation*}%
Because skilled administrators are in scarce supply in LICs, ambitious
public investment programs are often plagued by poor planning, weak
oversight, and myriad coordination problems, all of which contribute to
large cost overruns during the implementation phase.\footnote{%
Development agencies report that cost overruns of 35\% and more are common
for new projects in Africa. The most important factor by far is inadequate
competitive bidding for tendered contracts. See Foster and Briceno-Garmendia
(2010).} To capture this, we multiply new investment $(i_{z,t}-\bar{\imath}%
_{z})$ by $\mathcal{H}_{t}=\left( 1+\frac{i_{z,t}}{z_{t-1}}-\delta -g\right)
^{\phi }$, where $\phi \geq 0$ determines the severity of the the absorptive
capacity---or \textquotedblleft bottleneck\textquotedblright ---constraint
in the public sector. The constraint affects only implementation costs for
new projects: in a steady state, $\left( 1+\frac{\bar{\imath}_{z}}{\bar{z}}%
-\delta -g\right) ^{\phi }=1$ as $\bar{\imath}_{z}=(\delta +g)\bar{z}$.

Policy makers accept all concessional loans proffered by official creditors.
The borrowing and amortization schedule for these loans is fixed
exogenously. Given the paths for public investment and concessional
borrowing, the fiscal gap \textit{before }policy adjustment ($\mathfrak{Gap}%
_{t}$) can be defined as:%
\begin{equation}
\mathfrak{Gap}_{t}=\frac{1+r_{d}}{1+g}d_{t-1}-d_{t}+\frac{r_{dc,t-1}-g}{1+g}%
dc_{t-1}+\frac{r_{t-1}-g}{1+g}P_{t}b_{t-1}+P_{z,t}\mathbb{I}_{t}+\mathcal{T}%
_{o}-h_{o}P_{t}c_{t}-\mathcal{G}_{t}-\mathcal{N}_{t}-\mu z_{t-1}^{e}.
\label{gapdef}
\end{equation}%
That is, $\mathfrak{Gap}_{t}$ corresponds to expenditures (including
interest rate payments on debt) less revenues and concessional borrowing,
when transfers and taxes are kept at their \textit{initial} levels $\mathcal{%
T}_{o}$and $h_{o}$, respectively. Using this definition, we can rewrite the
budget constraint (\ref{Gov_BC}), in any given year, as: 
\begin{equation}
\mathfrak{Gap}_{t}=P_{t}\Delta b_{t}+\Delta d_{c,t}+(h_{t}-h_{o})P_{t}c_{t}-(%
\mathcal{T}_{t}-\mathcal{T}_{o}).  \label{exgap}
\end{equation}

In the short/medium run, this gap $\mathfrak{Gap}_{t}$ in (\ref{exgap}) can
be covered by domestic and/or external commercial borrowing $P_{t}\Delta
b_{t}+\Delta d_{c,t}$, tax adjustments $(h_{t}-h_{o})P_{t}c_{t},$ and/or
transfers adjustments $-(\mathcal{T}_{t}-\mathcal{T}_{o}).$ For the sake of
comparing different borrowing schemes, in the experiments below, we will
focus on cases where part of this gap can be filled with either external
commercial loans \textit{or} domestic loans, but not with both at the same
time.

Debt sustainability requires, however, that the VAT and transfers eventually
adjust to cover the entire gap (i.e., $P_{t}\Delta b_{t}+\Delta d_{c,t}=0$).
We let policy makers divide the burden of adjustment (net windfall when $%
\mathfrak{Gap}_{t}<0$) between transfers cuts and tax increases. The
adjustments, defined below according to some reaction functions, have as
part of their targets the following debt-stabilizing values for transfers
and the VAT 
\begin{equation}
{\small h}_{t}^{\text{target}}{\small =h}_{o}+(1-\lambda )\frac{\mathfrak{Gap%
}_{t}}{P_{t}c_{t}}  \label{h_target}
\end{equation}%
and%
\begin{equation}
\mathcal{T}_{t}^{\text{target}}=\mathcal{T}_{o}-\lambda \mathfrak{Gap}_{t},
\label{T_target}
\end{equation}%
where $\lambda $ is a policy parameter that splits the fiscal adjustment
between taxes and transfers and therefore satisfies $0\leq \lambda \leq 1.$
For the extreme case of $\lambda =0$ (respectively $\lambda =1$) all the
adjustment falls on taxes (respectively transfers).

Taxes and transfers are defined according to the following reaction
functions:%
\begin{equation}
h_{t}=Min\left\{ h_{t}^{r},h^{u}\right\}  \label{hmin}
\end{equation}%
and%
\begin{equation}
\mathcal{T}_{t}=Max\left\{ \mathcal{T}_{t}^{r},\mathcal{T}^{l}\right\} ,
\label{Tmax}
\end{equation}%
where $h^{u}$ is a ceiling on taxes, $\mathcal{T}^{l}$ is a floor for
transfers, and $h_{t}^{r}$ and $\mathcal{T}_{t}^{r}$ are determined by the
fiscal rules%
\begin{equation}
h_{t}^{r}=h_{t-1}+\lambda _{1}({\small h}_{t}^{\text{target}%
}-h_{t-1})+\lambda _{2}\frac{(x_{t-1}-x^{\text{target}})}{y_{t}},\quad \text{%
with}\quad \text{ }\lambda _{1},\lambda _{2}>0,  \label{h_reaction}
\end{equation}%
and%
\begin{equation}
\mathcal{T}_{t}^{r}=\mathcal{T}_{t-1}+\lambda _{3}(\mathcal{T}_{t}^{\text{%
target}}-\mathcal{T}_{t-1})-\lambda _{4}(x_{t-1}-x^{\text{target}}),\quad 
\text{with}\quad \text{ }\lambda _{3},\lambda _{4}>0,  \label{T_reaction}
\end{equation}%
with $y=P_{n}q_{n}+P_{x}q_{x}$ and $x=b$ or $d_{c},$ depending on whether
the rules respond to domestic debt or commercial debt. The target for debt $%
x^{\text{target}}$ is given exogenously. The ceiling $h^{u}$ on taxes and
the floor $\mathcal{T}^{l}$ on transfers can also have discrete jumps over
time, as we show below.\footnote{%
For instance, to introduce discrete jumps in the cap on taxes we can
respecify the rule as $h_{t}=Min\left\{ h_{t}^{r},h_{t}^{u}\right\} $, where 
$h_{t}^{u}=h_{o}^{u}+\Delta h_{t}^{u},$ $h_{o}^{u}$ is the initial cap and $%
\Delta h_{t}^{u}$ is a discrete jump (shock) at time $t,$ which can be
temporary or permanent.} This allows us to model staggered tax and transfers
structures.

Given the targets, the reaction functions defined in (\ref{hmin})-(\ref%
{T_reaction}) together with the budget constraint (\ref{exgap}) embody the
core policy dilemma. Fiscal adjustment is painful, especially when
administered suddenly in large doses. The government would prefer therefore
to phase-in tax increases and expenditure cuts slowly ($\lambda _{1}>0$ and $%
\lambda _{3}>0$). Since fiscal adjustment is gradual, the debt instrument
that varies endogenously to satisfy the government budget constraint may
rise above its target level in the time it takes $h_{t}$ and $\mathcal{T}%
_{t} $ to reach ${\small h}_{t}^{\text{target}}$ and $\mathcal{T}_{t}^{\text{%
target}}$. When this happens, the transition path includes a phase in which $%
\mathcal{T}_{t}<\mathcal{T}_{t}^{\text{target}}$ and/or $h_{t}>{\small h}%
_{t}^{\text{target}}$ to generate the fiscal surpluses needed to pay down
the debt. In addition, and despite the fact that the rules respond to debt
(i.e., $\lambda _{2}>0$ and $\lambda _{4}>0$), if the government moves too
slowly (i.e., $\lambda _{1}$ and $\lambda _{3}$ are too low), or if the
bounds $h^{u}$ and $\mathcal{T}^{r}$ constrain adjustment too much, interest
payments will rise faster than revenue net of transfers, causing the debt to
grow explosively. Large debt-financed increases in public investment are
undeniably risky---the economy converges to a stationary equilibrium only if
policy makers win the race against time.

\subsection{Market-Clearing Conditions and External Debt Accumulation}

\quad\ \thinspace Flexible wages and prices ensure that demand continuously
equals supply in the labor market 
\begin{equation}
L_{x}+L_{n}=L,  \label{L_mkt}
\end{equation}%
where the labor supply $L=L^{\mathfrak{s}}+L^{\mathfrak{h}}$ is fixed.

In the non-tradables market, after aggregating across types of consumers, we
obtain 
\begin{equation}
q_{n,t}=\rho _{n}\left( \frac{P_{n,t}}{P_{t}}\right) ^{-\epsilon
}c_{t}+a_{k}\left( i_{x,t}+i_{n,t}+AC_{x,t}+AC_{n,t}\right) +a_{z}\mathbb{I}%
_{z,t}.  \label{NT_mkt}
\end{equation}%
The first term of the right-hand side of (\ref{NT_mkt}) is the demand for
non-traded consumer goods, while the second and third terms link public and
private investment to orders for new capital goods.

Finally, aggregating across consumers and adding the public and private
sector budget constraints produce the accounting identity that growth in the
country's net foreign debt equals the difference between national spending
and national income: 
\begin{eqnarray}
d_{t}-d_{t-1}+d_{c,t}-d_{c,t-1}+b_{t}^{\ast }-b_{t-1}^{\ast } &=&\frac{%
r_{d}-g}{1+g}d_{t-1}+\frac{r_{dc,t-1}-g}{1+g}d_{c,t-1}+\frac{r_{t-1}^{\ast
}-g}{1+g}b_{t-1}^{\ast }  \label{CA_Eq} \\
&&+\mathcal{P}_{t}+P_{z,t}\mathbb{I}_{z,t}+P_{k,t}\left(
i_{x,t}+i_{n,t}+AC_{x,t}+AC_{n,t}\right)  \notag \\
&&+P_{t}c_{t}-P_{n,t}q_{n,t}-P_{x,t}q_{x,t}-\mathcal{R}_{t}-\mathcal{G}_{t}-%
\mathcal{N}_{t}.  \notag
\end{eqnarray}%
Equation (\ref{CA_Eq}) includes extra terms that reflect the impact of trend
growth on real interest costs and the contributions of remittances, grants,
and natural resource related foreign transfers to gross income. The textbook
identity emerges when $g=\mathcal{R}_{t}=\mathcal{G}_{t}=\mathcal{N}_{t}=0$.

\section{Calibration of the Model}

\quad\ \thinspace Calibration of the model requires data on cost shares,
elasticities of substitution, consumption shares, depreciation rates, sector
shares in GDP, tax rates, debt stocks, and the return on infrastructure at
the benchmark equilibrium. Once values are set for these parameters, all
other variables that enter the model can be tied down by budget constraints,
the first-order conditions associated with the solution to the private
agents' optimization problems, and various adding-up constraints.

The values in Table 1 are based on a mixture of data and guesstimates. We
assume that $P_{n,o}=P_{x,o}=P_{m,o}=P_{mm,o}=w_{o}=P_{o}=1$ and discuss
below the rationale for the value assigned to each parameter and the
problems that arose in calibrating certain parts of the model:

\begin{itemize}
\item \textit{The distribution parameters }$(\rho _{n},\rho _{x},\rho _{m})$%
. In the base case, we pick $\rho _{n}$ and $\rho _{m}$ so that, the share
of non-tradables in GDP is about $50\%$ and the share of imports in GDP is
around $45\%$, which correspond to average shares of non-tradables and
imports for LICs during 1998-2008 using WEO data. This gives us $\rho
_{n}=0.43$ and $\rho _{m}=0.37.$ Then $\rho _{x}$ is calculated as $\rho
_{x}=1-\rho _{n}-\rho _{m}.$

\item \textit{Intertemporal elasticity of substitution }$(\tau )$. According
to Agenor and Montiel (1999), most estimates of $\tau $ for LDCs lie between 
$0.10$ and $0.50$. The value in the base case, $0.34$, equals the average
estimate for LICs in Ogaki et al.\thinspace (1996).

\item \textit{Elasticity of substitution in consumption }$(\epsilon )$. We
fix $\epsilon $ at $0.50$ as estimates of compensated elasticities of demand
tend to be small at high levels of aggregation, especially when food claims
a large share of total consumption.\footnote{%
See Lluch et al.\thinspace (1977, chapter 3), Deaton and Muellbauer (1980,
p.71), Blundell (1988, p.35), and Blundell et al. (1993, Table 3b).}

\item \textit{Capital's share in value added }$(\alpha _{n},\alpha _{x})$.
Data on factor shares may be found in social accounting matrices assembled
by the Global Trade Analysis Project (GTAP) and the International Food
Policy Research Institute (IFPRI). The GTAP5 database for SSA suggests a
capital share of $55-60\%$ in the non-tradables sector and $35-40\%$ in the
tradables sector.\footnote{%
The nontradables sector comprises trade and transport, private services,
dwellings, and construction. The tradables sector consists of agriculture
and manufacturing.} The data in Thurlow et al.\thinspace (2004) and Perrault
et al.\thinspace (2010) suggest similar numbers.\footnote{%
The average factor shares cited here conceal tremendous variation. For
example, the value added share of capital in the services sector is 59\% in
Zambia but only 27\% in Malawi. See Thurlow et al. (2004) and Thurlow et al.
(2008).} Accordingly, we set $\alpha _{n}=0.55$ and $\alpha _{x}=0.40$.

\item \textit{Learning externalities }$(\xi _{x},\xi _{n},\sigma _{x},\sigma
_{n})$. The base case does not incorporate learning externalities. In
alternative runs that allow for learning effects, $\xi _{x}$ and $\xi _{n}$
are set to $0.08$ so that the social return to capital in the traded sector
is about $30\%$ higher than the private return.\footnote{%
The net social return to capital, evaluated at a steady state is $(r+\delta
)(1+\xi _{j}/\alpha _{j})-\delta $. To set $\xi _{j}$ so that the social
return is 30\% above the private return, solve $(r+\delta )(1+\xi
_{j}/\alpha _{j})-\delta =(1.30)r$ for $\xi _{j}$.}

\item \textit{Cost share of non-traded inputs in the production of capital
goods} ($\alpha _{k},\alpha _{z}$). Data on the ratio of imported machinery
and equipment to aggregate investment indicate that $\alpha _{k}$ and $%
\alpha _{z}$ are around $0.5$ in SSA. One-half is also the guesstimate used
by the IMF (2007a) in its analysis of scaling up public investment in
Nigeria.

\item \textit{Elasticities of sectoral output with respect to the stock of
infrastructure} ($\psi _{x},\psi _{n}$). The ratio $\psi _{x}/\psi _{n}$ is
set independently. This ratio and other values assigned elsewhere in
calibrating the model---most notably, the return on infrastructure---pin
down $\psi _{n}$ and $\psi _{x}$.\footnote{$\psi _{n}$ and $\psi _{x}$ are
linked to other parameters and variables through $R_{z}=(\psi
_{n}VA_{n}+\psi _{x}VA_{x})(\delta +g)/iz_{y}$, where $R_{z}=R+\delta $ is
the gross return on infrastructure, $VA_{j}$ is the share of sector j
production in GDP, and $iz_{y}$ is the ratio of infrastructure investment to
GDP.} We assume $\psi _{x}/\psi _{n}=1$ in all runs and obtain that $\psi
_{n}=0.17.$

\item \textit{Depreciation rate }$(\delta )$. There is little hard data on
depreciation rates in LICs. Our choice of $5\%$ is in line with estimates
for developed countries.
\end{itemize}

\begin{center}
\begin{tabular}{ccc}
\multicolumn{3}{c}{\textbf{Table 1}} \\ 
\multicolumn{3}{c}{\textbf{Base Case Calibration}} \\ 
&  &  \\ \hline\hline
\textbf{Parameter} & \textbf{Value} & \textbf{Definition} \\ \hline\hline
${\small \rho }_{n}$ & {\small 0.43} & \multicolumn{1}{l}{\small %
Distribution parameter for non-traded goods} \\ 
${\small \rho }_{m}$ & {\small 0.37} & \multicolumn{1}{l}{\small %
Distribution parameter for traded goods} \\ 
${\small \tau }$ & {\small 0.34} & \multicolumn{1}{l}{\small Intertemporal
elasticity of substitution} \\ 
${\small \epsilon }$ & {\small 0.50} & \multicolumn{1}{l}{\small %
Intratemporal elasticity of substitution across goods} \\ 
${\small \alpha }_{x}$ & {\small 0.40} & \multicolumn{1}{l}{\small Capital's
share in value added in the traded sector} \\ 
${\small \alpha }_{n}$ & {\small 0.55} & \multicolumn{1}{l}{\small Capital's
share in value added in the non-traded sector} \\ 
${\small \xi }_{x}{\small ,\xi }_{n}$ & {\small 0.00} & \multicolumn{1}{l}%
{\small Capital learning externalities} \\ 
${\small \sigma }_{x}{\small ,\sigma }_{n}$ & {\small 0.00} & 
\multicolumn{1}{l}{\small Sectoral output learning externalities} \\ 
${\small \alpha }_{k}{\small ,\alpha }_{z}$ & {\small 0.50} & 
\multicolumn{1}{l}{\small Cost share of non-traded inputs in the production
of capital} \\ 
${\small \psi }_{x}{\small ,\psi }_{n}$ & {\small 0.17} & \multicolumn{1}{l}%
{\small Elasticities of sectoral output with respect to infrastructure} \\ 
${\small \delta }$ & {\small 0.05} & \multicolumn{1}{l}{\small Depreciation
rate} \\ 
${\small v}$ & {\small 6.41} & \multicolumn{1}{l}{\small Capital adjustment
cost parameter} \\ 
$\mu $ & {\small 0.05} & \multicolumn{1}{l}{\small User fees parameter for
infrastructure services} \\ 
${\small g}$ & {\small 0.015} & \multicolumn{1}{l}{\small Trend growth rate}
\\ 
${\small r}_{o}$ & {\small 0.10} & \multicolumn{1}{l}{\small Initial real
interest rate on domestic debt} \\ 
${\small r}_{o}^{\ast }$ & {\small 0.10} & \multicolumn{1}{l}{\small Initial
real interest rate on private external debt} \\ 
${\small r}^{f}$ & {\small 0.04} & \multicolumn{1}{l}{\small Real risk-free
foreign interest rate} \\ 
${\small r}_{d}$ & {\small 0.00} & \multicolumn{1}{l}{\small Real interest
rate on concessional loans} \\ 
${\small r}_{dc,o}$ & {\small 0.06} & \multicolumn{1}{l}{\small Initial real
interest rate on public commercial loans} \\ 
${\small \eta }$ & {\small 1.00} & \multicolumn{1}{l}{\small The portfolio
adjustment costs parameter} \\ 
${\small \eta }_{g}$ & {\small 0.00} & \multicolumn{1}{l}{\small Public debt
risk premium parameter} \\ 
$\mathfrak{u}$ & {\small 0.04} & \multicolumn{1}{l}{\small Private debt risk
premium} \\ 
${\small \upsilon }_{g}$ & {\small 0.02} & \multicolumn{1}{l}{\small Public
debt risk premium} \\ 
${\small R}_{{\small o}}$ & {\small 0.25} & \multicolumn{1}{l}{{\small %
Initial} {\small return on infrastructure}} \\ 
${\small b}_{o}$ & {\small 0.20} & \multicolumn{1}{l}{\small Initial public
domestic debt to GDP ratio} \\ 
${\small d}_{o}$ & {\small 0.50} & \multicolumn{1}{l}{\small Initial public
concessional debt to GDP ratio} \\ 
${\small d}_{c,o}$ & {\small 0.00} & \multicolumn{1}{l}{\small Initial
public external commercial debt to GDP ratio} \\ 
${\small b}_{o}^{\ast }$ & {\small 0.00} & \multicolumn{1}{l}{\small Initial
private external debt to GDP ratio} \\ 
$\mathcal{G}_{o}$ & {\small 0.05} & \multicolumn{1}{l}{\small Grants to GDP
ratio} \\ 
$\mathcal{R}_{o}$ & {\small 0.04} & \multicolumn{1}{l}{\small Remittances to
GDP ratio} \\ 
$\mathcal{N}_{o}$ & {\small 0.00} & \multicolumn{1}{l}{\small Natural
Resource Revenues to GDP ratio} \\ 
${\small i}_{z,o}$ & {\small 0.06} & \multicolumn{1}{l}{\small Initial ratio
of infrastructure investment to GDP} \\ 
$\bar{s},s$ & {\small 0.60} & \multicolumn{1}{l}{\small Efficiency of public
investment} \\ 
$\phi $ & {\small 0.00} & \multicolumn{1}{l}{\small Absorptive capacity
parameter} \\ 
$h_{o}$ & {\small 0.15} & \multicolumn{1}{l}{\small Initial consumption VAT}
\\ 
$\mathcal{T}_{o}$ & {\small 11.93} & \multicolumn{1}{l}{\small Initial
transfers to GDP ratio} \\ 
${\small \lambda }$ & {\small 0.00} & \multicolumn{1}{l}{\small Division of
fiscal adjustment parameter} \\ 
${\small \lambda }_{1}{\small ,\lambda }_{3}$ & {\small 0.25} & 
\multicolumn{1}{l}{\small Fiscal reaction parameters (policy instrument
terms)} \\ 
${\small \lambda }_{2}{\small ,\lambda }_{4}$ & {\small 0.02} & 
\multicolumn{1}{l}{\small Fiscal reaction parameters (debt terms)} \\ 
${\small a}$ & {\small 1.50} & \multicolumn{1}{l}{\small Labor ratio of
non-savers to savers} \\ \hline
\multicolumn{3}{l}{\small Note: See the calibration discussion in the main
text.}%
\end{tabular}
\end{center}

\begin{itemize}
\item \textit{The capital adjustment costs parameter (}$v$\textit{)}.
Evaluated at the initial equilibrium, this parameter is related to $\Omega $%
, the elasticity of investment with respect to Tobin's q, according to $%
\Omega =1/(\delta +g)v$.\footnote{%
In each sector $j$, the first-order condition for investment reads $%
[1+v(i_{j,t}/k_{j,t-1}-\delta -g)]\varpi _{t}P_{k,t}=\varsigma _{t}$, where $%
\varpi _{t}$ and $\varsigma _{t}$ are the multipliers associated with the
budget constraint and the law of motion for the capital stock. Since $\frac{%
\varsigma _{t}}{\varpi _{t}}$ is the shadow price of capital measured in
dollars, $\frac{\varsigma _{t}}{\varpi _{t}}P_{k}$ is effectively Tobin's $q$%
, the ratio of the demand price to the supply price of capital. Adopting
this notation, we have that at a stationary equilibrium $v(\delta +g)\frac{%
\bar{\imath}}{\bar{q}}=1$. Define $\Omega \equiv \frac{\bar{\imath}}{\bar{q}}
$ to be the q-elasticity of investment spending. Then $v=1/(\delta +g)\Omega 
$.} There are no reliable estimates of this elasticity for LICs. The
assigned value, $2$, is at the high end of estimates for developed
countries. This implies $v=6.41.$ The results do not change substantively
when $\Omega $ equals $1$ or $10$.

\item \textit{The user fees for infrastructure services (}$\mu $\textit{)}.
The user fee for infrastructure services is a fixed multiple/fraction $f$ of
recurrent costs $\mu =f\delta P_{zo}$. Fuel taxes, which are earmarked for
road maintenance and construction, electricity tariffs, and user charges for
water and sanitation are low but not trivial in LICs. According to Brice\~{n}%
o-Garmendia et al. (2008), on average, user fees recoup $50\%$ of recurrent
costs in SSA. Again, however, there is considerable variation---Zambia's
average electricity tariff was three cents per kWH in 2008. We decided
therefore to let $f$ vary from $0.20$ to unity, with $f=0.50$ in the base
case. Since in the baseline calibration $P_{zo}=\frac{1}{1-\alpha _{z}}=2$
and $\delta =0.05$, then $\mu =0.05$.

\item \textit{Trend growth rate }$(g)$. The trend growth rate of $1.5\%$
equals the 1990-2008 per capita growth rate for SSA reported in African
Development Indicators.

\item \textit{The real interest rate on domestic bonds }($r$), \textit{the
real return on private capital, and the real interest rate on external
private debt (}$r^{\ast }$\textit{)}. Across steady states, the real
interest rate on domestic debt and the real return on private capital equal $%
(1+\varrho )(1+g)^{\tau }-1,$ where $\varrho $ is the subjective discount
rate. We choose $\varrho $ jointly with $\tau $ and $g$ so that the domestic
real interest rate is $10\%$ at the initial equilibrium. This is consistent
with the data for SSA in Fedelino and Kudina (2003), with the estimated
return on private capital in Dalgaard and Hansen (2005), and with the
stylized fact that domestic debt in low- and middle-income countries is
usually more expensive than external commercial debt. There is tremendous
variation in real interest rates across countries and time periods, however.
Note that at the steady state equation (\ref{UIP}) implies that the interest
rate that the private sector pays on foreign debt must satisfy $r^{\ast }=r.$
So\ for the base case we also have $r^{\ast }=0.10.$

\item \textit{The risk-free foreign real interest rate }$(r^{f})$. We fix $%
r^{f}$ at $4\%$, the approximate average of the historical real returns on
stocks and 3-10 year T bills in the United States.

\item \textit{Real interest rates on concessional and non-concessional loans 
}$(r_{d},r_{dc,o})$. Ghana paid $8.7\%$ on the $\$750$ mn.\thinspace
Eurobond it floated in 2007. This is slightly above Gueye and Sy's (2010)
estimate of the average interest rate SSA pays ($8.55\%$) on debt raised in
external capital markets, excluding Seychelles and South Africa. The
IMF-WB's DSAs show an average interest rate of $2.3\%$ on concessional loans
taken out by LICs in 2009-2010. Assuming $2.5\%$ inflation in world prices
of traded goods, the corresponding (initial) real rates in dollars are about 
$6\%$ for commercial debt and $0\%$ for concessional debt. The latter is
assumed to be constant through the analysis.

\item \textit{The portfolio adjustment costs parameter (}$\eta $\textit{)
and the private and public debt risk premia parameters} ($\mathfrak{u,}%
\upsilon _{g},\eta _{g}$). The parameter $\eta $ controls the degree of
openness of the capital account. We set $\eta =1,$ in the base case, to
capture the fact that the private sector has limited access to international
capital markets. We assume that the public risk premium is constant---i.e.,
we set $\eta _{g}=0$ in equation (\ref{rdc})---and calibrate it as the
difference between the interest rate on public commercial debt and the
risk-free foreign interest rate. So, at the initial steady state
equilibrium, $\upsilon _{g}=r_{dc}-r^{f}=0.02.$ The constant private risk
premium is set as the difference between the domestic interest rate and the
interest rate on public commercial debt. Therefore $\mathfrak{u}=r^{\ast
}-r_{dc}=0.04.$

\item \textit{Return on infrastructure }($R_{o}$).\footnote{%
The production function parameters $\psi _{x}$ and $\psi _{n}$ that govern $%
R_{o}$ are deduced from the calibration of $R_{o}$, given the rest of the
calibration.} Estimates of the return on infrastructure are all over the
map, but the weight of the evidence in both micro and macro studies points
to a high average return. The median rate of return on World Bank projects
circa 2001 was $20\%$ in SSA and $15$-$29\%$ for various sub-categories of
infrastructure investment. Foster and Brice\~{n}o-Garmendia (2010) estimate
returns for electricity, water and sanitation, irrigation, and roads range
from $17\%$ to $24\%.$ Similarly, the macro-based estimates in Dalgaard and
Hansen (2005) cluster between $15\%$ and $30\%$ for a wide array of
different estimators. Hulten et al.\thinspace (2006), Escribano et
al.\thinspace (2008), Calder\'{o}n et al.\thinspace (2009), and Calder\'{o}n
and Serv\'{e}n (2010) supply additional evidence of high returns.\footnote{%
Some growth regressions suggest low or insignificant returns, but these are
dominated by studies that use cumulative public investment instead of
physical indicators to measure the stock of instructure.} All of this adds
up to a presumption that high returns are the norm. We consider a
high-return scenario as the base case by setting $R_{o}=0.25,$ at the
initial steady state.\footnote{%
Thirty percent may raise some eyebrows, but it is not as big as some of the
numbers thrown around in the literature and in policy work. See for instance
the scaling-up exercise in Box 4.1 in Barkbu et al.\thinspace (2008) and
Gupta, Powell, and Yang (2006).}

\item \textit{Domestic debt }($b_{o}$). Different datasets give different
numbers for the ratio of domestic debt to GDP in LICs. We settled on $20\%$
by averaging the figures reported in IMF(2009a), Panizza (2008), and Arnone
and Presbitero (2010).

\item \textit{Private foreign debt and public external debt (}$b_{o}^{\ast
},d_{o},d_{c,o}$). We set concessional external debt equal to $50\%$ of GDP
at the initial equilibrium, given that the ratio of total public debt to GDP
and the share of concessional loans in total debt were about $70\%$ and $%
69\%,$ respectively, for LICs during 2007-2008.\footnote{%
See IMF (2009a) and IMF staff calculations.} As little is known about the
likely value of private foreign debt (or assets) in LICs, we set $%
b_{o}^{\ast }=0$ for the base case. We also assume that initially the
economy has no access to external commercial loans implying that $d_{c,o}=0.$

\item \textit{Remittances, Grants, and Natural Resource Revenues} ($\mathcal{%
R}_{o}$, $\mathcal{G}_{o},$ $\mathcal{N}_{o}$). For the base case,
remittances and grants are assumed to be $5\%$ and $4\%$ of GDP at their
initial equilibrium, respectively. These are in line with averages for LICs
in the last decade. For the baseline calibration we assume that the economy
is not endowed with natural resources.

\item \textit{Initial ratio of infrastructure investment to GDP }$(\frac{%
i_{z,o}}{y_{o}})$. We set the initial infrastructure investment to be equal
to $6\%$ of GDP$.$ This initial figure includes the net investment
associated with trend growth and the outlays on operations and maintenance
(O+M)---which average about $3.4\%$ of GDP for LICs in SSA.\footnote{%
But true O+M costs are probably higher, as argued by Brice\~{n}o-Garmendia
et al. (2008). Because of underspending on O+M, 30\% of Africa's
infrastructure assets are in need of rehabilitation.} This figure is close
to the average for LICs in SSA, which in 2008 corresponded to $6.09\%$, as
suggested by Brice\~{n}o-Garmendia et al. (2008).

\item \textit{Efficiency of public investment (}$s,\bar{s}$)\textit{\ and
the absorptive capacity parameter }($\phi $).\footnote{%
Appendix A describes how calibration of $s$ and $\bar{s}$ interact with that
of $R_{o}$ and the production function parameters $\psi _{x}$ and $\psi _{n}$%
. The bottom line is that the operator needs to be thoughtful when
calibrating efficiency, particularly in considering what it might imply for
the marginal product of effective capital. A change in the calibration of
the efficiency $s$ has two different interpretations: (i) as a level change
that applies to the past and the future, as when comparing two countries at
a point in time. In this case, $\bar{s}$\ should in principle change as
well, but a change in $\bar{s}$ is unnecessary, because any effect of
changing $\bar{s}$ is undone by the change in $\psi _{x}$ and $\psi _{n}$
required to preserve the value of $R_{o}$; or (ii) as a change that applies
to future investment but not the past, as for example because of an
improvement in public financial management. It is up to the operator to
decide which applies and whether other adjustments to the calibration are
necessary in light of that interpretation.} The base case assumes that
investment is somewhat efficient ($\bar{s}=0.60$ and $s=0.60)$ and that
scaling up does not strain absorptive capacity ($\phi =0$ and $\bar{s}=s$).
Motivated by the findings in Hulten (1996), Pritchett (2000), and Foster and
Brice\~{n}o-Garmendia (2010), we also investigate scenarios in which the
scaling up is associated with extreme inefficiency ($s=0.2$) and a tight
absorptive capacity constraint ($\phi =5$).\footnote{%
Pritchett's estimates of $s$ range from $0.08$ to $0.49$ for SSA and from $%
0.09$ to $0.54$ for South Asia. In Africa, large cost overruns stemming from
planning/coordination/management problems and low capital budget execution
ratios (average = $66\%$) suggest that absorptive capacity may be a binding
constraint in many countries. See Foster and Brice\~{n}o-Garmendia (2010).}

\item \textit{Consumption VAT }$(h_{o})$. The consumption VAT in the model
proxies for the average indirect tax rate. Our rate of $15\%,$ at the
initial steady state.\footnote{%
For 2003-2006, consumption, indirect taxes, and trade taxes averaged 81\%,
9.6\%, and 4.6\% of GDP, respectively. If duties on consumer imports
accounted for half of trade taxes, then the average consumption tax was
14.7\%. (Data from IMF, 2007b). Ideally this rate will also reflect VAT
productivity adjustments.} This is comparable to the average VAT of LICs,
which using 2005-06 data by the International Bureau of Fiscal Documentation
is estimated to be close to $15.8\%.$

\item \textit{Net Transfers }$(\mathcal{T}_{o})$. At the initial steady
state, transfers ensure that the budget constraint of the government holds.
Given the other parameters we obtain $\mathcal{T}_{o}=11.93$ percent of GDP.
Given the definition of the other fiscal variables, this concept of
transfers includes other taxes different from VAT as well as non-capital
expenditures such as public wages.

\item \textit{Division of fiscal adjustment between expenditure cuts and tax
increases (}$\lambda $). Across steady states, we assume that only taxes
share the burden of fiscal adjustment ($\lambda =0$).

\item \textit{Policy reaction parameters (}$\lambda _{1},\lambda
_{2},\lambda _{3},\lambda _{4}$). There are no estimates of these parameters
for LICs. For the scenarios that allow commercial debt accumulation or
domestic debt accumulation we set $\lambda _{1}=\lambda _{3}=0.25$ and $%
\lambda _{2}=\lambda _{4}=0.02.$ We also study the implications of lowering $%
\lambda _{1}$.

\item \textit{Ratio of labor supply of non-savers to labor supply of savers }%
($a$). We set $a=1.5,$ so $60\%$ percent of the consumers are non-savers.
This is broadly in line with survey findings in LICs.
\end{itemize}

The public investment scaling-up scenario we study is exogenous and front
loaded. Concessional borrowing and an increase in grants, which are also
exogenous, help finance this scaling up.

The time lines for the infrastructure investment surge\textit{\ (valued at
the initial price level }$P_{z,o}=2$), concessional borrowing, and the
increase in grants, as percentage of \textit{initial} GDP ($y_{o}=100$), are
shown in Table 2. The time lines\ are hump-shaped. In year 1 public
investment is at $6\%$ of GDP and there is no increase. Then the \textit{%
scaling up} of this investment jumps to $5\%$ of \ initial GDP in year 2,
rises to $7\%$ in years 3 and 4, and tapers off gradually to its permanent
level of $3\%$ of initial GDP. Net concessional loans, on the other hand,
increase by $4\%$ of GDP in year 2, go slightly up to $5\%$ of initial GDP
in year 3 and then decline gradually until year 9.\footnote{%
The repayment period is leisurely stretched out over 27 years, after 8 years
of grace period. These correspond roughly to the average maturity and
grace-period years for new concessional loans to LICs in 2009-2010,\ based
on available IMF-WB's DSAs. We assume that the country contracts a
concessional loan of $20.25\%$ of initial GDP in year 2, with the previously
described disbursements. Then we apply an equal principal payment formula,
together with these grace and maturity periods and an interest rate of 0\%,
to obtain the repayment profile. The grant element of this loan is about $%
62\%$.} From year 10 to year 28, the country repays these loans ($-1.01\%$
of initial GDP).\ The increase in grants $\mathcal{G}$ corresponds to $0.4\%$
of initial GDP from year 2 to year 9, and declines to $0.2\%$ before dying
off by year 32. Concessional borrowing and the increase in grants will only
cover about $50\%$ of the investment surge during the first 8 years of the
scaling up. The rest will require some fiscal adjustment and, potentially,
other sources of financing such as external commercial or domestic borrowing.

\begin{center}
\begin{tabular}{ccccccccccccc}
&  &  &  &  &  &  &  &  &  &  &  &  \\ 
\multicolumn{13}{c}{\textbf{Table 2}} \\ 
\multicolumn{13}{c}{\textbf{Public Investment Scaling Up, Concessional
Borrowing, and Grants}} \\ 
&  &  &  &  &  &  &  &  &  &  &  &  \\ \hline\hline
$Year$ & ${\small 1}$ & ${\small 2}$ & ${\small 3}$ & ${\small 4}$ & $%
{\small 5}$ & ${\small 6}$ & ${\small 7}$ & ${\small 8}$ & ${\small 9}$ & $%
{\small 10...}$ & ${\small 29...}$ & ${\small 32...}$ \\ \hline\hline
&  &  &  &  &  &  &  &  &  &  &  &  \\ 
$\frac{P_{z,o}(i_{z,t}-i_{z,o})}{y_{o}}$ & ${\small 0.0}$ & ${\small 5.0}$ & 
${\small 7.0}$ & ${\small 7.0}$ & ${\small 6.6}$ & ${\small 5.8}$ & ${\small %
5.0}$ & ${\small 4.4}$ & ${\small 4.0}$ & ${\small 3.0...}$ & ${\small 3.0...%
}$ & ${\small 3.0}$ \\ 
&  &  &  &  &  &  &  &  &  &  &  &  \\ 
$\frac{Net\text{ }Loans\text{ }shock}{y_{o}}$ & ${\small 0.0}$ & ${\small 4.0%
}$ & ${\small 5.0}$ & ${\small 4.0}$ & ${\small 3.0}$ & ${\small 2.0}$ & $%
{\small 1.0}$ & ${\small 0.8}$ & ${\small 0.5}$ & {\small -}${\small 1.01...}
$ & ${\small 0.0...}$ & ${\small 0.0}$ \\ 
&  &  &  &  &  &  &  &  &  &  &  &  \\ 
$\frac{\mathcal{G}_{t}-\mathcal{G}_{o}}{y_{o}}$ & ${\small 0.0}$ & ${\small %
0.4}$ & ${\small 0.4}$ & ${\small 0.4}$ & ${\small 0.4}$ & ${\small 0.4}$ & $%
{\small 0.4}$ & ${\small 0.4}$ & ${\small 0.4}$ & ${\small 0.2...}$ & $%
{\small 0.2...}$ & ${\small 0.0}$ \\ 
&  &  &  &  &  &  &  &  &  &  &  &  \\ \hline
\multicolumn{13}{l}{{\small Note: figures are in percent of \ initial GDP (}$%
{\small y}_{o}{\small =100}$){\small .}} \\ 
&  &  &  &  &  &  &  &  &  &  &  & 
\end{tabular}
\end{center}

\section{ The Long-Run Outcome}

\quad\ \thinspace In this section, we use a mix of analytical and numerical
methods to demonstrate that in the long run (i) infrastructure and private
capital are strong complements and (ii) increases in infrastructure
investment can be self-financing, depending on structural conditions of the
economy.

\subsection{Insights From a Simplified Model}

\quad\ \thinspace Although the model of Section II has many moving parts, it
is not a black box. To highlight the key interactions that drive the
long-run (steady-state) outcome, consider a stripped-down model that ignores
trend growth, non-traded goods, foreign traded goods, private capital flows,
remittances, grants, natural resource revenues, public sector debt and
efficiency and absorptive capacity issues (so $s=\bar{s}=1$ and $\phi =0$).
For notational simplicity we ignore the upper bars on the variables, which
we used before to denote steady-state values. With these elements gone, the
steady-state equilibrium simplifies to 
\begin{equation}
q=az^{\psi }k^{\xi +\alpha }L^{1-\alpha },  \label{q_sss}
\end{equation}%
\begin{equation}
r+\delta =\alpha az^{\psi }k^{\xi +\alpha -1}L^{1-\alpha },  \label{asset_p}
\end{equation}%
\begin{equation}
i_{z}=\delta z,  \label{iz_ss}
\end{equation}%
\begin{equation}
hc+\mu z=\delta z+\mathcal{T},  \label{g_BC_ss}
\end{equation}%
and%
\begin{equation}
c=q-\delta (k+z);  \label{eq_mkt_ss}
\end{equation}%
five equations that can be solved for $q$, $k$, $z$, $c$ and either $%
\mathcal{T}$ or $h$ as a function of $i_{z}$. Equation (\ref{q_sss}) is the
production function and equations (\ref{asset_p})-(\ref{eq_mkt_ss}) are the
steady-state versions of equations (\ref{Kx_DD}), (\ref{z_accum}),(\ref%
{Gov_BC}), and (\ref{CA_Eq}) in Section II.

\frame{ftbpFU}{4.5766in}{3.9574in}{0pt}{\Qcb{\textbf{The Long-run Outcome in
the Simplified Model.}}}{}{Figure}{\special{language "Scientific Word";type
"GRAPHIC";maintain-aspect-ratio TRUE;display "USEDEF";valid_file "T";width
4.5766in;height 3.9574in;depth 0pt;original-width 12.6358in;original-height
10.9165in;cropleft "0";croptop "1";cropright "1";cropbottom "0";tempfilename
'M3TX1I0F.wmf';tempfile-properties "XPR";}}

Figure 1 depicts the steady-state equilibrium when transfer payments adjust
to satisfy the government budget constraint, but similar results obtain if
all the fiscal adjustment falls on taxes. The ray OZ in the first quadrant
relates $z$ to $i_{z}$. For the results that follow, it is important to note
the obvious: the slope of OZ is quite flat because comparatively small
increases in $i_{z}$ map into very large increases in $z$ over the long run $%
(dz=di_{z}/\delta )$.

Proceeding south, the KK schedule in the fourth quadrant shows how the
equilibrium private capital stock depends on the stock of infrastructure.
From (\ref{asset_p}), 
\begin{equation}
dk=\left( \frac{k}{z}\right) \left( \frac{\psi }{1-\xi -\alpha }\right) dz>0.
\label{dk_dz}
\end{equation}%
Substitute $k=\frac{\alpha q}{r+\delta }$ and $z=\frac{\psi q}{q_{z}}=\frac{%
\psi q}{R+\delta }$ into (\ref{dk_dz}), where $R=q_{z}-\delta $ is the net
return to infrastructure and $q_{z}=\frac{\psi q}{z}$ is the marginal
product of infrastructure. After canceling terms, we have 
\begin{equation}
\left. \frac{dk}{dz}\right\vert _{KK}=\left( \frac{R+\delta }{r+\delta }%
\right) \left( \frac{\alpha }{1-\xi -\alpha }\right) .  \label{dk_dz_v}
\end{equation}

Since we have assumed a Cobb-Douglas technology, growth in the stock of
infrastructure stimulates private investment by increasing the marginal
product of capital. Equation (\ref{dk_dz_v}) tells us, in addition, that the
long-run crowding-in effect may be quantitatively large. The ratio of the
gross return on infrastructure to the gross return on private capital, $%
\frac{R+\delta }{r+\delta }$, multiplies a term that lies somewhere between $%
0.49$ and $1.83$, if $\alpha \in \lbrack 0.33,0.55]$ and $\xi \in \lbrack
0,0.15]$. Thus, if empirical estimates are right and the mean return on
infrastructure is much higher than the mean return on private capital, then
the long-run crowding-in coefficient can approach or exceed two. Suppose,
for example, that $r=0.10$, $R=0.25$, $\delta =0.05$, and $\alpha =0.40$ as
there is only one domestic traded good. The crowding-in coefficient $\frac{dk%
}{dz}$ then ranges from about $1.33$ to $1.78$ when $\xi \in \lbrack 0,0.15]$%
. \textit{Productive} infrastructure and private capital are very strong
complements in the long run. This, however, depends on structural conditions
of the economy. As we will see below, public investment inefficiencies and
absorptive capacity constraints, among others, can lower the crowding in.

The schedules in the second and third quadrants connect the increases in the
stocks of infrastructure and private capital to consumption, tax revenue,
and the change in transfers needed to balance the fiscal budget. CC relates $%
c$ to $k$ for given $z$, while FB depicts the locus in the $c$-$i_{z}$ plane
for which revenue equals government expenditure. The slope of CC equals $%
(r+\delta )(1+\xi /\alpha )-\delta $, the net social marginal product of
capital. The FB schedule 
\begin{equation*}
\left. i_{z}\right\vert _{FB}=\frac{hc-\mathcal{T}}{1-\frac{\mu }{\delta }}
\end{equation*}%
has a horizontal intercept at $c=\frac{\mathcal{T}}{h}$ and a positive slope
that rises from $h$ at $\mu =0$ to infinity at $\mu =\delta $.

When $z$ increases, the CC schedule shifts horizontally to the left by $%
R(z_{1}-z_{o})$. Crowding-in of private capital increases consumption
another $[(\rho +\delta )(1+\xi /\alpha )-\delta ](k_{1}-k_{o})$. Expressed
relative to the policy instrument $i_{z}$, the combined effect is 
\begin{equation*}
dc=\left[ \left( \xi +\frac{r\alpha }{r+\alpha }\right) \frac{R+\delta }{%
1-\xi -\alpha }+R\right] \frac{di_{z}}{\delta }.
\end{equation*}

Comparing the increase in consumption taxes and user fees $(hdc+\mu dz=hdc+%
\frac{\mu }{\delta }di_{z})$ to the increase in investment determines
whether the FB schedule shifts left or right. In Figure 1 the revenue gain
pays for the increase in investment and leaves something left over to
finance higher transfer payments. This case occurs when 
\begin{equation}
R>\frac{\delta (1+h-\frac{\mu }{\delta })(1-\xi -\alpha )}{h\left( 1-\frac{%
\alpha \delta }{r+\delta }\right) }-\delta .  \label{R_ineq}
\end{equation}%
The crucial implication of (\ref{R_ineq}) is that $R$ does not have to be
unusually high for the increase in infrastructure investment to be
self-financing. Consider the base case values of Table 1 but bear in mind
that in this simplified model $\alpha =0.40$ (since there is only one
domestic traded good) and $\mu =0.025$ (since $\mu =\delta f$ and $f=0.50).$
In this case, the borderline value of $R$ is a modest $10\%$. Even with $\mu
=0$, however, (\ref{R_ineq}) holds for $R>21.5\%$. This is high but well
within the range of empirical estimates. Moreover, note that this borderline
value decreases with the externalities $\xi $.

\subsection{Numerical Solutions}

\quad\ \thinspace Table 3 summarizes the long-run effects of a permanent
increase in public investment equal to $3\%$ of initial GDP using the full
model. Consider first the base case. Transfers are kept constant, by
assumption, while taxes require a small adjustment from $15\%$ to $16.3\%$.
As predicted by the simplified model, the crowding-in coefficient,
calculated as $\frac{\Delta k}{\Delta z},$ exceeds unity. Consequently there
are some gains in the economy. Real output, real wages, and private capital
increase by more than $13\%$ while consumption rises by $9.3\%$. Sectoral
outputs also move up, and there is a permanent real depreciation of $1.9\%$.
Our base case calibration then delivers a positive scenario, to some extent.
But even with a rate of return on public capital of $25\%$ the public
investment does not pay for itself. The average tax rate is quite low, and
most of the benefits accrue to the private sector are not sufficient to
cover the recurrent costs.

The long-run impact of the public investment scaling up depends on the
structural conditions of the economy. To see this, consider scenario 1 of
Table 3. This scenario is very optimistic. The return on capital is $35\%,$
user fees recoup all recurrent costs---i.e., $f=1$ implying that $\mu
=f\delta P_{zo}=0.1$---the public investment surge is fully efficient $s=1$,
and there are some positive externalities $\xi =0.08$, which imply a social
return to capital in the traded sector that is $30\%$ higher than the
private return. With these conditions, the scaling up turns out to be more
than self-financing and produces striking benefits in the real economy.
Taxes decline to $12.3\%$ in the long run, while real GDP, wages, and
private capital expand by about $33-35\%$, as the crowding-in coefficient is
above $2.$ Furthermore, consumption jumps by $29.3\%$.

We issue two caveats.

First, it is also possible to recreate very troublesome scenarios under weak
structural conditions that may characterize some LICs. Scenario 2 in Table 3
is an example. This scenario features a low return on public capital $R=0.10$%
, a deficient collection rate of user fees of $20\%$ of recurrent
costs---i.e., $f=0.2$ implying that $\mu =f\delta P_{zo}=0.02$---and an
insignificant efficiency of additional public investment $s=0.2$. As a
result, taxes have to adjust permanently to $18.7\%$ and consumption falls
by $1.2\%$.\footnote{%
Note that the adjustment in taxes would be smaller if government transfers
were cut. But this is not an easy task when public sector employees are
pressing for wage increases that match the increase in the private sector
wages.} In addition, real GDP and private capital have disappointing
increases of about $2.5\%,$ despite the large increase in public investment.
All of this is consistent with a crowding-in coefficient that is below 1.

\begin{center}
\begin{tabular}{cccc}
&  &  &  \\ 
\multicolumn{4}{c}{\textbf{Table 3}} \\ 
\multicolumn{4}{c}{\textbf{Long-run Effects of Scaling Up Public Investment
by 3 Percent of Initial GDP}} \\ 
&  &  &  \\ \hline\hline
& \multicolumn{1}{|c}{\small Base Case} & \multicolumn{1}{|c}{\small %
Scenario 1} & \multicolumn{1}{|c}{\small Scenario 2} \\ \cline{2-4}
& \multicolumn{1}{|c}{${\small R=0.25,}$ $\mu {\small =0.06,}$} & 
\multicolumn{1}{|c}{${\small R=0.35,}$ $\mu {\small =0.1,}$} & 
\multicolumn{1}{|c}{${\small R=0.10,}$ $\mu {\small =0.02,}$} \\ 
& \multicolumn{1}{|c}{$s{\small =0.60,}$ and ${\small \xi =0}$} & 
\multicolumn{1}{|c}{$s{\small =1,}$ and ${\small \xi =0.08}$} & 
\multicolumn{1}{|c}{$s{\small =0.2,}$ and ${\small \xi =0}$} \\ \hline\hline
& \multicolumn{1}{|c}{} & \multicolumn{1}{|c}{} & \multicolumn{1}{|c}{} \\ 
\multicolumn{1}{l}{{\small Taxes }$h$} & \multicolumn{1}{|c}{\small 16.29} & 
\multicolumn{1}{|c}{\small 12.26} & \multicolumn{1}{|c}{\small 18.73} \\ 
\multicolumn{1}{l}{{\small CIC} ${\small \Delta (k}_{x}{\small +k}_{n}%
{\small )/\Delta z}^{e}$} & \multicolumn{1}{|c}{\small 1.51} & 
\multicolumn{1}{|c}{\small 2.31} & \multicolumn{1}{|c}{\small 0.85} \\ 
\multicolumn{1}{l}{{\small Public Effective Capital} ${\small z}^{e}$ \ \ \
\ \ \ \ \ } & \multicolumn{1}{|c}{\small 50.00} & \multicolumn{1}{|c}{\small %
83.33} & \multicolumn{1}{|c}{\small 16.67} \\ 
\multicolumn{1}{l}{{\small Private Capital} ${\small k}_{x}{\small +k}_{n}$}
& \multicolumn{1}{|c}{\small 13.23} & \multicolumn{1}{|c}{\small 33.70} & 
\multicolumn{1}{|c}{\small 2.49} \\ 
\multicolumn{1}{l}{{\small Real GDP }${\small y}^{r}$} & \multicolumn{1}{|c}%
{\small 13.45} & \multicolumn{1}{|c}{\small 34.31} & \multicolumn{1}{|c}%
{\small 2.48} \\ 
\multicolumn{1}{l}{{\small Real Wages }$w{\small /P}$} & \multicolumn{1}{|c}%
{\small 13.53} & \multicolumn{1}{|c}{\small 34.66} & \multicolumn{1}{|c}%
{\small 2.44} \\ 
\multicolumn{1}{l}{{\small Consumption} ${\small c}$} & \multicolumn{1}{|c}%
{\small 9.29} & \multicolumn{1}{|c}{\small 29.30} & \multicolumn{1}{|c}%
{\small -1.24} \\ 
\multicolumn{1}{l}{{\small Traded Output }${\small q}_{x}$} & 
\multicolumn{1}{|c}{\small 13.83} & \multicolumn{1}{|c}{\small 35.99} & 
\multicolumn{1}{|c}{\small 2.21} \\ 
\multicolumn{1}{l}{{\small Non-Traded Output }${\small q}_{n}$} & 
\multicolumn{1}{|c}{\small 13.04} & \multicolumn{1}{|c}{\small 32.57} & 
\multicolumn{1}{|c}{\small 2.75} \\ 
\multicolumn{1}{l}{{\small Real Exchange Rate }${\small p}_{n}$} & 
\multicolumn{1}{|c}{-{\small 1.90}} & \multicolumn{1}{|c}{\small -3.90} & 
\multicolumn{1}{|c}{-{\small 0.36}} \\ 
& \multicolumn{1}{|c}{} & \multicolumn{1}{|c}{} & \multicolumn{1}{|c}{} \\ 
\hline
\multicolumn{4}{l}{\small Note:\ The effects are measured as the percentage
change between the steady states, except \ } \\ 
\multicolumn{4}{l}{\small for taxes and the crowding-in coefficient (CIC).
Fiscal adjustment is assumed to fall exclusively} \\ 
\multicolumn{4}{l}{\small on taxes, so transfers are kept constant.} \\ 
&  &  & 
\end{tabular}
\end{center}

The second caveat pertains to difficulties that may arise on the transition
path. As we will show in the next section, even when the long run looks
good, medium-run tax increases appear to be unavoidable. Non-concessional
borrowing can smooth away difficult fiscal adjustments that threaten to
undermine support for the investment program. But if the capacity for fiscal
adjustment is too limited and the government borrows too much in the
short/medium run, there is a risk that future revenue gains will not arrive
fast enough to prevent explosive growth of the debt. Therefore, the
viability of the borrowing-for-development strategy depends not only on the
structural conditions of the economy and the long-run outcome, but also on
whether policy makers can solve the problem of how to get from here to there.

\section{The Medium-Term Fiscal and Macroeconomic Adjustments Under
Different Financing Schemes}

\subsection{Unconstrained Tax Adjustment}

\quad\ \thinspace There are a variety of ways to finance a large increase in
public investment. We start by analyzing the case in which, given the public
investment surge and the path for concessional borrowing and grants, taxes
and/or transfers adjust continuously and freely to satisfy the budget
constraint of the government. This provides the counterfactual for the
analysis of riskier schemes that supplement concessional borrowing with
domestic or external commercial loans.

For now, we assume the government is unable or unwilling to take out
non-concessional debt. Commercial and domestic debt thus stay at their
initial steady-state levels; so $d_{c,t}=\bar{d}_{c}=d_{c,o}$ and $b_{t}=%
\bar{b}=b_{o}$. Taxes and transfers adjust continuously and without bounds.
To capture this we set ${\small h}^{u}\rightarrow \infty ,$ $\mathcal{T}%
^{l}\rightarrow \infty $, $\lambda _{1}=\lambda _{3}=1$ and $\lambda
_{2}=\lambda _{4}=0$ in equations (\ref{hmin})-(\ref{T_reaction}) and
combine them with (\ref{h_target}) and (\ref{T_target}) to obtain 
\begin{equation*}
h_{t}={\small h}_{t}^{\text{target}}\text{ \ \ \ \ and \ \ \ \ }\mathcal{T}%
_{t}=\mathcal{T}_{t}^{\text{target}}.
\end{equation*}

To fix ideas and simplify the analysis, in the experiments below we assume
the burden of fiscal adjustment falls exclusively on taxes. In other words,
we set $\lambda =0.$ This implies that transfers are kept constant at their
initial level $\mathcal{T}_{t}=\mathcal{T}_{t}^{\text{target}}=\mathcal{T}%
_{o}$, while taxes follow the path $h_{t}={\small h}_{t}^{\text{target}},$
which, given $\lambda _{1}=1$ and $\lambda _{2}=0,$ ensures that taxes
always adjust to satisfy the government budget constraint.

\subsubsection{The Base Case}

\quad\ \thinspace The medium-run dynamics presented in Figure 2 are
discouraging.\footnote{%
The numerical simulations are free of approximation error. In all scenarios,
the simulations track the global nonlinear saddle path. The solutions were
generated by set of programs written in Matlab 7.7.0.471 (R2008b) and Dynare
4.1.1. See http://www.cepremap.cnrs.fr/dynare.} Since cumulative borrowing
and grants in years 2-9 equals about $50\%$ of cumulative net investment,
the fiscal adjustment has to be demanding. The VAT rises to $18.9\%$ at $t=5$
and to $18.8\%$ at $t=10$. The former increase is associated with the
demands from the scaling up, while the latter is related to the needs for
concessional debt repayment. Although conditions improve slightly
thereafter, by year 20, the VAT still remains above $18\%$.

\FRAME{ftbpFU}{7.2826in}{4.4114in}{0pt}{\Qcb{\textbf{Base Case:
Unconstrained Tax Adjustment.} The transition path in the base case when the
government takes out only concessional loans and the tax adjustment is
unconstrained. Variables are expressed as percentage deviations from the
initial steady state, unless otherwise noted.}}{}{Figure}{\special{language
"Scientific Word";type "GRAPHIC";maintain-aspect-ratio TRUE;display
"USEDEF";valid_file "T";width 7.2826in;height 4.4114in;depth
0pt;original-width 17.7952in;original-height 9.7421in;cropleft
"0.0940";croptop "1";cropright "1";cropbottom "0";tempfilename
'M3V4P302.wmf';tempfile-properties "XPR";}}

The government's protracted fiscal problems in the first years also stem
from disappointing effects on private consumption and investment.
Unfortunately, this disappointment is a robust feature of the transition
path. Because savers cannot smooth consumption by borrowing from abroad
(imperfect capital mobility) and non-savers live hand to mouth, the sharp
increase in taxes translates into a reduction in consumption. The positive
impact of increases in the stock of infrastructure on future productivity
could spur, in theory, an immediate increase in private investment. But this
is not possible. In our base case, the increase in the real interest
rate---a by-product of the significant VAT adjustment---is enough to force
savers to cut investment in the short run.\footnote{%
To see why the interest rate rises when taxes go up significantly, recall
the Euler equation for savers consumption, which can be written as:%
\begin{equation*}
r_{t}=\frac{1+g}{\beta }\left( \frac{1+h_{t+1}}{1+h_{t}}\right) \left( \frac{%
c_{t+1}^{\mathfrak{s}}}{c_{t}^{\mathfrak{s}}}\right) ^{1/\tau }-1.
\end{equation*}%
} In summary, private consumption and investment contract relative to their
initial steady state in the short run: by the year 2 consumption and
investment decrease by about $1\%$ and then stay depressed for another
couple of years before they start rising slowly.

Due to the period of stagnant investment, growth of the private capital
stock and real income lag far behind growth in the stock of infrastructure.
At the typical 20-year horizon of the Excel-based DSAs and relative to the
steady state, public capital has increased almost $45\%,$ which corresponds
to the $90\%$ of the $50\%$ increase across steady states; but private
capital is only about $3.3\%$ higher (vs.\thinspace almost $13\%$ higher
across steady states).\footnote{%
It is common practice to discount econometric estimates of the q-elasticity
of investment and assign $\Omega $ a value of 5-10 to speed up adjustment of
the capital stock. That does not work in our case. When $\Omega =10$, the
increases in private capital and real GDP are still small.} Growth (in per
capita terms) increases above $2\%$ in the first decade, but by year 20 is
close to $1.8\%,$ which is only $0.3$ percentage points higher than the
exogenously assumed long-run growth rate of $1.5\%.$ Because income and the
tax base grow slowly during the borrowing phase, the government is
ill-prepared for the subsequent repayment phase. The revenue demands of debt
service keep the VAT above $18\%$, which, in turn, slows the pace of capital
accumulation and perpetuates the fiscal bind. Eventually policy makers are
rewarded for their perseverance with a tiny revenue windfall and a moderate
increase in the private capital stock, but this occurs in a distant future.

As expected, total public debt follows the sustainable path of concessional
debt. Public debt peaks at year $8$ reaching $84\%$ of GDP and then declines
as the country repays the concessional loans and GDP increases. Because, by
construction, taxes always adjust to satisfy the government budget
constraint, unstable dynamics for debt are ruled out from the analysis.
Therefore the key question from the fiscal perspective is whether the tax
adjustment is feasible in practice. For the base case, an increase from $%
15\% $ to $18.9\%$ of the VAT in four years seems implausible.

\subsubsection{More Optimistic and Troublesome Scenarios}

\quad\ \thinspace We study now the dynamic implications of changing some of
the structural conditions of the economy. We consider conditions that
represent optimistic and troublesome scenarios.

\FRAME{ftbpFU}{7.146in}{3.7853in}{0pt}{\Qcb{\textbf{Unconstrained Tax
Adjustment: Optimistic and Troublesome Scenarios. }The transition paths for
taxes, private consumption, and private investment in optimistic and
troublesome scenarios, when the government takes out only concessional
loans. Variables are expressed as percentage deviations from the initial
steady state, unless otherwise noted.}}{}{Figure}{\special{language
"Scientific Word";type "GRAPHIC";maintain-aspect-ratio TRUE;display
"USEDEF";valid_file "T";width 7.146in;height 3.7853in;depth
0pt;original-width 17.7952in;original-height 9.3962in;cropleft "0";croptop
"1";cropright "1";cropbottom "0";tempfilename
'M3V4QC03.wmf';tempfile-properties "XPR";}}

Panel A of Figure 3 shows the paths of the tax rate, consumption, and
private investment of an optimistic scenario. The initial return on
infrastructure is $35\%$, user fees pay for all recurrent costs $f=1$ ($\mu
=0.1$), and the public investment surge is fully efficient $s=1$. The paths
of this scenario are notably better than those from the base case,
especially in the medium term. The peak of the VAT, which occurs in the
short run, is still high and close to $18\%$. But the medium-term tax is
much more tolerable. By year 20, the tax is below $16\%$. Moreover,
consumption does not necessarily contract in the first years and by year 4,
it is already on a fast growth trend. Interestingly, this optimistic
scenario delivers a more pronounced short-term crowding out of private
investment. The reason is simple. Since the capital account is closed,
avoiding a consumption contraction in the short term is at the expense of
reducing further private investment. Nevertheless, and relative to the base
case, the robust medium term outlook of investment seems to compensate these
initial negative effects.

It is also possible to simulate many troublesome scenarios. In the runs
presented in panel B of Figure 3, the return on infrastructure is relatively
low $R=0.1$, user fees finance only $20\%$ of recurrent costs $f=0.2$ ($\mu
=0.02$), and the public investment surge is very inefficient $s=0.2$. In
this scenario, taxes climb to almost $20\%$ in the year 7 and stay around
this level during the debt repayment period. This depresses consumption by
almost $4\%$ in the short and medium term. Investment rises very timidly
over time.

When, in addition, scaling up strains absorptive capacity, causing cost
overruns of about $25\%$ on new projects in the first year, the tax rate
skyrockets to $22\%$ by the fourth year (see Panel C of Figure 3). The
consequences for consumption and private investment are dramatic. Relative
to the initial steady state, consumption and investment fall by almost $4\%$%
. As public investment reaches its new permanent level, the absorptive
capacity costs diminish and consumption and private investment converge to
similar paths to those of the base case.

Certainly, the size of the public investment surge determines the severity
of the fiscal adjustment and the role that structural conditions of the
economy play for this adjustment. Small public investment shocks lead to
small and feasible tax adjustments. This is straightforward. As Figure 4
shows, when there is a small-scale expansion of public investment the tax
adjustment of the economy even with bad structural conditions (e.g., very
low return on public capital, user fees collection, efficiency, or
absorptive capacity) is very similar to that of the base case. This,
however, should not be taken as evidence to support the view that LICs
should \textit{not }improve their structural conditions. On the contrary,
because of their dire infrastructure, for instance, LICs are in need of
scaling up public investment; and, as Figures 3 and 4 reveal, it is
precisely under a scaling up that bad structural conditions matter for the
severity and the persistence of the tax adjustment.

\FRAME{ftbpFU}{7.146in}{3.9245in}{0pt}{\Qcb{\textbf{Unconstrained Tax
Adjustment: The Size of the Scaling Up.} Comparison of the tax adjustment
with and without a scaling up of public investment, under different
structural conditions. The increase of public investment in the no scaling
up scenario corresponds to 5\% of the increase in the scaling up scenario.}}{%
}{Figure}{\special{language "Scientific Word";type
"GRAPHIC";maintain-aspect-ratio TRUE;display "USEDEF";valid_file "T";width
7.146in;height 3.9245in;depth 0pt;original-width 17.7952in;original-height
9.7421in;cropleft "0";croptop "1";cropright "1";cropbottom "0";tempfilename
'M3V4S104.wmf';tempfile-properties "XPR";}}

\subsubsection{Gradually Increasing Transfers, Efficiency, and the
Collection Rate of User Fees}

\quad\ \thinspace So far we have assumed that government transfers are
constant in the analysis. Perhaps this is an extreme and, to some extent,
shaky assumption of the base case. After all, even if public investment is
not scaled up, LICs are expected to increase transfers associated with their
anti-poverty programs in order to achieve the Millennium Development Goals
(MDGs). Similarly, if public sector unions demand wage raises in line with
those granted in the private sector, it will be difficult to prevent
expenditure (transfers) from increasing. For these reasons, we examine a
scenario where the government increases transfers over time.

Our framework is flexible enough to incorporate different rules for
transfers.\footnote{%
When $\lambda >0,$ the fiscal adjustment falls on both taxes and transfers.
In this case the model can still allow for scenarios with downwardly
inflexible expenditure. For instance, transfers can be kept constant until
growth generates a fiscal windfall according to $\mathcal{T}_{t}=Max\{%
\mathcal{T}_{t}^{\text{target}},\mathcal{T}_{o}\},$ where $\mathcal{T}_{t}^{%
\text{target}}$ is defined (\ref{T_target}); or the government can be
assumed to be unable to cut other expenditures to offset increases in public
sector wages, implying that transfers evolve as 
\begin{equation*}
\mathcal{T}_{t}=Max\left\{ \mathcal{T}_{t}^{\text{target}},\mathcal{T}%
_{o}+0.05\,y_{o}\,\frac{(w_{t}-w_{o})}{w_{o}}\right\} ,
\end{equation*}%
where the initial public wage bill is 5\% of GDP and raises of this bill are
as large as raises in private sector wages.} For expositional reasons, here
we assume that transfers are raised as a fixed proportion $\mathfrak{m}$ of
increases in real GDP as described by%
\begin{equation*}
\mathcal{T}_{t}=\mathcal{T}_{o}+\mathfrak{m}\frac{\mathcal{T}_{o}}{y_{o}}%
(y_{t}-y_{o}),
\end{equation*}%
where $y_{t}^{r}=$ $P_{x,o}q_{x,t}+P_{n,o}q_{n,t}$ is real GDP at period $t$%
. To fix ideas, assume $\mathfrak{m}=1$.

\FRAME{ftbpFU}{7.146in}{3.9245in}{0pt}{\Qcb{\textbf{Unconstrained Tax
Adjustment: Increasing Transfers.} Comparison of the tax and private
consumption adjustments in the base case and in the case when government
transfers, the efficiency of public investment and user fees collection
increase over time. Variables are expressed as percentage deviations from
the initial steady state, unless otherwise noted.}}{}{Figure}{\special%
{language "Scientific Word";type "GRAPHIC";maintain-aspect-ratio
TRUE;display "USEDEF";valid_file "T";width 7.146in;height 3.9245in;depth
0pt;original-width 17.7952in;original-height 9.7421in;cropleft "0";croptop
"1";cropright "1";cropbottom "0";tempfilename
'M3V4TX05.wmf';tempfile-properties "XPR";}}

When transfers increase over time, the damage report for the fiscal
adjustment is grim, while the consumption benefits are nil. As shown in
panel A of Figure 5, taxes not only soar and reach a peak of $19.7\%$ at $%
t=7 $, but also stay above $19\%$ over most of the debt repayment period. Of
course transfers grow, and by year 20 their increase is above $0.6\%$ of
GDP. However, their effect on consumption is fully offset by the more
demanding tax adjustment. As a result, the path for consumption when
transfers increase is practically the same as that when transfers are kept
constant.

Does this mean that, in the context of public investment scaling ups, LICs
cannot raise transfers without both suffering drastic tax adjustments and
producing consumption benefits? The answer may lie, once more, on the key
role that structural conditions, such as efficiency of public investment and
the ability to collect user fees, play in the analysis. Unfortunately, there
is evidence that these structural conditions are weak in LICs. The public
investment management quality index of Dabla-Norris et al. (2011) suggests
that there is a considerable variation across LICs. But their efficiency is
in general lower than that of emerging and advanced economies.\footnote{%
\baselineskip=12pt In fact, public management capacity constraints in
themselves may call for further investments in capacity, i.e.
\textquotedblleft to invest in investing\textquotedblright\ in the words of
Collier (2007).} On the other hand, average user fees equal or slightly
exceed average costs for operation and maintenances of physical
infrastructure in SSA. But because of low collection rates, revenues cover
only half of recurrent costs, as discussed by Brice\~{n}o-Garmendia et al.
(2008) and Eberhard et al. (2008).

Improving the efficiency of public investment and the collection rate of
infrastructure user fees may allow LICs to increase transfers, while easing
fiscal adjustment. Raising both structural conditions to $100\%$ would
capture more revenue not only from the expansion of infrastructure services,
but also from services supplied by the existing network. Of course the
prevalence of low efficiency and collection rates suggest that the problem
is not easy to solve. But the transition from partial to full efficiency and
collection does not have to occur overnight to \textit{greatly} reduce the
demands made on taxes. As shown in panel B of Figure 5, even when it takes a
decade to complete the task, by year 20 the VAT is reduced to$\ 16.2\%$,
whereas the increase in transfers is about $1.25\%$ of GDP. As more
infrastructure-related fiscal revenue is available, then the government can
support a bigger increase in transfers than that of an economy where
efficiency and the collection rate do not improve. More importantly with
less tax adjustment and much higher transfers, consumption is boosted over
the medium term.

Undoubtedly some governments will succeed in raising efficiency and the
collection rate in upcoming decades, but this poses new questions.

First, even when efficiency improves, it is not clear whether the collection
rate would be expected to rise or fall over time. The complicating factor is
that scaling up public investment adds to the collection task by increasing
the share of the population with access to infrastructure services. If
collection capacity does not increase as fast as the supply of services,
which is proportional to the stock of \textit{effective} infrastructure, the
collection rate might fall.

Second, if steady, systematic increases in efficiency and the collection
rate are sure to happen in some country, then, given the magnitude of
infrastructure deficits in LICs, policy makers should aim to scale up
investment even more. A country that can achieve the paths shown in panel B
of Figure 5 might choose not settle for an increase in public investment of $%
3\%$ of initial GDP in the medium to long run; it could try to scale up much
more. But this brings us back to the main scenario where the challenge is to
ascertain how the scope for scaling up depends on feasible fiscal
adjustments in the VAT.

\subsection{Constrained Tax Adjustment Combined with External Commercial
Borrowing}

\quad\ \thinspace The analysis of the previous section suggests that the
transition can be challenging even when the investment is self-financing in
the long run. The most obvious remedy, investigated next, is to borrow
against future revenue gains in the external debt market with view to
smoothing the path of fiscal adjustment.

Increasingly LICs have access to external debt markets. In recent years,
Uganda, Tanzania, Senegal, Ghana, Angola, Congo DRC, Mali, Mauritania, and
Rwanda have all entered into non-concessional loan agreements or issued
sovereign bonds in international capital markets. Recognizing that the trend
is likely to continue, the World Bank and the IMF have adopted a more
flexible approach to non-concessional borrowing. The World Bank has
introduced the IDA Non-concessional Borrowing Policy and the IMF has
approved supplemental commercial loans for critical, large-scale
infrastructure projects in countries with IMF-supported programs.

In this sub-section, we thus study the fiscal and macroeconomic implications
of external commercial borrowing. In particular, the path of commercial debt
is backed out from the government budget constraint 
\begin{equation}
d_{c,t}=P_{z,t}\mathbb{I}_{t}+\frac{1+r_{d}}{1+g}d_{t-1}-d_{t}+\frac{%
1+r_{dc,t-1}}{1+g}d_{c,t-1}+\frac{r_{t-1}-g}{1+g}P_{t}b_{o}+\mathcal{T}%
_{t}-h_{t}P_{t}c_{t}-\mathcal{G}_{t}-\mathcal{N}_{t}-\mu z_{t-1}^{e},
\label{commercial_d}
\end{equation}%
while we assume the reactions functions described in (\ref{hmin})-(\ref%
{T_reaction}) with $x=d_{c}$ determine the path for transfers and taxes.%
\footnote{%
Note that this is equivalent to say that taxes and transfers are determined
by the reactions functions (\ref{hmin})-(\ref{T_reaction}), where $x=d_{c}$
and the gap $\mathfrak{Gap}_{t}$ is defined in (\ref{gapdef}); while the
path of external commercial debt is determined by equation (\ref{exgap}),
which is a different way to express the government budget constraint.}
Concessional borrowing follows the same path as before and commercial loans
are fully repaid ($d_{c}^{\text{target}}=0$). These reaction functions
subscribe to the view that the proposed borrowing \ and investment program
is sustainable only if debt converges to a stationary level without
violating sociopolitical constraints on how much and how fast fiscal policy
can change. Inside the caps $h^{u}$ and $\mathcal{T}^{l}$, the parameters $%
\lambda _{1}$-\thinspace $\lambda _{4}$ determine whether policy adjustment
is slow or fast; setting these parameters requires a realistic assessment of
the country's capacity for fiscal adjustment over different time horizons.

For expositional purposes, as in the case of concessional borrowing, in the
experiments to follow we keep transfers constant. To do so we set the lower
bound $\mathcal{T}^{l}$ equal to the initial value $\mathcal{T}_{o}.$ By the
reaction function (\ref{Tmax}), this ensures that $\mathcal{T}_{t}=\mathcal{T%
}_{o}$ at all times. Regarding the cap on taxes we set $h^{u}=0.17$
initially, but allow for a discrete jump at $t=8$ of $2$ percentage points,
so $h^{u}=0.19$.

\subsubsection{Tax Smoothing and Private Demand Crowding Out}

\quad\ \thinspace Access to the commercial debt market can make the dynamic
transition associated with the scaling up smoother. Early on,
non-concessional borrowing allows tax increases to be phased in more slowly.
Later, when the staggered limit of fiscal adjustment (upper bound on VAT)
has been reached during the repayment phase for concessional debt, new
borrowing can avert default or a collapse in public investment. And even if
scaling up is feasible without recourse to non-concessional debt, smoothing
the path of $h$ with non-concessional borrowing improves welfare by reducing
intertemporal distortions in consumption.

\FRAME{ftbpFU}{7.146in}{3.9245in}{0pt}{\Qcb{\textbf{Unconstrained Tax
Adjustment versus Constrained Tax A\textbf{djustment with} External
Commercial Borrowing.} Variables are expressed as percentage deviations from
the initial steady state, unless otherwise noted.}}{}{Figure}{\special%
{language "Scientific Word";type "GRAPHIC";maintain-aspect-ratio
TRUE;display "USEDEF";valid_file "T";width 7.146in;height 3.9245in;depth
0pt;original-width 17.7952in;original-height 9.7421in;cropleft "0";croptop
"1";cropright "1";cropbottom "0";tempfilename
'M3V4VF06.wmf';tempfile-properties "XPR";}}

Commercial borrowing can indeed help with adjustment, as shown by the
simulations in Figure 6. In these runs, the initial return on infrastructure
is $25\%$, but, due to a staggered cap on taxes--- first at $17\%$ and then
at $19\%$---and an assumed pace at which taxes can rise$,$ scaling up is
feasible only when external commercial borrowing supplements concessional
loans. The commercial loans also help smooth the initial tax adjustment: it
takes $11$ years, instead of the $6$ years of the base case, to raise taxes
from $15\%$ to $19\%.$ The country rides these fiscal constraints until
growth finally generates enough revenue to reconcile repayment of commercial
debt with decreases in the VAT. Relative to the base case of concessional
borrowing only, it is particularly impressive that private consumption and,
to some extent, private investment do not decline in the first years. Thus
the private demand crowding out problem seems to be practically gone. And
although there is a more pronounced real appreciation, the short-term
negative impact on traded output is not much worse than that of the base
case. In the medium term, both the traded and non-traded sectors take off,
generating more than sufficient economic growth to make public debt
sustainable.

\subsubsection{Debt Blowups: Structural and Policy Conditions}

\quad\ \thinspace There are also risks, of course, to taking out additional,
more expensive debt. If non-concessional borrowing serves merely to delay
the day of reckoning for an unsound investment program, then the ensuing
debt crisis will be more costly and more traumatic than otherwise. If extra
credit does more than facilitate fiscal adjustment, if instead it diminishes
fiscal effort too much for too long---one interpretation of delaying the
initial tax adjustment and lowering the values for $\lambda _{1}$ and
\thinspace $\lambda _{2}$ in the reaction function (\ref{h_reaction})---the
country may end up in a debt crisis. Borrowing-on-all-fronts is a high-risk,
high-return strategy. It may greatly enhance the prospects for debt
sustainability or lead to spectacular failure; much depends on the fine
details governing debt contracts, the structural conditions of the economy,
the dynamics of growth, and the speed of fiscal adjustment.

\FRAME{ftbpFU}{7.146in}{3.9245in}{0pt}{\Qcb{\textbf{Constrained Tax
Adjustment with External Commercial Borrowing: Varying the Structural and
Policy Conditions.}The transition paths for taxes and debt when the fiscal
adjustment is slower or public investment is very inefficient.}}{}{Figure}{%
\special{language "Scientific Word";type "GRAPHIC";maintain-aspect-ratio
TRUE;display "USEDEF";valid_file "T";width 7.146in;height 3.9245in;depth
0pt;original-width 17.7952in;original-height 9.7421in;cropleft "0";croptop
"1";cropright "1";cropbottom "0";tempfilename
'M3V4X007.wmf';tempfile-properties "XPR";}}

Policy makers might overestimate their capacity for fiscal adjustment,
finance low-return projects, misplace the blueprints for reforms to improve
governance and the efficiency of public investment, or repeatedly succumb to
the temptation to put off necessary but unpopular tax increases and
expenditure cuts. In this event we get some dreadful results, as exemplified
in Figure 7. The run in Panel A simulates the case when the government
delays raising the VAT. Because of this delay, the government loses the race
against time and the debt blows up. In Panel B the government borrows
heavily in the commercial market as it pays off its concessional debt,
counting on growth in future tax revenues to ensure debt sustainability. But
since public investment is very inefficient ($s=0.2$) little effective
public capital is accumulated, so future revenue gains are too small to
stabilize the debt or even to prevent it from increasing at an accelerating
rate. Similar results of explosive debt growth obtain if the return on
public capital, the capacity to collect user fees or the absorptive capacity
are low enough.

\subsection{Constrained Tax Adjustment Combined with Domestic Borrowing}

\quad\ \thinspace Although the DSAs have traditionally focused on the public
external public debt, there is broad agreement that domestic debt plays an
important role in overall debt sustainability. The IMF emphasizes that
better integration of domestic debt into the DSF is critical to early
detection of external debt vulnerability. In a similar vein, UNCTAD (2004)
has warned that controlling growth of domestic debt may prove the main
obstacle to debt sustainability in many LICs.

Accordingly, our last financing scenario involves a government that can sell
debt in the domestic market but not in world capital markets. Domestic debt
is then determined by the government budget constraint according to 
\begin{equation}
P_{t}b_{t}=P_{z,t}\mathbb{I}_{z,t}+\frac{1+r_{d}}{1+g}d_{t-1}-d_{t}+\frac{%
1+r_{t-1}}{1+g}P_{t}b_{t-1}+\mathcal{T}_{t}-h_{t}P_{t}c_{t}-\mathcal{G}_{t}-%
\mathcal{N}_{t}-\mu z_{t-1}^{e},  \label{domestic}
\end{equation}%
whereas the reactions functions described in (\ref{hmin})-(\ref{T_reaction})
with $x=b$ still determine the path for transfers and taxes.\footnote{%
Note that this is equivalent to say that taxes and transfers are determined
by the reactions functions (\ref{hmin})-(\ref{T_reaction}), where $x=b$ and
the gap $\mathfrak{Gap}_{t}$ is defined in (\ref{gapdef}); while the path of
domestic debt is determined by equation (\ref{exgap}), which is a different
way to express the government budget constraint.} Concessional borrowing
follows the same path described in the calibration section and new domestic
loans are fully repaid ($b^{\text{target}}=0.2$). As before, for
expositional purposes, we assume that all the burden of the fiscal
adjustment fall on taxes; so we set $\mathcal{T}^{l}=\mathcal{T}_{o}$ in (%
\ref{Tmax}), implying $\mathcal{T}_{t}=\mathcal{T}_{o}$ at all times. We
also assume that there is a staggered structure for taxes: $h^{u}=0.17$
initially and then jumps to $h^{u}=0.19.$

Domestic borrowing is more prone to trigger unsustainable public debt
dynamics than external commercial borrowing. This is for two reasons.
Perhaps the most obvious is that domestic loans are more expensive than
other sources of financing. In our calibration, the real interest rate on
domestic loans at the initial steady-state equilibrium is $4$ percentage
points higher than the rate paid on external commercial debt.\footnote{%
Interest payments on the internal debt are usually several times larger than
interest payments on the external debt, exceeding $5\%$ in some countries.
Barkbu et al.\thinspace (2008) also note that empirical studies find that
rising domestic debt significantly increases the likelihood of external debt
distress.} Moreover, it rises with more domestic borrowing. The second
reason domestic borrowing is less effective is that it does not provide
additional resources from abroad. Instead of taxing residents to extract the
resources necessary to finance the higher public investment, the government
borrows from them. But either way, the resources have to shift to the public
sector, and the scaling up requires a crowding out of private consumption
and investment. Thus if rates on domestic borrowing were the same as on
foreign borrowing, the adjustment path would be much more difficult with
domestic borrowing.\footnote{%
This is analogus to a spend-and-don't-aborb response to aid surges and
creates similar macroeconomic challenges, as analyzed in detail in Berg et
al. (2010a)}

Given the calibration for domestic interest rates and other parameters,
domestically-financed investment scenarios are thus more likely to be
explosive. Recall that external commercial borrowing yields a sustainable
trajectory with fiscal reaction functions implying that a $19\%$ cap on the
tax rate is delayed for more than $10$ years. In the same scenario financed
by domestic borrowing, public domestic debt explodes. To avoid this, the
same changes to assumptions that eased adjustment with commercial borrowing
would work here: a higher rate of return, higher efficiency, higher rate of
user fee collection, a smaller investment scaling up, etc. To compare the
macroeconomic dynamics associated with domestic borrowing to those with
external commercial borrowing for nonexplosive scenarios, we now speed up
fiscal adjustment in the domestic borrowing case to ensure debt
sustainability. This can be achieved by raising the parameter $\lambda _{1}$
of the reaction function (\ref{h_reaction}) from 0.25 to 0.45 and by
increasing the original caps $h^{u}$ by $0.5$ percentage points. As a
result, in Figure 8 taxes go up to $17.5\%$ by $t=5$ and to $19.5\%$ by $%
t=11 $. We can see the higher interest rates and much higher degree of
crowding out of private consumption and investment. Both combine to increase
fiscal pressure and require that taxes stay at $19.5\%$ percent for almost
two decades.

\FRAME{ftbpFU}{7.146in}{3.9245in}{0pt}{\Qcb{\textbf{Constrained Tax
Adjustment: Domestic Borrowing versus External Commercial Borrowing}.
Variables are expressed as percentage deviations from the initial steady
state, unless otherwise noted.}}{}{Figure}{\special{language "Scientific
Word";type "GRAPHIC";maintain-aspect-ratio TRUE;display "USEDEF";valid_file
"T";width 7.146in;height 3.9245in;depth 0pt;original-width
17.7952in;original-height 9.7421in;cropleft "0";croptop "1";cropright
"1";cropbottom "0";tempfilename 'M3V4YC08.wmf';tempfile-properties "XPR";}}

\section{External Shocks and Risks}

\quad\ \thinspace Our previous analyses have abstracted from the possibility
of unexpected exogenous shocks. It is said, however, that LICs are
particularly vulnerable to terms-of-trade shocks, natural disasters shocks,
and other adverse exogenous shocks.\footnote{%
For a discussion on importance of these external shocks in explaining the
instability of output and the macroeconomic fluctuations in LICs, see Kose
and Riezman (2001) and Raddatz (2007) and references therein.} This is to
some extent due to the structural characteristics of these countries: LICs
tend to be more dependent on primary commodities and have a higher exposure
to natural disasters. Furthermore, as LICs become more financially
integrated to the rest of the world by borrowing in capital markets, they
will start resembling emerging economies. As such, they will be subject more
frequently to risk premium shocks that affect the volatility of the real
interest rate at which they can borrow.\footnote{%
Neumeyer and Perri (2005), Uribe and Yue (2006) and Fern\'{a}ndez-Villaverde
et al. (2011) discuss how country risk premia, country spread and the
volatility of the real exchange rate, respectively, affect business cycles
in emerging economies.}

\FRAME{ftbpFU}{7.146in}{3.9245in}{0pt}{\Qcb{\textbf{External TOT Shocks and
Risks: Shocks Persistence and Financing Schemes}. The paths of tax and
public debt in the context of unexpected TOT shocks, with different degrees
of persistence (temporary versus permanent) and financing sources to
supplement concessional loans (external commercial borrowing versus more
concessional borrowing), when tax adjustment is constrained }}{}{Figure}{%
\special{language "Scientific Word";type "GRAPHIC";maintain-aspect-ratio
TRUE;display "USEDEF";valid_file "T";width 7.146in;height 3.9245in;depth
0pt;original-width 17.7952in;original-height 9.7421in;cropleft "0";croptop
"1";cropright "1";cropbottom "0";tempfilename
'M3V5030A.wmf';tempfile-properties "XPR";}}

We proceed then to investigate the implications for debt sustainability of 
\textit{big} unexpected negative terms-of-trade (TOT) shocks, negative
total-factor productivity (TFP) shocks\footnote{%
Kraay and Nerhu (2006) find that shocks to real GDP growth are highly
significant predictors of debt distress.}---as a way to model natural
disaster shocks---and shocks to the risk premium of public commercial debt.%
\footnote{%
Because in our analysis there is perfect foresight, to model \textit{%
unexpected} shocks we have to paste two dynamic systems. Consider, for
instance, how to model \textit{unexpected} TOT shocks. In this case, the
first system ignores TOT shocks and gives us the dynamic paths for all the
endogenous variables under the public investment scaling up. This allows us
to retrieve the values of these variables for any particular time when we
want to hit the economy with the TOT shock. Assume that, without loss of
generality, this time corresponds to $t=9.$ Then we run a second system that
(i) has as initial conditions for endogenous and exogenous variables the
values retrieved from the first system at $t=9$, (ii) includes the TOT
shocks and (iii) appropriately incorporates the continuation values for the
rest of exogenous variables such as public investment, grants, and
concessional borrowing. The final path for each variables is then
constructed by pasting the series of the first system up to $t=9$ and the
series of the second system beyond this point. Note that by our perfect
foresight assumption, this methodology implies that the initial decrease in
TOT shocks is the unexpected part. The rest of the shock, beyond $t=9,$ is
still expected. We leave for further work the generalization of this
procedure to look at sequences of unexpected shocks, though because of the
perfect foresight assumption, we cannot allow agents to react to uncertainty
per se.} To model TOT shocks we assume a decline of $10\%$ in the relative
price of exports $P_{x,t}$ and an increase of $20\%$ in the relative price
of intermediate inputs $P_{mm,t},$ which together imply a negative shock of $%
25\%$ to the TOT measured as $\frac{P_{x,t}}{P_{mm,t}}.$ Negative TFP shocks
are just introduced as a simultaneous and equal decrease of $5\%$ in the
productivity scale factors $a_{x}$ and $a_{n}$ in the technologies (\ref{qx}%
) and (\ref{qn}); while the risk premium shock is modelled as a jump in the
premium ${\small \upsilon }_{g}$ that on impact raises the real interest
rate on external commercial debt from $6\%$ to almost $10\%.$ For this last
shock we also turn on the mechanism in our model that allows for a variable
government debt risk premium. This is achieved by setting $\eta _{g}$ at a
positive value in our specification in equation (\ref{rdc}). Following van
der Ploeg and Venables (2011) we assume $\eta _{g}=1.89.$

\FRAME{ftbpFU}{7.146in}{3.9245in}{0pt}{\Qcb{\textbf{TFP and Risk Premium
Shocks and Risks. }The paths of tax and public debt in the context of
unexpected TFP and risk premium shocks, under constrained tax adjustment
combined with external commercial borrowing.}}{}{Figure}{\special{language
"Scientific Word";type "GRAPHIC";maintain-aspect-ratio TRUE;display
"USEDEF";valid_file "T";width 7.146in;height 3.9245in;depth
0pt;original-width 17.7952in;original-height 9.7421in;cropleft "0";croptop
"1";cropright "1";cropbottom "0";tempfilename
'M3V5150B.wmf';tempfile-properties "XPR";}}

In the analysis to follow we focus on the case where external commercial
loans close the financing gap.

In addition, we concentrate on unexpected shocks that hit at the time when
the economy is very vulnerable. We consider that this time is when the total
public debt to GDP ratio would have reached a maximum, if shocks were
absent. The relevance of this timing stems from the following reasons. It is
precisely at this point in time when (i) public debt is more likely to
breach one of the IMF-WB's DSF indicative thresholds or (ii) an unexpected
adverse shock is most likely to place, in principle, public debt on an
unsustainable path.\footnote{%
There is nothing, however, that prevents the possibility of explosive debt
dynamics if the unexpected shock hits the economy at any point in the rising
path of debt.}

Three results stand out from our analysis, summarized by the simulations
presented in Figures 9 and 10.

First, the persistence of the shock matters for debt sustainability. Panel A
of Figure 9 shows the implications of varying the persistence of an
unexpected TOT shock. It is clear that when the shock is temporary, the
country manages a tax adjustment that is sufficient to avoid explosive
dynamics for public commercial debt. Hence public debt is sustainable. But
as the persistence of the shock increases, the debt outcome may become
gloomy. If the shock were permanent, the adjustment would become
insufficient, given the caps on taxes. Therefore, if shocks are persistent
enough, debt sustainability problems may easily arise. Of course, this also
depends on the magnitude of the shocks and the scale of additional public
investment.

Second, in the context of unexpected shocks, additional concessional
financing instead of external commercial loans could greatly reduce the
risks of debt unsustainability. Panel B of Figure 9 shows that when an
unexpected TOT shock hits the economy, additional concessional borrowing to
finance the public investment surge, while setting caps in the tax
adjustment, can ensure debt sustainability. In contrast, using commercial
loans may be risky, as unsustainable debt paths become feasible. The crux of
the matter is whether LICs can obtain additional aid in the event of adverse
shocks.

Third, unexpected TFP shocks or government debt risk premium shocks can also
induce explosive paths for public debt when the government contracts
commercial loans. This is shown in Figure 10 that presents the effects on
debt accumulation of permanent negative TFP shocks and government debt
premium shocks. As the reader can easily infer from this figure, similar
considerations to those we raised in the context of TOT shocks---shock
persistence and type of additional financing---apply to these shocks. The
risk premium shock case is interesting because it captures the idea that if
market fears of debt repudiation rise and the risk premium rises with the
stock of debt, borrowing and investment programs that were previously
sustainable may become unsustainable.\footnote{%
A more interesting but much more challenging approach will be to endogenize
default in the model and make the risk premium depend on the
model-consistent risk of default.}

Summarizing, our analysis of unexpected shocks reveals that the strategy of
external commercial borrowing, although a valuable resource for development,
may be risky because it may give rise to public debt sustainability
problems. Size matters here as well: the smaller the scaling up, the larger
the unexpected shock required to cause explosive behavior.

\section{Concluding Remarks}

\quad\ \thinspace In this paper we have developed a fully-articulated,
dynamic macroeconomic model to support debt sustainability analysis in
low-income countries. Our model allows for financing schemes that mix
concessional, external commercial, and domestic debt, while taking into
account the impact of public investment on growth and constraints on the
speed and magnitude of fiscal adjustment. In this regard, our model can help
country authorities, IMF country teams, and others build a wide variety of
logically consistent scenarios of public investment surges and other shocks,
to inform and complement the currently used Excel-based DSA. Follow-on work
will develop the apparatus to bring the model to data and conduct DSAs in a
build-and -distribute approach.

We calibrate the model to the \textquotedblleft average\textquotedblright\
LIC and provide simulations for a public investment surge, showing the
macroeconomic consequences and potential risks associated with different
financing schemes. We find that when the economy can only borrow
concessionally to cover part of the surge, difficult fiscal and private
sector adjustments seem unavoidable in the short to medium term, especially
when the surge is front-loaded and the structural conditions of the economy
are weak. The strategy of allowing for external commercial borrowing in LICs
may ease these adjustments; but the presence of weak structural and policy
conditions and/or unexpected exogenous shocks calls for caution, as this
strategy is also prone to induce unsustainable public debt dynamics.
Domestic borrowing, on the other hand, is not as effective as external
commercial borrowing, because it does not provide additional external
resources, requires more drastic fiscal adjustments, and worsens the
crowding out of the private sector.

The model is already being used to support debt sustainability analysis in
country cases.\footnote{%
The model was used to evaluate the authorities' public investment scaling-up
strategies in Togo in the context of the debt sustainability analysis.
Moreover it was used to carry out counterfactual simulations using the
projections of the macroeconomic framework underlying the standard
Excel-based DSA. These simulations provided a useful check on the
plausibility and internal consistency of the projections made in the DSA.
See Andrle et al. (2012). The model has been also applied to Burkina Faso
and is being applied to Afghanistan, Cape Verde, Cote d'Ivoire, Ethiopia,
Ghana, and Senegal.} These applications have motivated extensions of the
model, including considering external commercial borrowing and domestic
borrowing at the same time, modeling regional market borrowing, studying the
implications of financing public investment with natural resource revenues,
modeling investment in security, and financial repression issues, among
others.

Still, a number of additional exercises and extensions would be useful. The
prospects for debt sustainability would depend on the composition of public
investment, including in human capital. Our model now assumes that all new
public investment is allocated to physical infrastructure projects with a
short gestation period. But even infrastructure can take many years to come
on line. And from a debt sustainability perspective, the distinctive feature
of human capital investments is their long gestation period. Also, user fees
finance a much smaller fraction of recurrent costs for education and health
than for power, roads, and water and sanitation. The time profiles for
supporting fiscal adjustment and debt are certain therefore to look very
different from the time profiles for investment in physical infrastructure.

Another important extension will be to more systematically incorporate
uncertainty and shocks. It may be feasible to generalize the method used to
analyze specific shocks in Section VI. Stochastic processes could be assumed
for exogenous shocks, to the terms of trade and TFP, for example. Some
elements of parameter uncertainty, e.g., about the rate of return to public
investment, might be explicitly addressed.\footnote{%
See Clinton et al. (2010) as an example, in a different context.}

Finally, we have not investigated the trade off between costs and benefits
of higher long-run debt levels. All the scenarios examined here assume
debt/GDP returns to initial levels. It is plausible, however, that by
raising the long-run level of debt, a country can forgo some of the long-run
benefits of the investment surge in return for an easier transition. In
considering such scenarios, it would be important to consider the risks
associated with carrying a permanently higher debt level.

%TCIMACRO{\TeXButton{appendix}{\begin{appendix}}}%
%BeginExpansion
\begin{appendix}%
%EndExpansion

\section*{Appendix}

\subsection{On Public Investment Efficiency, Rates of Return, and Growth}

\quad\ \thinspace This Appendix elaborates on the concept of
\textquotedblleft efficiency\textquotedblright\ and the closely related
concept of the rate of return to public capital. It serves as background
information for the calibration of efficiency and the rate of return. The
bottom line is that the operator needs to be thoughtful when calibrating
efficiency, particularly in considering what it might imply for the marginal
product of effective capital. A change in the calibration of efficiency has
two different interpretations: as a level change that applies to the past
and the future, as when comparing two countries at a point in time; or as a
change that applies to future investment but not the past, as for example
because of an improvement in public financial management. It is up to the
operator to decide which applies and whether other adjustments to the
calibration are necessary in light of that interpretation.

In the analysis to follow, for simplicity, we assume that there is only one
productive sector in the economy and set the long-run growth rate equal to
zero.

\textquotedblleft Efficiency\textquotedblright\ refers to the rate at which
spending on public investment translates into public capital. The model
distinguishes between the efficiency of steady-state public investment ($%
\bar{s}$) and the efficiency of the additional public investment associated
with the surge ($s$). Thus, reproducing equation (\ref{I_efficiency}) from
the text with $g=0$:%
\begin{equation}
z_{t}^{e}=(1-\delta )z_{t-1}^{e}+s(i_{z,t}-\bar{\imath}_{z})+\bar{s}\bar{%
\imath}_{z},  \label{i-eff}
\end{equation}%
where $i_{z,t}$ is public investment (measured in real dollars), $\delta $
is the rate of depreciation of public capital, $\bar{\imath}_{z}=(\delta +g)%
\bar{z}$ is the public investment at the (initial) steady state, and $%
z_{t}^{e}$ is the stock of \textquotedblleft effective\textquotedblright\
public capital. It is \textquotedblleft effective\textquotedblright\ because
it is what appears in the economy's production function: 
\begin{equation}
q_{t}=A_{t}\left( z_{t-1}^{e}\right) ^{\psi }\left( k_{t-1}\right) ^{\alpha
}\left( L_{t}\right) ^{1-\alpha }.  \label{prod}
\end{equation}%
The only nonstandard feature here is that $s$ and $\bar{s}$ are not
necessarily equal to 1. If they were, then the usual capital accumulation
equation $z_{t}^{e}=(1-\delta )z_{t-1}^{e}+i_{z,t}$ would obtain. When they
are less than 1, a given rate of public investment results in a smaller
accumulation of effective public capital. What is the effect of additional
public investment on growth, and how does it depend on efficiency? The
starting point is that the effect of additional public investment on
output---the marginal product of investment $\left( \frac{\partial q_{t}}{%
\partial i_{z,t-1}}\right) $ or $MP_{i_{z}}$---depends on \textit{both} the
marginal product of effective public capital---$\left( \frac{\partial q_{t}}{%
\partial z_{t-1}^{e}}\right) $ or $MP_{z^{e}}$---and on the efficiency with
which public investment spending translates into effective public capital: 
\begin{equation}
MP_{i_{z}}\equiv \frac{\partial q_{t}}{\partial i_{z,t-1}}=\left( \frac{%
\partial q_{t}}{\partial z_{t-1}^{e}}\right) \left( \frac{\partial
z_{t-1}^{e}}{\partial i_{z,t-1}}\right) .  \label{MPI}
\end{equation}

Consider first the marginal product of effective public capital ($MP_{z^{e}}$%
), which is equal to the rate of return on effective public capital ($R_{o}$%
) plus the depreciation rate $\delta $. Taking derivatives of the production
function and rearranging terms (and suppressing time subscripts):%
\begin{equation}
MP_{z^{e}}\equiv \left( \frac{\partial q}{\partial z^{e}}\right) =\psi
A\left( z^{e}\right) ^{\psi -1}\left( k\right) ^{\alpha }\left( L\right)
^{1-\alpha }=\psi \frac{q}{z^{e}},  \label{MPz}
\end{equation}%
That is, the marginal product of effective public capital is proportional to
the production function parameter $\psi $ and the output/public capital
ratio. The production function embodies decreasing returns to public
capital, holding other factors constant, and given the value of the
parameter $\psi $.

Efficiency has a simple direct effect: $\frac{\partial z^{e}}{\partial i_{z}}%
=s$ from equation (\ref{i-eff}). To understand efficiency, it may be useful
to imagine that all the available public investment projects at a given
point in time are ranked from highest to lowest rate of return. This ranking
reflects the decreasing marginal product of public capital shown in equation
(\ref{MPz}). The marginal product is thus the return of the best project
available. In a fully efficient investment process, when an additional
dollar is spent, this next best project is chosen. Suppose, though, because
of incompetence, corruption, or just imperfect information, a government
chooses worse projects. A lower efficiency---a lower $s$---is a measure of
the degree of deviation from the optimal process. Another---and
complementary---way to think about a value of $s$ below 1 is simply that a
fraction of spending is literally wasted, e.g. misclassified as investment
when it in fact just covers transfers to civil servants.

It would seem straightforward to obtain the growth impact of public
investment. In calibrating the model, the operator chooses directly a value
for the rate of return to effective public capital $R_{o}$, and thus of the
marginal product of public capital. The computer can solve for the implied
value of $\psi $, given the other parameters of the model. Meanwhile,
assumptions about the efficiency parameters $s$ and $\bar{s}$ determine the
value of $\frac{\partial z^{e}}{\partial i_{z}}$. These together determine
the output effect of additional public investment from equation (\ref{MPI}),
holding other factors constant. The other parameters and assumptions in the
model about fiscal policy, private sector behavior, \textquotedblleft Dutch
disease\textquotedblright , and so on yield the overall macroeconomic impact
of the investment surge.

There is an important potential complication, however: efficiency and the
marginal product of public capital may be related. Whenever---as with the
Cobb-Douglas specification of equation (\ref{prod})---the marginal product
of public capital is declining with larger stocks of public capital, this
marginal product will tend to be higher when efficiency has been low in the
past. A country with a low value of $\bar{s},$ for a given rate of
historical or steady-state public investment, will have a low stock of
effective public capital $z^{e}$.\footnote{%
To see this, it may help to note that, for a case in which $s=\bar{s}$,
equation (\ref{i-eff}) can be solved iteratively to obtain: $%
z_{t}^{e}=s\sum_{t=1}^{\infty }(1-\delta )^{k}i_{z,t-k}$, so that $z_{t}^{e}$
is proportional to $s$.} Because the stock of public capital is scarcer, the
marginal product of public capital will tend to be higher. This makes some
sense: if a country has difficulty converting spending into public capital,
it will tend to have very little capital, and even a small increment could
be highly productive; think of the first road from the capital city to the
port.

In the specific but widely-used case of the Cobb-Douglas production
function, which is that used in the model in the text, the $MP_{z^{e}}$ will
vary inversely with $z^{e}$, and thus with $\bar{s},$ for a given value of
.the production function parameter $\psi $. In this case, countries with
different levels of efficiency $\bar{s}$\ will have the same $MP_{i_{z}}$,
holding output, rates of public investment, and other model parameters
constant. Thus, a given increase in public investment spending will have the
same effect on output whatever the value of $\bar{s},$ again given output,
rates of public investment, and other model parameters.\footnote{%
There is empirical support for declining marginal product of public capital
and for the specific Cobb-Douglas version. Isham and Kaufman (1999) show
that the ex post rate of return on World Bank-financed projects (which may
correspond more closely to $MP_{z^{e}}$ than $MP_{i_{z}}$) rises with
\textquotedblleft good policy\textquotedblright\ (measured very generally as
low inflation, free trade, and so on) and falls with the capital/labor
ratio. This implies decreasing $MP_{z^{e}}$. Arslanalp et al. (2010)
estimate a Cobb-Douglas production function. They include a further term
related to the level of public capital and find a negative coefficient,
consistent with a deviation from Cobb-Douglas in which the $MP_{z^{e}}$
falls even faster than implied by Cobb-Douglas. Serv\'{e}n (2010) provides a
direct test in the context of estimated production functions. He finds that
the output elasticity of infrastructure (the percent change in output
associated with a given percent change in infrastructure) does not vary with
the stock of infrastructure per worker, supporting a Cobb-Douglas
specification.}

This inverse relationship between $\bar{s}$ and $MP_{z^{e}}$ has been
overridden for the user of the model because, as was described above, the
operator chooses $MP_{z^{e}}$ directly. This means that when the calibration
of $\bar{s}$ is changed, the computer finds a different value of $\psi $ as
required to preserve the value of the $MP_{z^{e}}$.\footnote{%
This approach to calibration can be rationalized with the observation that
different measures of policy are positively correlated: a country that
cannot efficiently build an electrical grid probably is not good at keeping
it running well. Thus, countries with low investment efficiency likely also
have low marginal product of installed capital, for a given capital stock
(i.e., a relatively low value of $\psi )$. This offsets the negative effect
of higher efficiency on the $MP_{z^{e}}$ that operates through the stock of
public capital.}

The upshot is that the operator needs to be thoughtful when calibrating
efficiency, particularly in considering what it might imply for the marginal
product of effective capital. In particular, a different calibration of $s$
can be understood in two different ways. It is thus up to the operator to
decide which interpretation applies and whether it calls for other
recalibrations:

\begin{itemize}
\item According to one interpretation, different calibrations of the value
of $s$ represent a time-invariant level difference in efficiency, e.g.
between two countries. Higher steady-state efficiency will imply a higher
marginal product of public \textit{investment} in the model, because the
marginal product of public \textit{capital} is given by assumption.\footnote{%
Whether $\bar{s}$ is actually recalibrated does not matter, because the
effect of any such change on the marginal product of investment spending is
undone by changes in $\psi $.} However, the operator should consider whether
this makes sense in a particular case. How is it that the country wastes
less public investment, and thus presumably has a higher public capital
stock, and yet still has the same marginal product of public capital?

\item The second interpretation is that this represents a change in
efficiency relative to the past. Such a change in efficiency through time
will have an unambiguous effect on the marginal product of public
investment. This is because it raises the effect of investment spending on
the growth rate of public capital, with no potentially offsetting effect on
the size of the stock. Thus, if the idea is that the country is improving
its efficiency relative to its own history, then there is no need to reflect
on possible implications for the marginal product of public capital.
\end{itemize}

Similar thoughtfulness would seem to apply in the application of the Fund's
policy on non-concessional borrowing (IMF, 2009b). The policy emphasizes the
role of efficiency framed as the capacity to manage public resources
well---along with debt levels---in determining whether non-concessional
borrowing for public investment is warranted. But from equation (\ref{MPI}),
the $MP_{z^{e}}$ also matters for the growth---and hence the debt
sustainability---impact of public investment. This may depend on the broad
policy environment and on the scarcity of public capital, and thus could be
negatively correlated with steady-state efficiency. Presumably, these
considerations need to be brought in through case-by-case analysis as
discussed in the policy.

%TCIMACRO{\TeXButton{appendix}{\end{appendix}}}%
%BeginExpansion
\end{appendix}%
%EndExpansion
\pagebreak

\begin{thebibliography}{99}
\bibitem{} Adam, C. and D. Bevan, 2006, \textquotedblleft Aid and the Supply
Side: Public Investment, Export Performance, and Dutch Disease in Low-Income
Countries,\textquotedblright\ \ The World Bank Economic Review, Vol. 20(2),
pp. 261-290.

\bibitem{} Agenor, R. and P. Montiel, 1999, \textit{Development
Macroeconomics, }2$^{\text{nd}}$ edition (Princeton, N.J.; Princeton
University Press).

\bibitem{} Agenor, R., 2010, \textquotedblleft A Thoery of
Infrastructure-Led Development,\textquotedblright\ Journal of Economics
Dynamics and Control, Vol. 34(5), pp. 932-950.

\bibitem{} Andrle, M., A. David, R. Espinoza, M. Mills, and L.F. Zanna,
2012, \textquotedblleft As You Sow So Shall You Reap: Public Investment
Surges, Growth, and Debt Sustainability in Togo,\textquotedblright\ IMF
Working Paper, forthcoming.

\bibitem{} Arestoff, F. and C. Hurlin (2006), \textquotedblleft Estimates of
Government Net Capital Stocks for 26 Developing Countries,
1970-2002,\textquotedblright\ World Bank Policy Research Working Paper 3858.

\bibitem{} Arslanalp, Serkan Fabian Bornhorst, Sanjeev Gupta, and Elsa Sze,
2010, \textquotedblleft Public Capital and Growth\textquotedblright , IMF
Working Paper 10/175 (Washington: International Monetary Fund).

\bibitem{} Arnone, M. and A. Presbitero, 2010, \textit{Debt Relief
Initiatives: Policy Design and Outcomes}, forthcoming (Ashgate).

\bibitem{} Arrow, K., 1962, \textquotedblleft The Economic Implications of
Learning by Doing.\textquotedblright\ Review of Economic Studies 29, 153-173.

\bibitem{} Asea, P. and C. Reinhart, 1996, \textquotedblleft Le Prix de
l'Argent: How (Not) to Deal with Capital Flows.\textquotedblright\ Journal
of African Economies 5 (AERC Supplement), 231-272.

\bibitem{} Barkbu, B., C. Beddies, and M. Le Manchec, 2008,
\textquotedblleft The Debt Sustainability Framework for Low-Income
Countries.\textquotedblright\ IMF Occasional Paper, No.\thinspace 266.

\bibitem{} Barro, R., 1990, \textquotedblleft Government Spending in a
Simple Model of Endogenous Growth,\textquotedblright\ Journal of Political
Economy, Vol. 98(5), pp. S103-26.

\bibitem{} Barro, R. and X. Sala-i-Martin, 1992, \textquotedblleft Public
Finance in Models of Economic Growth,\textquotedblright\ Review of Economic
Studies, Vol. 59(4), pp. 645-61.

\bibitem{} Berg, A., T. Mirzoev, R. Portillo, and L.F. Zanna, 2010a,
\textquotedblleft The Short-Run Macroeconomics of Aid Inflows: Understanding
the Interaction of Fiscal and Reserve Policy,\textquotedblright\ IMF Working
Paper 10/65.

\bibitem{} Berg, A., J. Gottschalk, R. Portillo, and L.F. Zanna, 2010b,
\textquotedblleft The Macroeconomics of Medium-Term Aid Scaling-Up
Scenarios.\textquotedblright\ IMF Working Paper 10/160.

\bibitem{} Berg, A., R. Portillo, S. Yang, and L.F. Zanna, 2012,
\textquotedblleft Public Investment in Resource Abundant Low-Income
Countries,\textquotedblright\ IMF Working Paper, forthcoming .

\bibitem{} Blundell, R., 1988, \textquotedblleft Consumer Behavior: Theory
and Evidence --- A Survey.\textquotedblright\ Economic Journal 98, 16-65.

\bibitem{} Blundell, R., P. Pashardes, and G. Weber, 1993, \textquotedblleft
What do we Learn about Consumer Demand Patterns from Micro
Data?\textquotedblright\ American Economic Review 83, 570-597.

\bibitem{} Brice\~{n}o-Garmendia, C., K. Smits, and V. Foster, 2008,
\textquotedblleft Financing Public Infrastructure in Sub-Saharan Africa:
Patterns and Emerging Issues.\textquotedblright\ AICD Background Paper 15
(World Bank).

\bibitem{} Bohn, H., 1998, \textquotedblleft The Behavior of US Public Debt
and Deficits,\textquotedblright\ Quarterly Journal of Economics, Vol 113(3),
pp. 949-963.

\bibitem{} Calder\'{o}n, C., E. Moral-Benito, and L. Serv\'{e}n, 2009,
\textquotedblleft Is Infrastructure Capital Productive? A Dynamic
Heterogeneous Approach.\textquotedblright\ Mimeo (World Bank).

\bibitem{} Calder\'{o}n, C. and L. Serv\'{e}n, 2010, \textquotedblleft
Infrastructure in Latin America.\textquotedblright\ Policy Research Working
Paper, No.\thinspace 5317 (World Bank).

\bibitem{} Chatterjee, S. and S. Turnovsky, 2007, \textquotedblleft Foreign
Aid and Economic Growth: The Role of Flexible Labor
Supply,\textquotedblright\ Journal of Development Economics, Vol. 84(1), pp.
507--533.

\bibitem{} Celasun, O., X. Debrun, and J. D. Ostry, 2007, \textquotedblleft
Primary Surplus Behavior and Risks to Fiscal Sustainability in Emerging
Market Countries: A `Fan-Chart' Approach,\textquotedblright\ IMF Staff
Papers, Vol. 53(3), pp. 401-425.

\bibitem{} Cerra, V., S. Tekin, and S. Turnovsky, 2008, \textquotedblleft
Foreign Aid and Real Exchange Rate Adjustments in a Financially Constrained
Dependent Economy,\textquotedblright\ IMF Working Paper 08/204.

\bibitem{} Clinton, K., R. Garcia-Saltos, M. Johnson, O. Kamenik, and D.
Laxton, 2010, \textquotedblleft International Deflation Risks under
Alternative Macroeconomic Policies,\textquotedblright\ Journal of the
Japanese and International Economies, Vol. 24(2), pp. 140-177.

\bibitem{} Collier, P., 2007, \textquotedblleft The Bottom Billion: Why the
Poorest Countries are Failing and What Can Be Done About
It,\textquotedblright\ Oxford University Press, New York.

\bibitem{} Cowan, D., 2010, \textquotedblleft Sub-Saharan Africa Macro
View.\textquotedblright\ Citigroup Global Markets.

\bibitem{} Dabla-Norris, E., J. Brumby, C. Papageorgiou, A. Kyobe, and Z.
Mills, 2011, \textquotedblleft Investing in Public Investment: An Index of
Public Investment Management Quality,\textquotedblright\ IMF Working Paper,
11/37.

\bibitem{} Dagher, J., J. Gottschalk, and R. Portillo, 2012,
\textquotedblleft The Short-run Impact of Oil Windfalls in Low-income
Countries: A DSGE Approach.\textquotedblright\ Journal of African Economies,
forthcoming.

\bibitem{} Dalgaard, C. and H. Hansen, 2005, \textquotedblleft The Return to
Foreign Aid.\textquotedblright\ Discussion Paper No.\thinspace 05-04,
Institute of Economics, University of Copenhagen.

\bibitem{} Deaton, A., and D. Muellbauer, 1980, \textit{Economics and
Consumer Behavior} (New York, Cambridge University Press).

\bibitem{} Eaton, J., 2002, \textquotedblleft The HIPC Initiative: The
Goals, Additionality, Eligibility, and Debt
Sustainability.\textquotedblright\ Mimeo (Operations Evaluation Department,
World Bank).

\bibitem{} Eberhard, A., V. Foster, C. Bricendo-Garmendia, F. Ouedraogo, D.
Camos, and M. Shkaratan, 2008, \textquotedblleft Underpowered: The State of
the Power Sector in Sub-Saharan Africa.\textquotedblright\ AICD Background
Paper 6 (World Bank).

\bibitem{} Escribano, A., J. Guasch, and J. Pena, 2008, \textquotedblleft
Impact of Infrastructure Constraints on Firm Productivity in
Africa.\textquotedblright\ AICD Working Paper No.\thinspace 9 (World Bank).

\bibitem{} EURODAD, 2001, \textquotedblleft Putting Poverty Reduction
First.\textquotedblright\ (European Network on Debt and Development)

\bibitem{} EURODAD, 2009, \textquotedblleft Review of the DSF: Bank \& Fund
Declare Countries Can Borrow More.\textquotedblright\ (European Network on
Debt and Development)

\bibitem{} Fedelino, A. and A. Kudina, 2003, \textquotedblleft Fiscal
Sustainability in African HIPC Countries: A Policy
Dilemma?\textquotedblright\ IMF Working Paper, No.\thinspace 03/187.

\bibitem{} Fern\'{a}ndez-Villaverde, J., P. Guerr\'{o}n-Quintana, J.
Rubio-Ram\'{\i}rez, and M. Uribe, 2011. \textquotedblleft Risk Matters: The
Real Effects of Volatility Shocks,\textquotedblright\ American Economic
Review, Vol. 101(6), pp. 2530-2561.

\bibitem{} Foster, V. and C. Brice\~{n}o-Garmendia, 2010, \textit{Africa's
Infrastructure: A Time for Transformation}, forthcoming (Agence Francaise de
Developpement and the World Bank).

\bibitem{} Futagami, K., Y. Morita, and A. Shibata, 1993, \textquotedblleft
Dynamic Analysis of an Endogenous Growth Model with Public
Capital,\textquotedblright\ Scandinavian Journal of Economics, Vol. 95(4),
pp. 607-25.

\bibitem{} Garcia, M. and R. Rigobon, 2005, \textquotedblleft A Risk
Management Approach to Emerging Market's Sovereign Debt Sustainability with
an Application to Brazilian Data,\textquotedblright\ in \textit{Inflation
Targeting, Debt, and the Brazilian Experience: 1999 to 2003}, editors,
Francesco Giavazzi, Ilan Goldfajn, and Santiago Herrera, MIT Press.

\bibitem{} Ghosh, A., J. Kim, E. Mendoza, J. D. Ostry, and M. Qureshi, 2011.
\textquotedblleft Fiscal Fatigue, Fiscal Space and Debt Sustainability in
Advanced Economies,\textquotedblright\ NBER Working Papers 16782, National
Bureau of Economic Research.

\bibitem{} Glomm, Gerhard and B. Ravikumar, 1994, \textquotedblleft Public
investment in infrastructure in a simple growth model,\textquotedblright\
Journal of Economic Dynamics and Control, Vol. 18(6), pp. 1173-1187.

\bibitem{} Greiner, A., 2007, \textquotedblleft An Endogenous Growth Model
with Public Capital and Sustainable Government Debt,\textquotedblright\ The
Japanese Economic Review, Vol. 58(3), pp. 345-361.

\bibitem{} Greiner, A., W. Semmler, and G. Gong, 2005, \textit{The Forces of
Economic Growth: A Time Series Perspective, }1$^{\text{st}}$ edition
(Princeton, N.J.; Princeton University Press).

\bibitem{} Gueye, C. and A. Sy, 2010, \textquotedblleft Beyond Aid: How Much
Should African Countries Pay to Borrow?\textquotedblright\ IMF Working
Paper, No.\thinspace 10/140.

\bibitem{} Gupta, S., R. Powell, and Y. Yang, 2006, \textit{Macroeconomic
Challenges of Scaling Up Aid to Africa: A Checklist for Practitioners}
(International Monetary Fund).

\bibitem{} Hjertholm, P., 2003, \textquotedblleft Theoretical and Empirical
Foundations of HIPC Debt Sustainability Targets.\textquotedblright\ Journal
of Development Studies 39, 67-100.

\bibitem{} Hulten, C., 1996, \textquotedblleft Infrastructure Capital and
Economic Growth: How Well You Use It May Be More Important Than How Much You
Have.\textquotedblright\ NBER Working Paper, No.\thinspace 5847.

\bibitem{} Hulten, C., E. Bennathan, and S. Srinivasan, 2006,
\textquotedblleft Infrastructure, Externalities, and Economic Development: A
Study of Indian Manufacturing.\textquotedblright\ World Bank Economic Review
20, xx-xx.

\bibitem{} International Monetary Fund, 2007a, \textquotedblleft Regional
Economic Outlook: Sub-Saharan Africa,\textquotedblright\ SM/07/319,
available at
http://www.imf.org/external/pubs/ft/reo/2007/afr/eng/sreo1007.pdf

\bibitem{} International Monetary Fund, 2007b, \textquotedblleft Ghana:
Staff Report for the 2007 Article IV Consultation,\textquotedblright\
SM/07/156, available at
http://www.imf.org/external/pubs/ft/scr/2007/cr07210.pdf

\bibitem{} International Monetary Fund, 2009a, \textquotedblleft Changing
Patterns in Low-Income Country Financing and Implications for Fund Policies
on External Financing and Debt,\textquotedblright\ SM/09/56, available at
http://www.imf.org/external/np/pp/eng/2009/022509a.pdf

\bibitem{} International Monetary Fund, 2009b, \textquotedblleft Debt Limits
in Fund-Supported Programs: Proposed New Guidelines\textquotedblright ,
SM/09/215, available at http://www.imf.org/external/np/pp/eng/2009/080509.pdf

\bibitem{} International Monetary Fund, 2010, \textquotedblleft Staff
Guidance Note on the Application of the Joint Fund-Bank Debt Sustainability
Framework for Low-Income Countries,\textquotedblright\ SM/10/16, available
at http://www.imf.org/external/np/pp/eng/2010/012210.pdf

\bibitem{} International Monetary Fund and World Bank, 2006,
\textquotedblleft Applying the Debt Sustainablity Framework for Low-Income
Countries: Post-Debt Relief,\textquotedblright\ SM/06/364, available at
http://www.imf.org/external/np/pp/eng/2006/110606.pdf

\bibitem{} International Monetary Fund and World Bank, 2009,
\textquotedblleft A Review of Some Aspects of the Low-Income Country Debt
Sustainability Framework,\textquotedblright\ SM/09/216, available at
http://www.imf.org/external/np/pp/eng/2009/080509a.pdf

\bibitem{} International Monetary Fund and World Bank, 2010,
\textquotedblleft Preserving Debt Sustainability in Low-Income Countries in
the Wake of the Global Crisis,\textquotedblright\ SM/10/76 , available at
http://www.imf.org/external/np/pp/eng/2009/080509a.pdf

\bibitem{} Isham, Jonathan and Daniel Kaufmann, 1999, \textquotedblleft The
Forgotten Rationale For Policy Reform: The Productivity of Investment
Projects\textquotedblright\ Quarterly Journal of Economics, February, Vol.
114, No. 1, pp. 149--184.

\bibitem{} Kose, M. A., and R. Riezman, 2001, \textquotedblleft Trade Shocks
and Macroeconomic Fluctuations in Africa,\textquotedblright\ Journal of
Development Economics, Vol. 65(1), pp. 55-80.

\bibitem{} Kraay, A. and V. Nehru, 2006, \textquotedblleft When Is External
Debt Sustainable?,\textquotedblright\ World Bank Economic Review, Vol.
20(3), pp. 341-365.

\bibitem{} Lluch, C., A. Powell, and R. Williams, 1977, \textit{Patterns in
Household Demand and Saving} (London, Oxford University Press).

\bibitem{} Mendoza, E. and M. Oviedo, 2004, \textquotedblleft Public Debt,
Fiscal Solvency and Macroeconomic Uncertainty in Latin America: The Cases of
Brazil, Colombia, Costa Rica, and Mexico,\textquotedblright\ NBER Working
Papers 10637, National Bureau of Economic Research.

\bibitem{} Neumeyer, P., and F. Perri, 2005, \textquotedblleft Business
Cycles in Emerging Economies: the Role of Interest Rates,\textquotedblright\
Journal of Monetary Economics, Vol. 52(2), pp. 345-380.

\bibitem{} Ogaki, M., J. Ostry, and C. Reinhart, 1996, \textquotedblright
Saving Behavior in Low- and Middle-Income Developing
Countries.\textquotedblright\ IMF Staff Papers 43, 38-71.

\bibitem{} Panizza, U., 2008, \textquotedblleft Domestic and External Public
Debt in Developing Countries.\textquotedblright\ UNCTAD Discussion Paper,
No.\thinspace 188.

\bibitem{} Perrault, J., L. Savard, and A. Estache, 2010, \textquotedblleft
The Impact of Infrastructure Spending in Sub-Saharan Africa: A CGE Modeling
Approach.\textquotedblright\ Policy Research Working Paper, No.\thinspace
5386 (World Bank).

\bibitem{} Pritchett, L., 2000, \textquotedblleft The Tyranny of Concepts:
CUDIE (Cumulated, Depreciated, Investment Effort) is Not
Capital.\textquotedblright\ Journal of Economic Growth 5, 361-384.

\bibitem{} Raddatz, C., 2007, \textquotedblleft Are External Shocks
Responsible for the Instability of Output in Low-Income
Countries?,\textquotedblright\ Journal of Development Economics, Vol. 84(1),
pp. 155-187.

\bibitem{} Redifer, L., 2010, \textquotedblleft New Financing Sources for
Africa's Infrastructure Deficit.\textquotedblright\ IMF Survey Magazine:
Countries and Regions.

\bibitem{} Sachs, J., 2002, \textquotedblleft Resolving the Debt Crisis of
Low-Income Countries.\textquotedblright\ Brookings Papers on Economic
Activity 1, 257-286.

\bibitem{} Schmitt-Groh\'{e}, S. and M. Uribe, 2003, \textquotedblleft
Closing Small Open Economy Models,\textquotedblright\ Journal of
International Economics, Vol. 61(1), pp. 163--185.

\bibitem{} Serv\'{e}n, Luis, 2010, \textquotedblleft Infrastructure
investment and growth\textquotedblright , World Bank, in progress;
presentation available at
http://www.imf.org/external/np/seminars/eng/2010/spr/lic/index.htm

\bibitem{} Steadman Group (2009), \textquotedblleft Financial Access Survey
for Financial Sector Deepening,\textquotedblright\ Presentation.\ 

\bibitem{} Straub, S., 2008, \textquotedblleft Infrastructure and Growth in
Developing Countries: Recent Advances and Research
Challenges.\textquotedblright\ Policy Research Working Paper, No.\thinspace
4460 (World Bank).

\bibitem{} Thurlow, J., D. Evans, and S. Robinson, 2004, \textquotedblleft A
2001 Social Accounting Matrix for Zambia.\textquotedblright\ International
Food Policy Research Institute (Washington, D.C.)

\bibitem{} Thurlow, J., X. Diao, and C. McCool, 2008, \textquotedblleft 2004
Social Accounting Matrix: Malawi.\textquotedblright\ International Food
Policy Research Institute.

\bibitem{} Turnovsky, S., 1999, \textit{Methods of Macroeconomic Dynamics, }2%
$^{\text{nd}}$ edition (Cambridge, Mass.; MIT Press).

\bibitem{} Uribe, M. and V. Yue, 2006, \textquotedblleft Country Spreads and
Emerging Countries: Who Drives Whom?,\textquotedblright\ Journal of
International Economics, Vol. 69(1), pp. 6-36.

\bibitem{} UNCTAD, 2004, \textquotedblleft Analysis of Eligibility and Debt
Sustainability Criteria of the HIPC Initiative.\textquotedblright\ Chapter 2
in \textit{Debt Sustainablility: Oasis or Mirage? }(United Nations
Conference on Trade and Development).

\bibitem{} van der Ploeg, F. and A. Venables, 2011, \textquotedblleft
Harnessing Windfall Revenues: Optimal Policies for Resource-Rich Developing
Economies,\textquotedblright\ Economic Journal, Vol. 121(551), pp. 1-30.

\bibitem{} Wyplosz, C., 2007, \textquotedblleft Debt Sustainability
Assessment: The IMF Approach and Alternatives.\textquotedblright\ HEI
Working Paper, No.\thinspace 03/2007 (Graduate Institute of International
Studies, Geneva).
\end{thebibliography}

\end{document}
