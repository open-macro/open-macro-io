
\documentclass[11pt]{article}

\usepackage{amsfonts}
\usepackage{amssymb}
\usepackage{graphicx}
\usepackage{amsmath}
\usepackage{lineno}
\usepackage{fancybox}
\usepackage{rotating}
\usepackage{fancyhdr}
\usepackage{sectsty}

\begin{document}

\section{The Model}

\quad\ \thinspace Our framework is the standard two-sector model of a small
open economy embellished with multiple types of public sector debt and
multiple tax and spending variables. The country produces a traded good $%
q_{x}$ and a non-traded good $q_{n}$ from private capital $k$, labor $L$,
and government-supplied infrastructure $z$. Besides these domestically
produced goods, agents can import a traded good for consumption $c_{m}$ and
machines $\mathfrak{m}_{mm}$ to produce factories (private capital) and
infrastructure (public capital). All quantity variables except labor are
detrended by $(1+g)^{t}$, where $g$ is the exogenous long-run growth rate of
real GDP.\footnote{%
In the long run all variables, including real GDP, grow at the same
exogenous growth rate $g.$\ However, in the short to medium term,
significant public and private capital accumulation, resulting from scaling
up investment, implies that the growth rate of the economy can go above $g.$}
A composite good produced abroad is the numeraire, with the associated
consumer price index (CPI) denoted by $P_{t}^{\ast },$ which is assumed to
be equal to one for simplicity. Since the time horizon for the DSA is about
20 years, the model abstracts from money and all nominal rigidities.%
\footnote{%
We are currently working on a version that includes money and nominal price
rigidities along the lines of the model in Berg et al. (2010b).}

We lay out the model in stages, starting with the specification of
technology.

\subsection{Firms}

\subsubsection{Technology}

\quad\ \thinspace In each sector $j$, the representative firm use
Cobb-Douglas technologies to convert labor $L_{j,t}$, private capital $%
k_{j,t-1}$, and effectively productive infrastructure $z_{t-1}^{e}$, which
is a public good, into output:\footnote{%
We assume Cobb-Douglas technologies but, to some extent, we do not expect
significant changes in our results by considering CES technologies.} 
\begin{equation}
q_{x,t}=A_{x,t}\left( z_{t-1}^{e}\right) ^{\psi _{x}}\left( k_{x,t-1}\right)
^{\alpha _{x}}\left( L_{x,t}\right) ^{1-\alpha _{x}},  \label{qx}
\end{equation}%
and%
\begin{equation}
q_{n,t}=A_{n,t}\left( z_{t-1}^{e}\right) ^{\psi _{n}}\left( k_{n,t-1}\right)
^{\alpha _{n}}\left( L_{n,t}\right) ^{1-\alpha _{n}}.  \label{qn}
\end{equation}%
The firm productivities are expressed as 
\begin{equation*}
A_{x,t}=a_{x}\left( \frac{q_{x,t-1}^{I}}{\bar{q}_{x}^{I}}\right) ^{\sigma
_{x}}\left( k_{x,t-1}^{I}\right) ^{\xi _{x}}\text{ \ \ \ \ \ \ \ and \ \ \ \
\ \ \ \ }A_{n,t}=a_{n}\left( \frac{q_{n,t-1}^{I}}{\bar{q}_{n}^{I}}\right)
^{\sigma _{n}}\left( k_{n,t-1}^{I}\right) ^{\xi _{n}}
\end{equation*}%
and feature sector-specific externalities of two types, with variables with
the superindex $I$ denoting sectoral quantities: a \textquotedblleft
static\textquotedblright\ externality associated with private capital
accumulation---$\left( k_{j,t-1}^{I}\right) ^{\xi _{j}}$ for $j=x,n$ ---as
in Arrow (1962); and a \textquotedblleft
learning-by-doing\textquotedblright\ externality that depends on the
deviations of the lagged sector output from the (initial) steady state---$%
\left( \frac{q_{j,t-1}^{I}}{\bar{q}_{j}^{I}}\right) ^{\sigma j}$ for $j=x,n$%
. When the latter externality is greater in the traded sector than that in
the non-traded sector, then it can capture the notion of Dutch-disease as in
Berg et al. (2010b), where a decline in the traded sector imposes an
economic cost through a sectoral loss in total-factor productivity (TFP).

Factories and infrastructure are built by combining one imported machine
with $a_{j}$ ($j=k,z$) units of a non-traded input (e.g., construction). The
supply prices of private capital and infrastructure are thus 
\begin{equation}
P_{k,t}=P_{mm,t}+a_{k}P_{n,t},  \label{Pk}
\end{equation}%
and%
\begin{equation}
P_{z,t}=P_{mm,t}+a_{z}P_{n,t},  \label{Pz}
\end{equation}%
where $P_{n}$ is the (relative) price of the non-traded good and $P_{mm}$ is
the (relative) price of imported machinery.

\subsubsection{Factor Demands}

\quad\ \thinspace Competitive profit-maximizing firms equate the marginal
value product of each input to its factor price. This yields the input
demand equations 
\begin{equation}
P_{n,t}(1-\alpha _{n})\frac{q_{n,t}}{L_{n,t}}=w_{t},  \label{Ln_DD}
\end{equation}%
\begin{equation}
P_{x,t}(1-\alpha _{x})\frac{q_{x,t}}{L_{x,t}}=w_{t},  \label{Lx_DD}
\end{equation}%
\begin{equation}
P_{n,t}\alpha _{n}\frac{q_{n,t}}{k_{n,t-1}}=r_{n,t},  \label{Kn_DD}
\end{equation}%
and%
\begin{equation}
P_{x,t}\alpha _{x}\frac{q_{x,t}}{k_{x,t-1}}=r_{x,t},  \label{Kx_DD}
\end{equation}%
where $w$ is the wage and $r_{j}$ is the rental earned by capital in sector $%
j$. Labor is intersectorally mobile, so the same wage appears in (\ref{Ln_DD}%
) and (\ref{Lx_DD}). Capital is sector-specific, but $r_{x}$ differs from $%
r_{n}$ only on the transition path. After adjustment is complete and $k_{x}$
and $k_{n}$ have settled at their equilibrium levels, the rentals are equal.

\subsection{Consumers}

\quad\ \thinspace There are two types of private agents, savers and
non-savers, with the former and the latter distinguished by the superscripts 
$\mathfrak{s}$ and $\mathfrak{h,}$ respectively. Labor supply of savers is
fixed at $L^{\mathfrak{s}}$ while that of non-savers is $L^{\mathfrak{h}%
}=aL^{\mathfrak{s}}$ with $a>0$. The two types of agents consume the
domestic traded good $c_{x,t}^{i}$, the foreign traded good $c_{m,t}^{i}$,
and the domestic non-traded good $c_{n,t}^{i}$ for $i=\mathfrak{s},\mathfrak{%
h}.$ These goods are combined into a CES basket%
\begin{equation}
c_{t}^{i}=\left[ \rho _{x}^{\frac{1}{\epsilon }}\left( c_{x,t}^{i}\right) ^{%
\frac{\epsilon -1}{\epsilon }}+\rho _{m}^{\frac{1}{\epsilon }}\left(
c_{m,t}^{i}\right) ^{\frac{\epsilon -1}{\epsilon }}+(\rho _{n})^{\frac{1}{%
\epsilon }}\left( c_{n,t}^{i}\right) ^{\frac{\epsilon -1}{\epsilon }}\right]
^{^{\frac{\epsilon }{\epsilon -1}}}\text{ \ \ \ for }i=\mathfrak{s},%
\mathfrak{h}  \label{c_basket}
\end{equation}%
where $\rho _{x}$, $\rho _{m}$, and $\rho _{n}$ are CES distribution
parameters and $\epsilon $ is the intratemporal elasticity of substitution.
In addition $\rho _{n}=1-\rho _{x}-\rho _{m}$.

The (relative) CPI associated with the basket (\ref{c_basket}) is $P_{t}=%
\left[ \rho _{x}P_{x,t}^{1-\epsilon }+\rho _{m}P_{m,t}^{1-\epsilon }+\rho
_{n}P_{n,t}^{1-\epsilon }\right] ^{^{\frac{1}{1-\epsilon }}},$ while the
demand functions for each good can be expressed as 
\begin{equation*}
c_{j,t}^{i}=\rho _{j}\left( \frac{P_{j,t}}{P_{t}}\right) ^{-\epsilon
}c_{t}^{i}\text{ \ \ \ \ \ \ for \ \ \ }j=x,m,n\ \ \ \text{and \ \ }\ \text{ 
}i=\mathfrak{s},\mathfrak{h}.
\end{equation*}

Savers can invest $i_{x}$ and $i_{n}$ amounts in private capital that
depreciates at the rate $\delta $, pay user fees charged for infrastructure
services according to $\mu z^{e}$, can buy domestic bonds $b$---which cannot
be bought in any market by foreigners---and pay a real interest rate $r,$
and can contract foreign debt $b^{\ast }$ that charges an exogenous real
interest rate $r^{\ast }$. They solve the intertemporal problem

\begin{equation*}
Max\sum_{t=0}^{\infty }\beta ^{t}\frac{\left( c_{t}^{\mathfrak{s}}\right)
^{1-1/\tau }}{1-1/\tau },
\end{equation*}%
subject to

\begin{eqnarray}
P_{t}b_{t}^{\mathfrak{s}}-b_{t}^{\mathfrak{s}\ast } &=&r_{x,t}k_{x,t-1}^{%
\mathfrak{s}}+r_{n,t-1}k_{n,t-1}^{\mathfrak{s}}+w_{t}L_{t}^{\mathfrak{s}}+%
\frac{\mathcal{R}_{t}}{1+a}+\frac{\mathcal{T}_{t}}{1+a}-\frac{%
1+r_{t-1}^{\ast }}{1+g}b_{t-1}^{\mathfrak{s}\ast }+\frac{1+r_{t-1}}{1+g}%
P_{t}b_{t-1}^{\mathfrak{s}}  \notag \\[0.08in]
&&-P_{k,t}\left( i_{x,t}^{\mathfrak{s}}+i_{n,t}^{\mathfrak{s}}+AC_{x,t}^{%
\mathfrak{s}}+AC_{n,t}^{\mathfrak{s}}\right) -P_{t}c_{t}^{\mathfrak{s}%
}(1+h_{t})-\mu z_{t-1}^{e}-\mathcal{P}_{t}^{\mathfrak{s}}-\Phi _{t}^{%
\mathfrak{s}},  \label{BCs}
\end{eqnarray}

\begin{equation}
(1+g)k_{x,t}^{\mathfrak{s}}=i_{x,t}^{\mathfrak{s}}+(1-\delta )k_{x,t-1}^{%
\mathfrak{s}},  \label{kx_accum}
\end{equation}%
and

\begin{equation}
(1+g)k_{n,t}^{\mathfrak{s}}=i_{n,t}^{\mathfrak{s}}+(1-\delta )k_{n,t-1}^{%
\mathfrak{s}},  \label{kn_accum}
\end{equation}%
where $\beta =1/[(1+\varrho )(1+g)^{(1-\tau )/\tau }]$ is the discount
factor; $\varrho $ is the pure time preference rate; $\tau $ is the
intertemporal elasticity of substitution; $\delta $ is the depreciation
rate; $\mathcal{R}$ are remittances; $\mathcal{T}$ are (net) transfers; $h$
denotes the consumption value added tax (VAT); and $\Phi ^{\mathfrak{s}}$
are profits from domestic firms. Remittances and transfers are proportional
to the agent's share in aggregate employment. Observe that in the budget
constraint (\ref{BCs}), the trend growth rate appears in several places in (%
\ref{BCs})-(\ref{kn_accum}), reflecting the fact that some variables are
dated at $t$ and others at $t$-$1$ and that $P_{t}$ multiplies $b_{t}^{%
\mathfrak{s}}$ and $b_{t-1}^{\mathfrak{s}}$because domestic bonds are
indexed to the price level.\footnote{%
The convention for detrending the capital stocks differs from that for other
variables. Because $K_{j,t-1}^{\mathfrak{s}}$---the capital stock before
detrending---is the capital stock in use at time $t$, we define $k_{j,t-1}^{%
\mathfrak{s}}\equiv K_{j,t-1}^{\mathfrak{s}}/(1+g)^{t}$. Under this
convention, $\bar{\imath}_{j}^{\mathfrak{s}}=(\delta +g)\bar{k}_{j}^{%
\mathfrak{s}}$ in the long run---as required for the capital stock to grow
at the trend growth rate $g$.} Also note that there are adjustment costs
incurred in changing the capital stock---$AC_{j,t}^{\mathfrak{s}}\equiv 
\frac{v}{2}\left( \frac{i_{j,t}^{\mathfrak{s}}}{k_{j,t-1}^{\mathfrak{s}}}%
-\delta -g\right) ^{2}k_{j,t-1}^{\mathfrak{s}},$ for $j=x,n$ and with $v>0$%
---and portfolio adjustment costs associated with foreign liabilities---$%
\mathcal{P}_{t}^{\mathfrak{s}}\equiv \frac{\eta }{2}(b_{t}^{\mathfrak{s}\ast
}-\bar{b}^{\mathfrak{s}\ast })^{2},$ where $\bar{b}^{\ast }$ is the
(initial) steady-state value of the private foreign liabilities.\footnote{%
For simplicity, we assume that adjustment costs are zero when the capital
stock grows at the trend growth rate $g$. This ensures that adjustment costs
are zero across steady states as in models that ignore trend growth.}

The choice variables in the optimization problem are $c_{t}^{\mathfrak{s}}$, 
$b_{t}^{\mathfrak{s}}$, $b_{t}^{\mathfrak{s}\ast }$, $i_{j,t}^{\mathfrak{s}}$%
, and $k_{j,t}^{\mathfrak{s}}$ for $j=x,n$. Routine manipulations of the
first-order conditions deliver: 
\begin{equation}
c_{t}^{\mathfrak{s}}=c_{t+1}^{\mathfrak{s}}\left( \beta \frac{1+r_{t}}{1+g}%
\frac{1+h_{t}}{1+h_{t+1}}\right) ^{-\tau },  \label{Euler Equation}
\end{equation}

\begin{equation}
(1+r_{t})\frac{P_{t+1}}{P_{t}}=\frac{1+r_{t}^{\ast }}{\left[ 1-\eta (b_{t}^{%
\mathfrak{s}\ast }-\bar{b}^{\mathfrak{s}\ast })\right] },  \label{UIP}
\end{equation}%
\begin{equation}
\frac{r_{x,t+1}}{P_{k,t+1}}+1-\delta +v\Upsilon _{x,t+1}^{\mathfrak{s}%
}\left( \frac{i_{x,t+1}^{\mathfrak{s}}}{k_{x,t}^{\mathfrak{s}}}+1-\delta
\right) -\frac{v}{2}\left( \Upsilon _{x,t+1}^{\mathfrak{s}}\right)
^{2}=(1+r_{t})\frac{P_{t+1}}{P_{t}}\frac{P_{k,t}}{P_{k,t+1}}\left(
1+v\Upsilon _{x,t}^{\mathfrak{s}}\right) ,  \label{ix_FOC}
\end{equation}%
and%
\begin{equation}
\frac{r_{n,t+1}}{P_{k,t+1}}+1-\delta +v\Upsilon _{n,t+1}^{\mathfrak{s}%
}\left( \frac{i_{n,t+1}^{\mathfrak{s}}}{k_{n,t}^{\mathfrak{s}}}+1-\delta
\right) -\frac{v}{2}\left( \Upsilon _{n,t+1}^{\mathfrak{s}}\right)
^{2}=(1+r_{t})\frac{P_{t+1}}{P_{t}}\frac{P_{k,t}}{P_{k,t+1}}\left(
1+v\Upsilon _{n,t}^{\mathfrak{s}}\right) ,  \label{in_FOC}
\end{equation}%
where $\Upsilon _{j,t}^{\mathfrak{s}}=\left( \frac{i_{j,t}^{\mathfrak{s}}}{%
k_{j,t-1}^{\mathfrak{s}}}-\delta -g\right) $ for $j=x,n.$ Each of these
equations admits a straightforward intuitive interpretation. Equation (\ref%
{Euler Equation}) is a slightly irregular Euler equation in which the slope
of the consumption path depends on the real interest rate adjusted for trend
growth and on changes in the VAT. The other three equations are arbitrage
conditions. Equation (\ref{UIP}) equalizes the real interest rate on
domestic bonds to the real interest rate on foreign private debt, adjusted
by portfolio costs. Similarly, equations (\ref{ix_FOC}) and (\ref{in_FOC})
require the return on capital in each sector, net of marginal adjustment
costs, to equal the real interest rate.

In our modelling decisions, we balance realism and flexibility in
introducing portfolio adjustment costs to capture different degrees of
integration of the private sector into world capital markets. Equation (\ref%
{UIP}) implicitly defines a private demand for foreign debt, which can be
explicitly expressed as:%
\begin{equation*}
\eta (b_{t}^{\mathfrak{s}\ast }-\bar{b}^{\mathfrak{s}\ast })=1-\frac{%
1+r_{t}^{\ast }}{(1+r_{t})\frac{P_{t+1}}{P_{t}}}.
\end{equation*}%
In this equation, the value of $\eta $ controls the degree of capital
mobility. For some emerging market economies, a low $\eta $ may be
appropriate reflecting an open capital account. Elastic capital flows then
keep the domestic rate close to the foreign rate. In LICs, where $\eta $ is
comparatively big, the capital account is fairly closed, and the private
sector has limited capacity to borrow from abroad.\footnote{\baselineskip%
=12pt From a technical point of view, the portfolio costs also help to
ensure stationarity of $b_{t}^{\mathfrak{s}\ast }.$ See Schmitt-Groh\'{e}
and Uribe (2003) for alternative methods to ensure stationarity of net
foreign assets.}

In addition, we assume that on its foreign debt the private sector pays a
constant premium $\mathfrak{u}$ over the interest rate that the government
pays on external commercial debt $r_{dc}$, so 
\begin{equation*}
r_{t}^{\ast }=r_{dc,t}+\mathfrak{u.}
\end{equation*}%
This specification together with (\ref{UIP}) allows us to match the low
capital and investment ratios observed in LICs. The reason is that at the
steady state the domestic interest rate $r$ and the foreign rate $r^{\ast }$
are equal, meaning that $\bar{r}=\bar{r}_{dc}+\mathfrak{u.}$ Thus even if $%
\bar{r}_{dc}$ is low, by picking an appropriate $\mathfrak{u,}$ we can
obtain a somewhat high domestic real interest rate $\bar{r}$ and,
consequently, a high return on private capital. But because of decreasing
returns, this high return implies realistically that the ratios of the
capital stock and investment to GDP must be low at the initial steady-state
equilibrium.

Non-savers have the same utility function as that of savers and consume all
of their income from wages, remittances, and transfers each period. The
non-saver's budget constraint then reads 
\begin{equation}
(1+h_{t})P_{t}c_{t}^{\mathfrak{h}}=w_{t}L^{\mathfrak{h}}+\frac{a}{1+a}(%
\mathcal{R}_{t}+\mathcal{T}_{t})\text{.}  \label{hm_BC}
\end{equation}

In specifying this part of the model we have aimed for realism combined with
flexibility and generality. The realism of hand-to-mouth consumers
(non-savers) is indisputable given that in LICs a substantial portion of
households fall into this category.\footnote{%
For instance the 2009 Steadman Survey finds that 62 percent of Ugandans do
not have access to financial services.} From the modeling perspective their
inclusion allows us to break Ricardian equivalence.

We aggregate across both types of households, so $x_{t}=x_{t}^{\mathfrak{s}%
}+x_{t}^{\mathfrak{h}}$\ for $x_{t}=c_{t},$ $c_{l,t},$ $L_{t},$ $b_{t}^{\ast
},$ $b_{t},$ $i_{j,t},$ $k_{j,t},$ $AC_{j,t},$ $\mathcal{P}_{t},$ $\Phi
_{t}, $ and the subindices $l=x,n,m$ and $j=x,n.$ Bear in mind that $b_{t}^{%
\mathfrak{h}\ast }=b_{t}^{\mathfrak{h}}=i_{j,t}^{\mathfrak{h}}=k_{j,t}^{%
\mathfrak{h}}=AC_{j,t}^{\mathfrak{h}}=\mathcal{P}_{t}^{\mathfrak{h}}=\Phi
_{t}^{\mathfrak{h}}=0$ for $j=x,n$.

\subsection{The Government}

\subsubsection{Infrastructure, Public Investment and Efficiency}

\quad\ \thinspace Casual observation and indirect empirical evidence support
the conjecture in Hulten (1996) and Pritchett (2000): often the productivity
of infrastructure is high but the return on public investment very low for
the simple reason that a good deal of public investment \textit{spending}
does not increase the stock of productive capital. Measurement error
introduced by equating growth of productive capital with net investment can
explain why estimated TFP growth is zero or negative in many LDCs, as
suggested by Pritchett (2000); and why empirical studies generally find a
much stronger positive relationship between growth and physical indicators
of infrastructure than between growth and capital stock series calculated
via the perpetual inventory method, as reviewed by Straub (2008).\footnote{%
\baselineskip=12pt To understand the role of efficiency, it may be useful to
imagine that all the available public investment projects at a given point
in time are ranked from highest to lowest rate of return. In an efficient
investment process, an additional dollar is spent on the best available
project. It is possible, though, because of incompetence, corruption, or
imperfect information, that a government may choose worse projects. A lower
efficiency is a measure of the degree of deviation from the optimal process.
A complementary way to think about efficiency is simply that a fraction of
spending is simply wasted, e.g. misclassified as investment when it in fact
just covers transfers to civil servants.}

We thus allow for inefficiencies in public capital creation. Public
investment $i_{z}$ produces additional infrastructure $z$ according to:%
\begin{equation}
(1+g)z_{t}=(1-\delta )z_{t-1}+i_{z,t},  \label{z_accum}
\end{equation}%
but some of the newly built infrastructure may not be economically valuable
productive infrastructure, since \textit{effectively productive} capital $%
z_{t}^{e},$ which is actually used in technologies (\ref{qx}) and (\ref{qn}%
), evolves according to: 
\begin{equation}
z_{t}^{e}=\bar{s}\bar{z}+s(z_{t}-\bar{z}),\quad \text{with}\quad \bar{s}\in
\lbrack 0,1]\quad \text{and}\quad s\in \lbrack 0,1],  \label{z efficiency}
\end{equation}%
where $\bar{s}$ and $s$ are parameters of efficiency at and off steady
state, and $\bar{z}$ is public capital at the (initial) steady state.

Note that by combining equations (\ref{z_accum}) and (\ref{z efficiency}),
we obtain%
\begin{equation}
(1+g)z_{t}^{e}=(1-\delta )z_{t-1}^{e}+s(i_{z,t}-\bar{\imath}_{z})+\bar{s}%
\bar{\imath}_{z},  \label{I_efficiency}
\end{equation}%
where $\bar{\imath}_{z}=(\delta +g)\bar{z}$ is the public investment at the
(initial) steady state$.$ This is the same specification for public
investment inefficiencies as that of Berg et al.\thinspace (2010b), and it
is similar to the one in Agenor (2010). Since $s\in \lbrack 0,1],$ this
specification makes clear that \textit{one} dollar of additional public
investment $(i_{z,t}-\bar{\imath}_{z})$ does not translate into \textit{one}
dollar of effectively productive capital ($z_{t}^{e}$). For the simulations
below, we assume $\bar{s}=0.6$ and $s=0.6.$ These values are slightly higher
that the estimates of Arestoff and Hurlin (2006) for some emerging
economies. In principle, the public investment management quality index of
Dabla-Norris et al. (2011) could help calibrate this parameter.

\subsubsection{Fiscal Adjustment and the Public Sector Budget Constraint}

\quad\ \thinspace The government spends on transfers, debt service, and
infrastructure investment. It collects revenue from the consumption VAT and
from user fees for infrastructure services, which are expressed as a fixed
multiple/fraction $f$ of recurrent costs, that is $\mu =f\delta P_{zo}$.
When revenues fall short of expenditures, the resulting deficit is financed
through domestic borrowing $\Delta b_{t}=b_{t}-b_{t-1},$ external
concessional borrowing $\Delta d_{t}=d_{t}-d_{t-1},$ or external commercial
borrowing $\Delta d_{c,t}=d_{c,t}-d_{c,t-1}$ viz.: 
\begin{eqnarray}
P_{t}\Delta b_{t}+\Delta d_{c,t}+\Delta d_{t} &=&\frac{r_{t-1}-g}{1+g}%
P_{t}b_{t-1}+\frac{r_{d,t-1}-g}{1+g}d_{t-1}+\frac{r_{dc,t-1}-g}{1+g}d_{c,t-1}
\label{Gov_BC} \\[0.08in]
&&+P_{z,t}\mathbb{I}_{z,t}+\mathcal{T}_{t}-h_{t}P_{t}c_{t}-\mathcal{G}_{t}-%
\mathcal{N}_{t}-\mu z_{t-1}^{e},  \notag
\end{eqnarray}%
where $d$, $dc$, $\mathcal{G}$, and $\mathcal{N}$ denote concessional debt,
external commercial debt, grants, and natural resource revenues (if any);%
\footnote{%
We model natural resources revenues as a net foreign transfer, following
Dagher et al. (2012). As such \ the measure of GDP used below corresponds to
non-oil GDP. For a more comprehensive analysis of oil production, foreign
investment, and managing natural resources in LICs see Berg et al. (2012).}
and $r_{d}$ and $r_{dc}$ are the real interest rates (in dollars) on
concessional and commercial loans. The interest rate on concessional loans
is assumed to be constant $r_{d,t}=r_{d},$ while the interest rate on
external commercial debt incorporates a risk premium that depends on the
deviations of the external public debt to GDP ratio $\left( \frac{%
d_{t}+d_{c,t}}{y_{t}}\right) $ from its (initial) steady-state value $\left( 
\frac{\bar{d}+\bar{d}_{c}}{\bar{y}}\right) $. That is, 
\begin{equation}
r_{dc,t}=r^{f}+\upsilon _{g}e^{\eta _{g}\left( \frac{d_{t}+d_{c,t}}{y_{t}}-%
\frac{\bar{d}+\bar{d}_{c}}{\bar{y}}\right) },  \label{rdc}
\end{equation}%
where $r^{f}$ is a risk-free world interest rate and $%
y_{t}=P_{x,t}q_{x,t}+P_{n,t}q_{n,t}$ is GDP. Van der Ploeg and Venables
(2011) provide positive estimates for $\eta _{g}$. But note by setting $%
\upsilon _{g}>0$ and $\eta _{g}=0,$ our specification embeds the case of an
exogenous risk premium that does depend on public debt.

The term $P_{z,t}\mathbb{I}_{z,t}$ in the budget constraint (\ref{Gov_BC})
corresponds to public investment outlays including costs overruns associated
with absorptive capacity constraints. It is defined as%
\begin{equation*}
\mathbb{I}_{z,t}=\mathcal{H}_{t}(i_{z,t}-\bar{\imath}_{z})+\bar{\imath}_{z}.
\end{equation*}%
Because skilled administrators are in scarce supply in LICs, ambitious
public investment programs are often plagued by poor planning, weak
oversight, and myriad coordination problems, all of which contribute to
large cost overruns during the implementation phase.\footnote{%
Development agencies report that cost overruns of 35\% and more are common
for new projects in Africa. The most important factor by far is inadequate
competitive bidding for tendered contracts. See Foster and Briceno-Garmendia
(2010).} To capture this, we multiply new investment $(i_{z,t}-\bar{\imath}%
_{z})$ by $\mathcal{H}_{t}=\left( 1+\frac{i_{z,t}}{z_{t-1}}-\delta -g\right)
^{\phi }$, where $\phi \geq 0$ determines the severity of the the absorptive
capacity---or \textquotedblleft bottleneck\textquotedblright ---constraint
in the public sector. The constraint affects only implementation costs for
new projects: in a steady state, $\left( 1+\frac{\bar{\imath}_{z}}{\bar{z}}%
-\delta -g\right) ^{\phi }=1$ as $\bar{\imath}_{z}=(\delta +g)\bar{z}$.

Policy makers accept all concessional loans proffered by official creditors.
The borrowing and amortization schedule for these loans is fixed
exogenously. Given the paths for public investment and concessional
borrowing, the fiscal gap \textit{before }policy adjustment ($\mathfrak{Gap}%
_{t}$) can be defined as:%
\begin{equation}
\mathfrak{Gap}_{t}=\frac{1+r_{d}}{1+g}d_{t-1}-d_{t}+\frac{r_{dc,t-1}-g}{1+g}%
dc_{t-1}+\frac{r_{t-1}-g}{1+g}P_{t}b_{t-1}+P_{z,t}\mathbb{I}_{t}+\mathcal{T}%
_{o}-h_{o}P_{t}c_{t}-\mathcal{G}_{t}-\mathcal{N}_{t}-\mu z_{t-1}^{e}.
\label{gapdef}
\end{equation}%
That is, $\mathfrak{Gap}_{t}$ corresponds to expenditures (including
interest rate payments on debt) less revenues and concessional borrowing,
when transfers and taxes are kept at their \textit{initial} levels $\mathcal{%
T}_{o}$and $h_{o}$, respectively. Using this definition, we can rewrite the
budget constraint (\ref{Gov_BC}), in any given year, as: 
\begin{equation}
\mathfrak{Gap}_{t}=P_{t}\Delta b_{t}+\Delta d_{c,t}+(h_{t}-h_{o})P_{t}c_{t}-(%
\mathcal{T}_{t}-\mathcal{T}_{o}).  \label{exgap}
\end{equation}

In the short/medium run, this gap $\mathfrak{Gap}_{t}$ in (\ref{exgap}) can
be covered by domestic and/or external commercial borrowing $P_{t}\Delta
b_{t}+\Delta d_{c,t}$, tax adjustments $(h_{t}-h_{o})P_{t}c_{t},$ and/or
transfers adjustments $-(\mathcal{T}_{t}-\mathcal{T}_{o}).$ For the sake of
comparing different borrowing schemes, in the experiments below, we will
focus on cases where part of this gap can be filled with either external
commercial loans \textit{or} domestic loans, but not with both at the same
time.

Debt sustainability requires, however, that the VAT and transfers eventually
adjust to cover the entire gap (i.e., $P_{t}\Delta b_{t}+\Delta d_{c,t}=0$).
We let policy makers divide the burden of adjustment (net windfall when $%
\mathfrak{Gap}_{t}<0$) between transfers cuts and tax increases. The
adjustments, defined below according to some reaction functions, have as
part of their targets the following debt-stabilizing values for transfers
and the VAT 
\begin{equation}
{\small h}_{t}^{\text{target}}{\small =h}_{o}+(1-\lambda )\frac{\mathfrak{Gap%
}_{t}}{P_{t}c_{t}}  \label{h_target}
\end{equation}%
and%
\begin{equation}
\mathcal{T}_{t}^{\text{target}}=\mathcal{T}_{o}-\lambda \mathfrak{Gap}_{t},
\label{T_target}
\end{equation}%
where $\lambda $ is a policy parameter that splits the fiscal adjustment
between taxes and transfers and therefore satisfies $0\leq \lambda \leq 1.$
For the extreme case of $\lambda =0$ (respectively $\lambda =1$) all the
adjustment falls on taxes (respectively transfers).

Taxes and transfers are defined according to the following reaction
functions:%
\begin{equation}
h_{t}=Min\left\{ h_{t}^{r},h^{u}\right\}  \label{hmin}
\end{equation}%
and%
\begin{equation}
\mathcal{T}_{t}=Max\left\{ \mathcal{T}_{t}^{r},\mathcal{T}^{l}\right\} ,
\label{Tmax}
\end{equation}%
where $h^{u}$ is a ceiling on taxes, $\mathcal{T}^{l}$ is a floor for
transfers, and $h_{t}^{r}$ and $\mathcal{T}_{t}^{r}$ are determined by the
fiscal rules%
\begin{equation}
h_{t}^{r}=h_{t-1}+\lambda _{1}({\small h}_{t}^{\text{target}%
}-h_{t-1})+\lambda _{2}\frac{(x_{t-1}-x^{\text{target}})}{y_{t}},\quad \text{%
with}\quad \text{ }\lambda _{1},\lambda _{2}>0,  \label{h_reaction}
\end{equation}%
and%
\begin{equation}
\mathcal{T}_{t}^{r}=\mathcal{T}_{t-1}+\lambda _{3}(\mathcal{T}_{t}^{\text{%
target}}-\mathcal{T}_{t-1})-\lambda _{4}(x_{t-1}-x^{\text{target}}),\quad 
\text{with}\quad \text{ }\lambda _{3},\lambda _{4}>0,  \label{T_reaction}
\end{equation}%
with $y=P_{n}q_{n}+P_{x}q_{x}$ and $x=b$ or $d_{c},$ depending on whether
the rules respond to domestic debt or commercial debt. The target for debt $%
x^{\text{target}}$ is given exogenously. The ceiling $h^{u}$ on taxes and
the floor $\mathcal{T}^{l}$ on transfers can also have discrete jumps over
time, as we show below.\footnote{%
For instance, to introduce discrete jumps in the cap on taxes we can
respecify the rule as $h_{t}=Min\left\{ h_{t}^{r},h_{t}^{u}\right\} $, where 
$h_{t}^{u}=h_{o}^{u}+\Delta h_{t}^{u},$ $h_{o}^{u}$ is the initial cap and $%
\Delta h_{t}^{u}$ is a discrete jump (shock) at time $t,$ which can be
temporary or permanent.} This allows us to model staggered tax and transfers
structures.

Given the targets, the reaction functions defined in (\ref{hmin})-(\ref%
{T_reaction}) together with the budget constraint (\ref{exgap}) embody the
core policy dilemma. Fiscal adjustment is painful, especially when
administered suddenly in large doses. The government would prefer therefore
to phase-in tax increases and expenditure cuts slowly ($\lambda _{1}>0$ and $%
\lambda _{3}>0$). Since fiscal adjustment is gradual, the debt instrument
that varies endogenously to satisfy the government budget constraint may
rise above its target level in the time it takes $h_{t}$ and $\mathcal{T}%
_{t} $ to reach ${\small h}_{t}^{\text{target}}$ and $\mathcal{T}_{t}^{\text{%
target}}$. When this happens, the transition path includes a phase in which $%
\mathcal{T}_{t}<\mathcal{T}_{t}^{\text{target}}$ and/or $h_{t}>{\small h}%
_{t}^{\text{target}}$ to generate the fiscal surpluses needed to pay down
the debt. In addition, and despite the fact that the rules respond to debt
(i.e., $\lambda _{2}>0$ and $\lambda _{4}>0$), if the government moves too
slowly (i.e., $\lambda _{1}$ and $\lambda _{3}$ are too low), or if the
bounds $h^{u}$ and $\mathcal{T}^{r}$ constrain adjustment too much, interest
payments will rise faster than revenue net of transfers, causing the debt to
grow explosively. Large debt-financed increases in public investment are
undeniably risky---the economy converges to a stationary equilibrium only if
policy makers win the race against time.

\subsection{Market-Clearing Conditions and External Debt Accumulation}

\quad\ \thinspace Flexible wages and prices ensure that demand continuously
equals supply in the labor market 
\begin{equation}
L_{x}+L_{n}=L,  \label{L_mkt}
\end{equation}%
where the labor supply $L=L^{\mathfrak{s}}+L^{\mathfrak{h}}$ is fixed.

In the non-tradables market, after aggregating across types of consumers, we
obtain 
\begin{equation}
q_{n,t}=\rho _{n}\left( \frac{P_{n,t}}{P_{t}}\right) ^{-\epsilon
}c_{t}+a_{k}\left( i_{x,t}+i_{n,t}+AC_{x,t}+AC_{n,t}\right) +a_{z}\mathbb{I}%
_{z,t}.  \label{NT_mkt}
\end{equation}%
The first term of the right-hand side of (\ref{NT_mkt}) is the demand for
non-traded consumer goods, while the second and third terms link public and
private investment to orders for new capital goods.

Finally, aggregating across consumers and adding the public and private
sector budget constraints produce the accounting identity that growth in the
country's net foreign debt equals the difference between national spending
and national income: 
\begin{eqnarray}
d_{t}-d_{t-1}+d_{c,t}-d_{c,t-1}+b_{t}^{\ast }-b_{t-1}^{\ast } &=&\frac{%
r_{d}-g}{1+g}d_{t-1}+\frac{r_{dc,t-1}-g}{1+g}d_{c,t-1}+\frac{r_{t-1}^{\ast
}-g}{1+g}b_{t-1}^{\ast }  \label{CA_Eq} \\
&&+\mathcal{P}_{t}+P_{z,t}\mathbb{I}_{z,t}+P_{k,t}\left(
i_{x,t}+i_{n,t}+AC_{x,t}+AC_{n,t}\right)  \notag \\
&&+P_{t}c_{t}-P_{n,t}q_{n,t}-P_{x,t}q_{x,t}-\mathcal{R}_{t}-\mathcal{G}_{t}-%
\mathcal{N}_{t}.  \notag
\end{eqnarray}%
Equation (\ref{CA_Eq}) includes extra terms that reflect the impact of trend
growth on real interest costs and the contributions of remittances, grants,
and natural resource related foreign transfers to gross income. The textbook
identity emerges when $g=\mathcal{R}_{t}=\mathcal{G}_{t}=\mathcal{N}_{t}=0$.

\end{document}
